\documentclass[../INDE250HW.tex]{subfiles} 
\usepackage{fancyhdr}
\usepackage{graphicx}
\usepackage{amsmath}
\usepackage{tikz}
\usepackage{amssymb}
\usepackage[margin=1in]{geometry}

\title{UW IND E 250 Notes}
\author{Anthony Le}

\newtheorem{exmp}{Example}
\newtheorem{exrc}{Excersize}
\newtheorem{proof}{Proof}
\newtheorem{defn}{Definition}


\begin{document}

\pagestyle{fancy}
\fancyhead{}
\fancyhead[R]{UW IND E 250}
\fancyhead[L]{Anthony Le}

\begin{center}
    \LARGE{\textbf{IND E 250 HW 3}}
\end{center}
\begin{center}
    Homework 3 Problems: \\
    11.4, 11.9, 5.6, 5.10, 5.30, 5.33 (bonus)
\end{center}
\section*{Problem 11.4}
\begin{exrc}
   For prices that are increasing at an annual rate of 5\% the first year and 8\% the second year determine the average inflation rate ($\bar{f}$) over the two years.
\end{exrc}
\begin{enumerate}
    \item Finding $CPI_1$ for year 1 (Assuming $CPI_0 = \$100$)
        \begin{equation*}
            \begin{aligned}
                CPI_1 &= CPI_0(1 + \bar{f_1})^n \\
                        &= 100(1 + 0.05)^1 \\
                CPI_1 &= 105
            \end{aligned}
        \end{equation*}
    \item Finding $CPI_2$ for year 2 (Assuming $CPI_0 = \$100$)
        \begin{equation*}
            \begin{aligned}
                CPI_2 &= CPI_1(1 + \bar{f_2})^n \\
                        &= 105(1 + 0.08)^1 \\
                CPI_2 &= 113.4
            \end{aligned}
        \end{equation*}
    \item Finding average inflation rate ($\bar{f}$) over the two years
        \begin{equation*}
            \begin{aligned}
                \bar{f} &= \left[\frac{CPI_2}{CPI_1}\right]^{1/n} - 1 \\
                        &= \left[\frac{113.4}{105}\right]^{1/2} - 1 \\
                \bar{f} &= \framebox{$6.49\%$}
            \end{aligned}
        \end{equation*}
\end{enumerate}

\newpage
\section*{Problem 11.9}
\begin{exrc}
    The purchase of a car requires a \$25000 loan to be repaid in monthly installments for four years at 9\% interest compounded monthly. If the general inflation rate is 4\% compounded monthly, find the actual and constant dollar value of the 20th payment.
\end{exrc}
\begin{enumerate}
    \item Converting annual interest rate to monthly interest rate:
        \begin{equation*}
            \begin{aligned}
                i = 9\%/12months = 0.75\%
            \end{aligned}
        \end{equation*}
    \item Converting annual inflation rate to monthly inflation rate:
        \begin{equation*}
            \begin{aligned}
                f = 4\$/12months = 0.33\%
            \end{aligned}
        \end{equation*}
    \item Finding actual dollar value of 20th payment %Note - it seems like so far that we have been working with the actual dollar values so far; finding annunity payments for 4 years
        \begin{equation*}
            \begin{aligned}
                A_{20} &= P(A/P,i,N) \\
                        &= 25000(A/P, 0.0075\%,4*12) \\
                A_{20} &= 25000(\frac{0.0075(1+0.0075)^{48}}{(1+0.0075)^{48}-1}) \\
                A_{20} &= \framebox{$\$622.13$}
            \end{aligned}
        \end{equation*}
    \item Finding constant dollar value of 20th payment %Note - the constant dollar value is the actual dollar value but taking into account the affects of inflation
        \begin{equation*}
            \begin{aligned}
                A'_{20} &= A_{20}(P/F,\bar{f},N) \\
                        &= \$622.13(P/F,0.33\%,20) \\
                        &= \$622.13(1 + 0.0033\%)^{-20}\\
                A'_{20} &= \framebox{$\$582.46$}
            \end{aligned}
        \end{equation*}
\end{enumerate}

\newpage
\section*{Problem 5.6}
\begin{exrc}
    A project costs \$120000 and the expected annual returns are as given in the table below:
    \begin{center}
        \begin{tabular}{ c c }
            Year & Cash Flows \\
            \hline
            1   &   \$18,500 \\
            2   &   \$25,500 \\
            3   &   \$27,980 \\
            4   &   \$32,660 \\
            5   &   \$40,230
        \end{tabular}
    \end{center}
    \begin{enumerate}
        \item What is the payback period of the project?
        \item What is the discounted payback period at an interest rate of 15\%?
    \end{enumerate}
\end{exrc}
\begin{enumerate}
    \item Finding cash flows:
        \begin{equation*}
            \begin{aligned}
                \text{Year 0:} \quad -\$120000 \\
                \text{Year 1:} \quad -\$120000 + \$18500 = -\$101500 \\
                \text{Year 2:} \quad -\$101500 + \$25500 = -\$76000 \\
                \text{Year 3:} \quad -\$76000 + \$27980 = -\$48020 \\
                \text{Year 4:} \quad -\$48020 + \$32660 = -\$15360 \\
                \text{Year 5:} \quad -\$15360 + \$40230 = \$24870
            \end{aligned}
        \end{equation*}
    \item Year 4 is the last year with negative cash flow, finding payback period:
        \begin{equation*}
            \begin{aligned}
                &4 + \frac{\$15360}{\$40230} \\
                &\framebox{$4.382\text{ years}$} 
            \end{aligned}
        \end{equation*}
    \item Finding discounted payback period: \\
        \begin{tabular}{ c c c c }
            Year & Cash Flows & Cost of Funds & Ending Balance\\
            \hline
            0   &   -\$120000   &   0   &   -\$120000 \\   
            1   &   \$18,500    &   -\$120000*0.15 = -\$18000   &   -\$119500 \\
            2   &   \$25,500    &   -\$119500*0.15 = -\$17925   &   -\$111925 \\
            3   &   \$27,980    &   -\$111925*0.15 = -\$16788.75    &   -\$100733.75 \\
            4   &   \$32,660    &   -\$100733.75*0.15 = -\$15110.06 &   -\$83183.81\\
            5   &   \$40,230    &   -\$83183.81*0.15 = -\$12477.57 & -\$55341.38
        \end{tabular}
        \newline
        \fbox{Assuming cash flows end at year 5, discounted payback period goes to infinity}
\end{enumerate}

\newpage
\section*{Problem 5.10}
\begin{exrc}
    You need to know whether the building of a new warehouse is justified under the following conditions:
    \begin{enumerate}
        \item The proposal is for a warehouse costing \$250,000.
        \item The warehouse has an expected useful life of 35 years and a net salvage value (net proceeds from sale after tax adjustments) of \$50,000.
        \item Annual receipts of \$67,000 are expected, annual maintenance and administrative costs will be \$12,000/year, and annual income taxes are \$15,000.
    \end{enumerate}
    Given the foregoing data, which of the following statements are correct?
    \begin{enumerate}
        \item The proposal is justified for a MARR of 15\%.
        \item The proposal has a net present worth of -\$50,254 when 20\% is used as the interest rate.
        \item The proposal is acceptable, as long as MARR $\leq$ 15.93\%.
        \item All of the preceding are correct.
    \end{enumerate}
\end{exrc}
\fbox{Statement 4 - All of the preceding are correct.}

\newpage
\section*{Problem 5.30}
\begin{exrc}
    Consider the project balance profiles shown in the table below for proposed investment projects.
    \begin{center}
        \begin{tabular}{ c c c c }
            n   &   A   &   B   &   C \\
            \hline
            0   &   -\$1000   &   -\$1000   &   -\$1000 \\
            1   &   -1000   &   -650   &   -1200 \\
            2   &   -900   &   -348   &   -1440 \\
            3   &   -690   &   -100   &   -1328 \\
            4   &   -359   &   85   &   -1194 \\
            5   &   105   &   198   &   -1000 \\
            Interest Rate    &      &      &     \\
            used   &   \fbox{10\%}   &   \fbox{?}   &   \fbox{20\%} \\
            NPW   &   \fbox{?}   &   \fbox{\$79.57}   &   \fbox{?} \\ 
        \end{tabular}
    \end{center}
    Project balance figures are rounded to nearest dollars.
    \begin{enumerate}
        \item Compute the net present worth of projects A and C.
        \item Determine the cash flows for project A.
        \item Identify the net future worth of project C.
        \item What interest rate would be used in the project balance calculations for project B?
    \end{enumerate}
\end{exrc}
\begin{enumerate}
    \item Compute the net present worth of projects A and C. \\
    \begin{center}
        \framebox{$NPW_A = 65.19 \quad NPW_C = -401.90$}
    \end{center}
    \item Determine the cash flows for project A. 
    \begin{center}
        \begin{tabular}{ c c }
            Year & Cash Flows \\
            \hline
            0   &   -\$1000 \\
            1   &   \$0 \\
            2   &   \$200 \\
            3   &   \$300 \\
            4   &   \$400 \\
            5   &   \$290
        \end{tabular}
    \end{center}
    \item Identify the net future worth of project C.
    \begin{center}
        \framebox{$FW_C = -1000$}
    \end{center}
    \item What interest rate would be used in the project balance calculations for project B?
    \begin{center}
        \framebox{$i = 20\%$}
    \end{center}
\end{enumerate}


\end{document}