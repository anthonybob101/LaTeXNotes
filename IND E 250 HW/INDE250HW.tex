\documentclass{report} % //TODO find why latex gives errors for report and using textsc!
\usepackage{fancyhdr}
\usepackage{graphicx}
\usepackage{amsmath}
\usepackage{tikz}
\usepackage{amssymb}
\usepackage[margin=1in]{geometry}


\usepackage{subfiles} % Best loaded last in the preamble


\title{UW IND E 250 Notes}
\author{Anthony Le}

\newtheorem{exmp}{Example}
\newtheorem{exrc}{Excersize}
\newtheorem{proof}{Proof}
\newtheorem{defn}{Definition}


\begin{document}

\pagestyle{fancy}
\fancyhead{}
\fancyhead[R]{UW IND E 250}
\fancyhead[L]{Anthony Le}

\subfile{Subfiles/INDE250_HW1}
\newpage
\subfile{Subfiles/INDE250_HW2}
\newpage
\subfile{Subfiles/INDE250_HW3}
\newpage
\begin{center}
    \LARGE{\textbf{IND E 250 HW 4}}
\end{center}
\begin{center}
    Homework 4 Problems: \\
    5.49, 5.51, 6.11, 6.20, 6.31, 6.43 (bonus)
\end{center}

\section*{Problem 5.49}
\begin{exmp}
    Consider the two mutually exclusively projects in the table below.

    Salvage values represent the net proceeds (after tax) from disposal of the assets if they are sold at the end of each year. Both projects B1 and B2 will be available (or can be repeated) with the same costs and salvage values for an indefinite period. 
    \begin{enumerate}
        \item Assuming an infinite planning horizon, which project is a better choice at MARR = 12\%?
    \end{enumerate}
    
        %//TODO - Insert table here!

\end{exmp}



\section*{Problem 5.51}
\begin{exmp}
    Consider the cash flows for two types of models given in the table below. 
    %//TODO - Insert table here!

    Both models will have no savage value upon their disposal (at the end of their respective service lives). The firm's MARR is known to be 12\%.

    \begin{enumerate}
        \item Notice that the models have different service lives. However, model A will be available in the future with the same cash flows. Model B is available at one time only. If you select model B now, you will have to replace it with model A at the end of year 2. If your firm uses the present worth as a decision criterion, which model should be selected, assuming that the firm will need either model for an indefinite period?
        \item Suppose that your firm will need either model for only two years. Determine the salage value of model A at the end of year 2 that makes both models indifferent (equally likely).
    \end{enumerate}
\end{exmp}


\section*{Problem 6.11}
\begin{exmp}
    Beginning next year, a foundation will support an annual seminar on ca mpus by the earnings of a \$200,00 gift it recieved this year. It is felt that 6\% interest will be realized for the first 10 years, but that plans should be made to anticipate an interest rate of 4\% after that time. What amount should be added to the foundation now to fund the seminar at the \$20,000 level into infinity?
\end{exmp}

\section*{Problem 6.20}
\begin{exmp}
    The cash flows for two investment projects are as given in the table below:
    %//TODO - Insert table here!
    \begin{enumerate}
        \item For project A, find the value of X that makes the equivalent annual receipts equal the equivalent annual disbursement at i = 15\%.
        \item Would you accept project B at i = 12\% based on an AE criterion
    \end{enumerate}
\end{exmp}

\section*{Problem 6.31}
\begin{exmp}
    A company is currently paying its employees \$0.55 per mile to drive their own cars on company business. The company is considering supplying employees with cars, which would involve purchasing at at \$25,000 with an estimated three-year life, a net salvage at \$8,000, taaxes and insurance at a cost \$1,200 per year, and operating and maintenance expenses of \$0.30 per mile. If the interest rate is 10\% and the company anticipates an employee's annual travel to be 30,000 miles, what is the equivalent cost per mile (neglecting income taxes)?
\end{exmp}


\end{document}

