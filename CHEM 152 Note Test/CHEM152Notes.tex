\documentclass{article}  % //TODO find why latex gives errors for report and using textsc!
\usepackage{fancyhdr}
\usepackage{graphicx}
\usepackage{amsmath}
\usepackage{mhchem}
\usepackage{amssymb}
\usepackage[margin=1in]{geometry}

\usepackage{subfiles} % Best loaded last in the preamble

\title{UW CHEM 152 Notes}
\author{Anthony Le}

\newtheorem{exmp}{Example}
\newtheorem{exrc}{Excersize}
\newtheorem{proof}{Statement}
\newtheorem{defn}{Definition}


\begin{document}

\pagestyle{fancy}
\fancyhead{}
\fancyhead[R]{UW CHEM 152}
\fancyhead[L]{Anthony Le}

\subfile{Subfiles/CHEM152_Chapter6}

\newpage

\subfile{Subfiles/CHEM152_Chapter7}

\newpage

\subfile{Subfiles/CHEM152_Chapter8.tex}

\newpage

\section*{Chapter 9.1-9.3 - Energy and Enthalpy}

\subsection*{What is Thermodynamics?}
\begin{enumerate}
    \item Imagine you want to describe the air in this room
    \begin{enumerate}
        \item A (classical) physicist might try to tell you the positions and velocities of every single atom (the classical "state" of the system)\footnote{\textbf{TECHNICALLY}, you can do this, but this is in practice impossible}
        \item A (quantum) physicist might try to tell you the waveform of all atoms (the quantum "state" of the system)\footnote{\textbf{TECHNICALLY}, you can do this, but this is in practice EVEN MORE impossible}
    \end{enumerate}
    \begin{enumerate}
        \item A reasonable person will tell you the pressure is 1 atmosphere and the temperature is 25C\footnote{Much easier to do.}
    \end{enumerate}
    \item Points 1a and 1b represent microscopic desciptions of nature - describing the universe on a microscopic level!
    \item Point 1c represents the macroscopic description of nature
    \item Thermodynamics is a macroscopic description of the universe in terms of a small number of readily observable properties - temperature, pressure, volume, etc etc.
    \item It (mostly) ignores the fact that matter is made of atoms and molecules. It doesn't need it! It's built around very few \textbf{laws of thermodynamics}. 
\end{enumerate}
$\left.
    \begin{tabular}{ll}
        (a) Cock \\
        (b) Cum
    \end{tabular}
\right\}$  = text
    
\subsection*{Energy of a Molecular System}
\begin{enumerate}
    \item Kinetic Energy ($E_g$) - Energy associated with \emph{motion} \\
    Different types of motion:
        \begin{enumerate}
            \item Translation - moving from one point to another.
            \item Rotation - motion about the center of mass\footnote{Technically, an atom cannot rotate due to how its symmetric, but a molecule can creating chemical energy!}
            \item Vibration - motion directed through chemical bonds
        \end{enumerate}
    \item Potential Energy ($E_p$) - Energy associated with \emph{interactions} \\
    In molecular systems, the most important interaction is usually the electrostatic interaction. This includes: \footnote{This is important in water due to how it exists BECAUSE of the \textbf{intra}molecular and \textbf{inter}molecular interactions between water molecules!}
        \begin{enumerate}
            \item Interactions within a molecule (\textbf{intra}molecular)
            \item Interactions between molecules (\textbf{inter}molecular)
        \end{enumerate}
    \item The total energy of a system is the sum of ineic and potential energy: $E = E_k + E_p$
    \item The unit of energy is Joules(J) - where $1J = 1kg*m^2/s^2$
    \item Also, calories are also another unit of energy - $1cal = 4.184J$\footnote{These calories that we deal with in food are technically not "calories" per se, its actually kilocalories}
    \item Also, electronvolts are another unit of energy, suited for the small interactions from chem - $1 eV = 1.6*10^{-19}J$
\end{enumerate}

\subsection*{Conservation of Energy}
\begin{enumerate}
    \item Different forms of energy can be converted into each oher (for example, kinetic and potential energy). But energy cannot be created or destroyed, thus
    \begin{proof}
        The total amount of energy in the universe is constant.
    \end{proof}
    \item Thinking about the entire universe is hard, Let's partition the universe into the "system" we're interested in and everything else (the "surroundings"):
        \begin{enumerate}
            \item The energy of the system can change if there is an equal but opposie change in energy of the suroundings.
            \item In other words: energy can flow between the system and its surroundings
        \end{enumerate}
\end{enumerate}

\subsection*{State and State Changes}
\begin{enumerate}
    \item The energy of the system depends on the current state of the system (we say that "energy is a state function"). Look at energy as a property describing the state of the system as a whole
        \begin{enumerate}
            \item Continuing this logic to an example - the state of a gaseous system is determined by pressure, volum , amount of sybstance, and temperature.
        \end{enumerate}
    \item The energy of the system \emph{does not} depend on how the system was brough into this state.
        \begin{enumerate}
            \item Let us consider a process in which the sate of the system changes from an inital state to a final sstate (for example, we change the temperature of a gas while keeping the volume and substance amount constant)
            \item The change in the system's energy $\Delta E$ is the difference between the energies of the final and inital state:
                \begin{equation*}
                    \begin{aligned}
                        \Delta E = E_{final} - E_{inital}
                    \end{aligned}
                \end{equation*}
            \item Note the sign convention here!
                \begin{equation*}
                    \begin{aligned}
                        \Delta E > 0 \quad \text{System Energy Increases} \\
                        \Delta E < 0 \quad \text{System Energy Decreases} \\
                    \end{aligned}
                \end{equation*}
            \item $\Delta E$ depends only on what the inital and final \emph{states} are, but \textbf{not} on what the \textbf{process} was that led to the change in state!
        \end{enumerate}
\end{enumerate}

\subsection*{First Law of Thermodynamics}
\begin{enumerate}
    \item We already know that if the system energy changes by $\Delta E$, there must be an equal but opposite change $-\Delta E$ in the energy of the surroundings.
    \item Energy flows from the system to the surroundings ($\Delta E < 0$) or from the surroundings to the system ($\Delta E > 0$)
    \item As we'll see, the flow of energy between the system and the surroundings can be seperated into two contributions - \emph{heat (q)} and \emph{work (w)}: \\
    \emph{The First Law of Thermodynamics}
        \begin{equation*}
            \begin{aligned}
                \Delta E = q + w 
            \end{aligned}
        \end{equation*}
    \item Our sign convention:\footnote{NOTE - some books \textbf{and Wikipedia} use the opposite sign convention for the work (w)!!!}
        \begin{enumerate}
            \item q > 0: heat flows from surroundings to system (system energy increases) 
            \item q < 0: heat flows from the system to surroundings (system energy decreases)
            \item w > 0: work is performed \emph{on} system (system energy increases)
            \item w < 0: work is performed \emph{by} system
        \end{enumerate}
\end{enumerate}

\subsection*{Exothermic Processes}
In an exothermic process, heat is released \emph{by} the system into the surroundings (q < 0).
\begin{enumerate}
    \item Physical Example - cooling water releases heat to the surroundings. \\
    Here heat loss is primarily due to the decrease in molecular kinetic energy due to the water molecules slowing down.
    \item Chemical Example - oxidation of glycerol $\ce{C3H8O3}$ by $\ce{KMnO4}$ releases heat to the surroundings.
    Here heat loss is primarily due to the decrease in molecular potential energy due to the molecules reacting to each other.
\end{enumerate}

\subsection*{Endothermic Processes}
In an endothermic process, heat is absorbed by the system from the surroundings (q > 0).
\begin{enumerate}
    \item Physical Example - heating water absorbs heat from the surroundings. \\
    Here heat gain is primarily due to the increase in molecular kinetic energy due to the water molecules slowing down.
    \item Chemical Example - mixure of barium hydroxide ($\ce{Ba(OH)2}$) and ammonium chloride ($NH4Cl$) absorbs heat from the surroundings.
    Here heat gain is primarily due to the increase in molecular potential energy due to the molecules reacting to each other.
\end{enumerate}

\subsection*{Pressure - Volume Work}
\begin{enumerate}
    \item How do chemical systems exchange energy as work?
    \item \emph{Motivating Example} - heat is applied to the air in a hot-air balloon, causing the volume of gas to increase
    \item As the gas expands, it pushes back the atmosphere ... \emph{it performs work on the atmosphere}
\end{enumerate}
The expansion of gas against a constant expternal pressue is a common type of work in chemical processes. This is called "pressure-volume" (PV) work. 
\newline
How much work does the system do to push against a constant external pressure? 
\newline
\begin{enumerate}
    \item Knowing work is the product of force and distance:
        \begin{equation*}
            \begin{aligned}
                w = F x \Delta h
            \end{aligned}
        \end{equation*}
    \item The force we are pushing against comes from the external pressure:
        \begin{equation*}
            \begin{aligned}
                F = P x A
            \end{aligned}
        \end{equation*}
    \item The work required to expand the gas by $\Delta V$ is therefore:
        \begin{equation*}
            \begin{aligned}
                w &= P x \Delta V \text{where $\Delta V$ is}\\
                \Delta V &= A * \Delta h
            \end{aligned}
        \end{equation*}
    \item This work is performed \emph{by} the system is then:
        \begin{equation*}
            \begin{aligned}
                w &= -P x \Delta V
            \end{aligned}
        \end{equation*}
    \item Note! 
        \begin{enumerate}
            \item If a gas \emph{expands} against an external pressure P, it performs work \emph{on the surroundings} (w < 0)
            \item If a gas \emph{compresses} against a external pressure P, the external pressure P is performing work \emph{on the system} (w > 0)
        \end{enumerate}
\end{enumerate}



\end{document}