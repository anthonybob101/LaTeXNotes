\documentclass{article}  % //TODO find why latex gives errors for report and using textsc!
\usepackage{fancyhdr}
\usepackage{graphicx}
\usepackage{amsmath}
\usepackage{mhchem}
\usepackage{amssymb}
\usepackage[margin=1in]{geometry}

\usepackage{subfiles} % Best loaded last in the preamble

\title{UW CHEM 152 Notes}
\author{Anthony Le}

\newtheorem{exmp}{Example}
\newtheorem{exrc}{Excersize}
\newtheorem{proof}{Statement}
\newtheorem{defn}{Definition}


\begin{document}

\pagestyle{fancy}
\fancyhead{}
\fancyhead[R]{UW CHEM 152}
\fancyhead[L]{Anthony Le}

\subfile{Subfiles/CHEM152_Chapter6}

\newpage

\subfile{Subfiles/CHEM152_Chapter7}

\newpage

\subfile{Subfiles/CHEM152_Chapter8.tex}

\newpage

\subfile{Subfiles/CHEM152_Chapter9.tex}

\newpage 

\section*{Chapter 10 - Isothermal ???}
\subsection*{Reminder: Isothermal Processes}
\begin{enumerate}
    \item "Isothermal" means "constant temperature". In an isothermal process, temperature does not change.
        \begin{equation*}
            \begin{aligned}
                \Delta T = 0
            \end{aligned}
        \end{equation*}
    \item Changes in energy are related to changes in temperature.
        \begin{equation*}
            \begin{aligned}
                \Delta E = C_v \Delta T
            \end{aligned}
        \end{equation*}
    In an isothermal process, $\Delta T$ is 0, and therefore $\Delta E$ is also 0 (unless heat capacity is infinite).
    \item The First Law of Thermodynamics is:
        \begin{equation*}
            \begin{aligned}
                ?? %//TODO - Insert equations here!
            \end{aligned}
        \end{equation*}
\end{enumerate}

\subsection*{Reversible and Irreversible Processes}
\begin{enumerate}
    \item In a thermodynamic process, we usually change one property in a controlled way, keep other properties constant, and observe how yet another property changes.
    \item For example, we observe how the volume V changes as we increase the temperature T of a gas while keeping the pressure P constant. 
    \item So far we classified processes based on which thermodynamic property stays constant: \emph{isothermal} (constant temperature), \emph{isobaric} (constant pressure), or \emph{isochloric} (constant volume).
    \item We also distinguish processes whether they are \emph{reversible} or \emph{irreversible}:
        \begin{enumerate}
            \item In a \emph{reversible} process, the system is in equilibrium (or at least very close to it) the entire time.
                \begin{enumerate}
                    \item This means that we can \emph{reverse the process by changing the property that we }
                    \item In practice, this often means that we perform a process very, \textbf{very} slowly (theoretically, infinitely slowly, adding an infinitely small amount of heat).
                \end{enumerate}
            \item In an \emph{irreversible} process, the system is not always in equilibrium. 
                \begin{enumerate}
                    \item This usually happens when there are sudden changes in a thermodynamic property.
                \end{enumerate}
        \end{enumerate} 
    \item Process functions like heat (q) and work (w) depend on the way a process is performed; therefore they depend on whether a process is reversible or irreversible. 
    \item Differences in state functions ($\Delta E$, $\Delta T$,...) do not depend on the path of a process, they will be the same fr reversible and irreversible processes (in an infinite time frame)
\end{enumerate}

\subsection*{Isothermal process example}
Let's consider the following process (im too lazy to copy paste from slides)
\begin{enumerate}
    \item Constant T, Changing Volume and Pressure 
    \item Work Performed by Mass $M_1$, ... %//TODO - Complete this example!
\end{enumerate}

\subsection*{Zero-step expansion}
\begin{enumerate}
    \item If we simply remove the weight, the external pressure drops to 0.
    \item However, the internal pressure of the gas isn't zero - the system's no longer at equilibrium!
    \item The gas will expand isothermally and irrersibly to fill the entire volume. 
    \item The gas doesn't have to expand against an external pressure. It does here due to the sudden transition from having an external pressure to having none. 
    \item Since the gas expands into a vaccum - no work is being done: \emph{w =  0}
    \item This process is called free expansion
    \item Since q = -w, no heat flows between the system and it's surroundings.
\end{enumerate}

\subsection*{One-step expansion}
Compared to taking away the entire mass, let's take away $M_1/4$.
\begin{enumerate}
    \item Taking away $M_1/4$ drops the external pressure to:
        \begin{equation*}
            \begin{aligned}
                P_{ext} = (M_1/4)g/A = P_1/4
            \end{aligned}
        \end{equation*}
    \item At that moment, the pressure of the gas is greater than the external pressure - causing the gas to expand.
    \item This leads to the mass being lifted until the internal pressure is equal to the external pressure. In this case:
        \begin{equation*}
            \begin{aligned}
                V_{final} = 4V_1
            \end{aligned}
        \end{equation*}
    \item Work:
        \begin{equation*}
            \begin{aligned}
                w &= -P_{ext} \Delta V \\
                    &= -P_1/4(4V_1-V_1) \\
                    &= -3/4P_1 V-1
            \end{aligned}
        \end{equation*}
    \item Heat: 
        \begin{equation*}
            \begin{aligned}
                q = -w >0
            \end{aligned}
        \end{equation*}
        So heat flows from the reservoir into the system.
\end{enumerate}

\subsection*{Two-Step Expansion} %//TODO - Write out two step expansion problem!
In this expansion, we'll do two steps
\begin{enumerate}
    \item Changing mass from $M_1$ to $M_1/2$
        \begin{enumerate}
            \item $P_{ext} = P_1/2, V_{final}=2V_1$
            \item  
        \end{enumerate}
\end{enumerate}

\subsection*{(No slide title name) Work between 1, 2, and many step processes}
For a 1 step process:
\begin{equation*}
    \begin{aligned}
        w &= -P^{ex}\Delta V \\
        &= -\frac{P}{4} 3V_1 \\
        q = -w &= \frac{3}{4}P_1 V_1
    \end{aligned}
\end{equation*}
Process is irreversible.
For a two step process:
\begin{equation*}
    \begin{aligned}
        w &= -\frac{P_1}{2}V_1 - \frac{P_1}{4}2 V_1 \\
        &= -P_1V_1 \quad \text{More work!} \\
        q = -w &= P_1V_1
    \end{aligned}
\end{equation*}
Process is irreversible.
For a many step process: (skipping a couple steps):
\begin{equation*}
    \begin{aligned}
        w &= -1.21 P_1V-1 \quad \text{Enen more work!} \\
        q = -w &= 1.21 P_1V_1
    \end{aligned}
\end{equation*}
What if we continued this process of adding more steps to infinity?

\subsection*{Infinite-Step Expansion}
\begin{enumerate}
    \item Imagine we perform this process by repeatedly reducing the weight by an infinitely small amount.
    \item We have to add up the pressure volume-work performed these infinitely many, infinitely small steps. (Think integrals - in this class we won't have to do them) 
    \item Mathematically, this is expressed through an integral:
        \begin{equation*}
            \begin{aligned}
                w &= -\int ^{v_2}_{v_1} P dV \\
                &= -\int ^{v_2}_{v_1} \frac{nRT}{V} dV \\
                &= -nRT \int ^{v_2}_{v_1} \frac{1}{V} dV \\
                &= -nRT [\ln V]^{v_2}_{v_1} \\
                &= -nRT(\ln V_2 - ln V_1) \\
                w &= -nRT \ln \frac{V_2}{V_1}
            \end{aligned}
        \end{equation*}
    \item Applying this formula to the previous example ($V_2 = 4V_1$):
        \begin{equation*}
            \begin{aligned}
                w = -P_1 V_1 \ln(4) \\
                q = -w &= 1.4P_1 V_1 
            \end{aligned}
        \end{equation*}
\end{enumerate}
\textbf{Note for step 3:}
\begin{enumerate}
    \item We assumed the external pressure is always the same as the pressure of the gas, which we calculate from the ideal gas law. 
    \item This is only true if the system is at equilibrium the \emph{entire time}.
    \item This process is "reversible" (or "quasi-static").
    \item In practice, this means the process happens very slowly. 
\end{enumerate}

\subsection*{Isothermal expansion summary}
\begin{enumerate}
    \item Expand gas from $V_1$ to $V_2=4V_1$ by reducing the weight $M_1$ (and thereby reducing the external pressure).
    \item $\Delta T = 0$ , therefore $\Delta E = 0$ and $q = -w$.
    \item Amount of work performed by the system depends on the path (ex 1 step, 2 steps, etc)
    \item The most efficient process/path is by taking infinitely many steps AKA a "reversible" process described in the equation below:
        \begin{equation*}
            \begin{aligned}
                w = -\int ^{v_2}_{v_1} P dV \\
            \end{aligned}
        \end{equation*}
    \item Throughout the reversible process, the pressure of the gas is always the same as the external pressure: the gas is always at equilibrium.
\end{enumerate}

\subsection*{(No slide title) Expansion and Compression Cycles}
\begin{enumerate}
    \item Consider a cycle - one step expansion, followed by one-step compression:
        \begin{equation*}
            \begin{aligned}
                w_{net} &= -0.75P_1V_1 + 3 P_1V_1 = 2.25 P_1V_1 \\
                q_{net} &= 0.75P_1V_1 - 3 P_1 V_1 = -2.25 P_1V-1 
            \end{aligned}
        \end{equation*}
    \item The system is back in its inital state, but the surrounding has performed work on the system, and recieved heat in return.
    \item This is true for all \emph{irreversible} cycles!
    \item Only if both expansion and compression are done \emph{reversibly}  we get $w_{net}=q_{net}=0$.
\end{enumerate}

\footnote{In this next section, we'll start to transition into entropy}
\subsection*{The Concept of Entropy (S)}
\begin{enumerate}
    \item Entropy is a \emph{state function} that measures the disorder of a system.
    \begin{enumerate}
        \item \textbf{Systems work to increase their entropy, and become more disordered.}
        \item Restating this - systems that have more order/less disorder (have lower entropy) compared to systems that have less order/more disorder (have higher entropy)
        \item Processes will happen spontaneously if the system's entropy will increase.
    \end{enumerate}
    \item Macrostate: A particular combination of \emph{macroscopic} variables (T, P, n, ...) - AKA a certain combination of variables that describe the system as a whole.
    \item Microstate: A particular combination of \emph{microscopic} variables (atom positions, velocities) - AKA a certain combination of variables describing components of individiual molecules/particles.
    \item For every macrostate, there are MANY possible microstates. Let's call the number of microstates $\Omega$.
    \item The number of microstates corresponds to how ordered your system is!
        \begin{enumerate}
            \item For example, if $\Omega$ is large, there are many microstates, corresponding to high "disorder" - high entropy
            \item In contrast, if $\Omega$ is low, this would correspond to few microstates and thus a low entropy.
        \end{enumerate}
    \item In 1877, Ludwig Boltzmann proposed the following equation for entropy:
        \begin{equation*}
            \begin{aligned}
                S = k_B \ln \Omega \quad \textbf{Statistical Definition of Entropy}
            \end{aligned}
        \end{equation*}
        Where:
        \begin{equation*}
            \begin{aligned}
                k_B = R/N_A = 1.38X10^-23 J/K \quad \text{Where $k_B$ is called Boltzmann's Constant.}
            \end{aligned}
        \end{equation*}
\end{enumerate}
\subsection*{Entropy Changes}
Sometimes, we're not interested in the value of entropy itself ($S$), but we want to find what the change in entropy is instead ($\Delta S$).
\begin{enumerate}
    \item Imagine a process from state 1 (with $\Omega_1$ microstates) to state 2 (with $\Omega_2$ microstates):
        \begin{enumerate}
            \item Inital Entropy: $S_1 = k_B \ln \Omega_1$
            \item Final Entropy: $S_2 = k_B \ln \Omega_2$
            \item Change in Entropy: 
                \begin{equation*}
                    \begin{aligned}
                        \Delta S = S_2 - S_1 
                    \end{aligned}
                \end{equation*}
        \end{enumerate}
    \item If $\Omega_2 > \Omega_1$, then $\Delta S > 0$ - Entropy increases if more microstates become available.
    \item If $\Omega_2 < \Omega_1$, then $\Delta S < 0$ - Entropy descreases if ... %//TODO - Add in definitions for entropy lecture on 5-17-2023
\end{enumerate}
\subsection*{Example: Isothermal expansion}
\begin{exmp}
    What is the change in entropy for the isothermal expansion of n moles of a monoatomic ideal gas from an inital volume $V_1$ to a final volume $V_2 = 2*V_1$? 
\end{exmp}
\begin{enumerate}
    \item First, focus on a single gas particle: \\
        How many possible positions are there inside the inital volume? \\
        You could start with splitting your inital volume into sections - but it's hard to tell at first since it depends on how you count states. \\
        \textbf{Regardless of how you count your \emph{microstates}}, there are twice as many positions/microstates in the final box. \\
        \begin{equation*}
            \begin{aligned}
                \frac{\Omega_2}{\Omega_1} = \frac{V_2}{V_1} = 2 \quad \text{for a single gas particle}
            \end{aligned}
        \end{equation*}
    \item But what if we have $N = n * N_A$ gas particles? We have to account for the n moles of gas particles! \\
        Every particle has twice as many possible positions in the box that is twice the size. From this, the total number of possible microstates increases as:
        \begin{equation*}
            \begin{aligned}
                \frac{\Omega_2}{\Omega_1} = \left(\frac{V_2}{V_1}\right)_1 + \left(\frac{V_2}{V_1}\right)_2 + ... \left(\frac{V_2}{V_1}\right)_N  = \left(\frac{V_2}{V_1}\right)^N \quad \text{for N gas particles}
            \end{aligned}
        \end{equation*}
    \item The change in entropy for this isothermal expansion is:
        \begin{equation*}
            \begin{aligned}
                \Delta S &= k_B \ln \frac{\Omega_2}{\Omega_1} \\
                    &= k_B \ln \left(\frac{V_2}{V_1}\right)^N \\
                    &= k_B N \ln \frac{V_2}{V_1} \\
                    %//TODO - complete this equation for entropy!
                 \Delta S &= nR\ln\frac{V_2}{V_1}   
            \end{aligned}
        \end{equation*}
\end{enumerate}

\subsection*{Entropy and Heat}
We just calculated that for an isothermal expansion of an ideal gas, the change in entropy is:
\begin{equation*}
    \begin{aligned}
        \Delta S = nR\ln \frac{V_2}{V_1}   
    \end{aligned}
\end{equation*}
Previously, we had shown that the amount of heat involved in an \emph{reversible}\footnote{Read - not true when the process is irreversible} isothermal expansion is:
\begin{equation*}
    \begin{aligned}
        q_{rev} = nRT \ln\frac{V_2}{V_1}
    \end{aligned}
\end{equation*}
Therefore, we can conclude that for a \emph{reversible} isothermal process:
\begin{equation*}
    \begin{aligned}
        \Delta S = \frac{q_{rev}}{T} \quad \textbf{Thermodynamic Definition of entropy\footnote{technically entropy \emph{changes}}}
    \end{aligned}
\end{equation*}
From this, we can relate entropy to the quanities we used in chapter 9! 
If a process is \emph{irreversible} (but still isothermal), then $\Delta S$ is still the same (since entropy is a state function), but q is less (as we have seen in the isothermal expansion example):
\begin{equation*}
    \begin{aligned}
        \Delta S > \frac{q}{T} \quad \text{for \emph{irreversible, isothermal processes}}
    \end{aligned}
\end{equation*}
%//TODO - Take notes on friday's lecture!

\subsection*{Building an absolute entropy scale}
\begin{enumerate}
    \item The thermodynamic definitioon of entropy,
        \begin{equation*}
            \begin{aligned}
                \Delta S = \int \frac{dq}{t}
            \end{aligned}
        \end{equation*}
        allows us to calculate \emph{differences} in entropy between two states.
    \item ...
\end{enumerate}
\subsection*{Standard Molar Entropy}
\begin{enumerate}
    \item The standard molar entropy $S^0$ of a substance is the entropy change per mole of that substance when heating it from  K to 298 K at a constant pressure of 1 atm:
        \begin{equation*}
            \begin{aligned}
                S^0 = \int_0^{298K} \frac{c_p}{T}dT
            \end{aligned}
        \end{equation*}
    \item It has units of J/(mol K).
    \item Remember that we could write the reaction enthalpy in terms of the formation enthalpies of products and reactants,
        \begin{equation*}
            \begin{aligned}
                \Delta H_{rxn}^0 = \sum_{products} n*\Delta H_f^0 - \sum_{reactants} m* \Delta H_f^0
            \end{aligned}
        \end{equation*}
        and the reference state (pure elements at standard state) had zero formation enthalpy.
    \item Similarly, we can calculate the reaction entropy in terms of molar entropies of products and reactants:
        \begin{equation*}
            \begin{aligned}
                \Delta S_{rxn}^0 = \sum_{products} n*\Delta S^0 - \sum_{reactants} m* \Delta S^0
            \end{aligned}
        \end{equation*}
        Here, the reference state is the perfect crystal, which has zero entropy.
\end{enumerate}

\subsection*{Standard Molar Entropy}
\begin{exmp}
    Consider the reaction:
    \begin{equation*}
        \begin{aligned}
            \ce{CO_(g) + 2H2_(g) -> CH3OH_(l)}
        \end{aligned}
    \end{equation*}
\end{exmp}
\begin{enumerate}
    \item What do you think the sign of $\Delta S$ will be? \\
    \\
    The sign of $\Delta S$ will potentially be negative as looking at the states of the system, we are going from gases to liquid - more disordered to ordered - being negative entropy.
    \item What is the change in entropy for this reaction? 
        \begin{equation*}
            \begin{aligned}
                \Delta S &= (127 - (198+2*131)) J/(mol K)\\
                &= -333 J/(mol K)
            \end{aligned}
        \end{equation*}
\end{enumerate}

\subsection*{Predicting Signs of Entropy Changes}
\begin{exmp}
    Consider the exothermic reaction:
    \begin{equation*}
        \begin{aligned}
            \ce{14KMnO4_(s) + 4C3H5(OH)3_(l) -> 7 K2CO3_(s) + 7 Mn2O3_(s) + 5 CO2_(g) + 16 H2O_(g)}
        \end{aligned}
    \end{equation*}
\end{exmp}
Givens:
\begin{enumerate}
    \item $\Delta H$ is negative - so the entropy change of the surroundings is negative (heat is being pumped into the system)
    \item A reaction is spontaneous \emph{if it increases the entropy of the universe}
    \item The 
\end{enumerate}
\begin{enumerate}
    \item Do you think it will be spontaneous? \\
        \begin{enumerate}
            \item Reactants are solid and liquid - Products include gases... the entropy of the system likely increases
            \item Reaction is exothermic, so heat flows into the surroundings. Therefore, the entropy of the surroundings increases.
            \item This means that the entropy of the universe increases, and the reaction is spontaneous for all temperatures
        \end{enumerate}
\end{enumerate}

\subsection*{Spontaneity and Temperature}
\begin{exmp}
    Determine the spontaneity of the reaction below at 298K and 3000K.
    \begin{equation*}
        \begin{aligned}
            \ce{Sn_(s) + O2_(g) -> SnO2_(s)} \quad \Delta H = -578 kJ/mol \quad \Delta S = -207 J/(mol*K)
        \end{aligned}
    \end{equation*}
\end{exmp}
\begin{enumerate}
    \item At 298 K:
        \begin{equation*}
            \begin{aligned}
                \Delta S _{surr} = \frac{-\Delta H}{T} = \frac{-(-578*10^3 J/mol)}{298K} = 1.94*10^3 J/(mol*K) \\
                \Delta S_{univ} = (-207 + 1.94*10^3) J/(mol*K) = 1.73*10^3 J/(mol*K) \quad \text{Spontaneous at 298 K}
            \end{aligned}
        \end{equation*}
    \item At 3000 K: 
        \begin{equation*}
            \begin{aligned}
                \Delta S _{surr} = \frac{-\Delta H}{T} = \frac{-(-578*10^3 J/mol)}{3000K} = 192 J/(mol*K) \\
                \Delta S_{univ} = (-207 + 192) J/(mol*K) = -14.3 J/(mol*K) \quad \text{Not spontaneous at 3000 K}
            \end{aligned}
        \end{equation*}
\end{enumerate}
This indicates that there is a breakpoint where the change in entropy is 0. 

\subsection*{Spontaneity and Temperature}
\begin{exmp}
    Determine the temperature at which the reaction below changes from spontaneous to nonspontaneous.
    \begin{equation*}
        \begin{aligned}
            \ce{Sn_(s) + O2_(g) -> SnO2_(s)} \quad \Delta H = -578 kJ/mol \quad \Delta S = -207 J/(mol*K)
        \end{aligned}
    \end{equation*}
\end{exmp}
We want to know the temperature at which $\Delta S_{univ} = 0$.
\begin{equation}
    \begin{aligned}
        \Delta S_{univ} &= \Delta S_{sys} + \Delta S_{surr} = 0  \\
        \Delta S_{surr} &= \frac{q_{surr}}{T} =\frac{- \Delta H_{sys}}{T} \\
        T &= \frac{\Delta H_{sys}}{\Delta S_{sys}} \quad \text{solving for T, see next equation}\\
        T &= \frac{-578*10^3 J/mol}{-207 mol/K} \\
        T &= 2792 K
    \end{aligned}
\end{equation}
From previous equationss
\begin{equation}
    \begin{aligned}
        \Delta S_{univ} = \Delta S_{surr} + \Delta S_{sys} &= 0\\
        -\frac{\Delta H_{sys}}{T} + \Delta S_{sys} &= 0 \\
        \Delta S_{sys} &= \frac{\Delta H_{sys}}{T} \\
        \text{Solving for T,} \\
        T &= \frac{\Delta H_{sys}}{\Delta S_{sys}}
    \end{aligned}
\end{equation}


\end{document}