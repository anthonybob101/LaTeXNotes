\documentclass{article}  % //TODO find why latex gives errors for report and using textsc!
\usepackage{fancyhdr}
\usepackage{graphicx}
\usepackage{amsmath}
\usepackage{mhchem}
\usepackage{amssymb}
\usepackage[margin=1in]{geometry}

\usepackage{subfiles} % Best loaded last in the preamble

\title{UW CHEM 152 Notes}
\author{Anthony Le}

\newtheorem{exmp}{Example}
\newtheorem{exrc}{Excersize}
\newtheorem{proof}{Statement}
\newtheorem{defn}{Definition}


\begin{document}

\pagestyle{fancy}
\fancyhead{}
\fancyhead[R]{UW CHEM 152}
\fancyhead[L]{Anthony Le}

\section*{Chapter 6 - Chemical Equilibrium: Law of Mass Action}

\subsection*{Chemical Reactions}
Consider a chemical reaction of the form \ce{A -> B}.
\begin{enumerate}
    \item The rate of a reaction tells you how many times a reaction happens in a certain amount of time.
    \item if the reaction \ce{A -> B} is first order, the the rate is proportational to the concentration of A: 
    \begin{equation*}
        \begin{aligned}
            \text{Rate} = k * [A]    
        \end{aligned}
    \end{equation*}
    \begin{enumerate}
        \item From the differential rate law, we can find how the concentration of A and B as a function of time. 
        \item Since the rate is dependent of the concentration of A, once all of A is consumed, then we can say the reaction is complete (sometimes)
    \end{enumerate} 
\end{enumerate}

However, this isn't the only type of reaction. There are also:

\subsection*{Reversible Reactions}
\begin{enumerate}
    \item If the reaction can happen both ways: \ce{A -> B} and \ce{B -> A}. 
    \item We write this as \ce{A <=> B}. 
    \item As you can expect, you'll have 2 different rates corresponding to their reactions - so you can find how the concentrations of A and B change with respect to time
\end{enumerate}
These reactions will continue to happen until a time where the forward and reverse rates become equal to each other, and thus the concentrations no longer change AKA chemical equilibrium.

\begin{defn}
    \textbf{Chemical Equilibrium} \\
    Where forward and backward reaction rates become equal to each other and concentrations no longer change.
\end{defn}

\textbf{The goal of this class is to predict the concentrations of all species in equilibrium.}

\subsection*{General Chemical Reactions}
Consider a general \textbf{elementary} (reaction order is equal to the stoichiometric coefficient) chemical reaction: \\
Forward Reaction:
\begin{equation*}
    \begin{aligned}
        \ce{aA + bB -> cC + dD}
    \end{aligned}
\end{equation*}

Backward Reaction:
\begin{equation*}
    \begin{aligned}
        \ce{aA + bB <- cC + dD}
    \end{aligned}
\end{equation*}

Forward Rate:
\begin{equation*}
    \begin{aligned}
        \text{Forward Rate} = k_f[A]^a[B]^b
    \end{aligned}
\end{equation*}

Backward Rate:
\begin{equation*}
    \begin{aligned}
        \text{Backward Rate} = k_b[C]^c[D]^d
    \end{aligned}
\end{equation*}

Chemical Equilibrium (In Terms of Rates):
\begin{equation*}
    \begin{aligned}
        \text{Forward Rate} = \text{Backward Rate} \\
        k_f[A]^a[B]^b = k_b[C]^c[D]^d
    \end{aligned}
\end{equation*}

This essentially describes that rates at where chemical equilibrium happens.
Solving for the stoichiometric constants $k_f$ and $k_b$
\begin{equation*}
    \begin{aligned}
        \frac{[C]^c[D]^d}{[A]^a[B]^b} = \frac{k_f}{k_b}     
    \end{aligned}
\end{equation*}
This gives us the stoichiometric coeffiencents for where equilibrium occurs. This gives us the constant that is dependent on temperature, and not concentration.


\begin{exmp}
    A scientist performs experiments at 500 C to study equilibrium concentrations of the ammonia synthesis reaction. 
    \begin{equation*}
        \begin{aligned}
            \ce{N_2 (g) + 3H_2(g) <=> 2NH_3 (g)}        
        \end{aligned}
    \end{equation*}
\end{exmp}
By finding the equilibrium coefficients for all 3 reactions from the previous equation: 
\begin{equation*}
    \begin{aligned}
        \frac{[C]^c}{[A]^a[B]^b} = \frac{k_f}{k_b} \\
        \frac{[NH_3]^2}{[N_2][H_2]^3} = \frac{k_f}{k_b}
    \end{aligned}
\end{equation*}
From this example (see class slides), we get the Law of Mass Action:

\begin{defn}
    \textbf{Law of Mass Action} \\
    For a reversible gas- or solution-phase chemical reaction 
    \begin{equation*}
        \begin{aligned}
            \ce{aA + bB <=> cC + dD}
        \end{aligned}
    \end{equation*}
    The ratio of concentrations K at equilibrium is constant (regardless of inital concentrations) at a given temperature, where K is defined as \\
    \begin{equation*}
        \begin{aligned}
            K = \frac{[C]^c[D]^d}{[A]^a[B]^b} 
        \end{aligned}
    \end{equation*}
    For gas phase reactions, we also use partial pressures instead of concentrations to describe the chemical equilibrium. \\
    The pressure-based equilibrium constant is:
    \begin{equation*}
        \begin{aligned}
            K_P = \frac{\left(\frac{P_C}{P_0}\right)^c \left(\frac{P_D}{P_0}\right) ^d}{\left(\frac{P_A}{P_0}\right) ^a \left(\frac{P_B}{P_0}\right) ^b}
        \end{aligned}
    \end{equation*}
    Where $P_0 = 1$ atm as the reference temperature. 
\end{defn}

\begin{proof}
    The equilibrium concentrations themselves \textbf{DO} depend on the inital concentrations are. \\
    However the ratio of equilibrium concentrations \textbf{DO NOT} depend on the inital conditions.
\end{proof}

However, using concentrations doesn't exactly make the units work out perfectly - so we use the "activities" of the molecular species instead of their concentrations. For ideal gases and molecules in solution, the activity is equal to the concentration divided by a reference concentration $c_0$, which by common convention 1 molar.
\begin{equation*}
    \begin{aligned}
        a_A  = \frac{[A]}{c_0} \\
        a_B  =\frac{[B]}{c_0} \\
        a_C  = \frac{[C]}{c_0} \\
        a_D  =\frac{[D]}{c_0} \\ 
    \end{aligned}
\end{equation*}
One thing to note is that even though we continue to write the equilibrium constant in form:
\begin{equation*}
    \begin{aligned}
        K = \frac{[C]^c[D]^d}{[A]^a[B]^b} 
    \end{aligned}
\end{equation*}
We actually mean to write the equilibrium constant in form:
\begin{equation*}
    \begin{aligned}
        K = \frac{[a_C]^c[a_D]^d}{[a_A]^a[a_B]^b} 
    \end{aligned}
\end{equation*}

\begin{exmp}
    The reaction 
    \begin{equation*}
        \begin{aligned}
            \ce{H_2O (g) + CO (g) <=> H_2 (g) + CO_2 (g)}
        \end{aligned}
    \end{equation*}
    Has the equilibrium constant K = 4 at 800K. What will the equilibrium concentrations be if you put 1 mole of \ce{H_2O} and 1 mole of \ce{CO} into a container of volume 1L at that temperature?
\end{exmp}

Due to stoichiometry, \ce{[H_2O] = [CO]}, \ce{[H_2] = [CO_2]}, \ce{[H_2O] + [CO] + [H_2] + [CO_2] = 2M}
\begin{enumerate}
    \item Due to how these these 2 chemicals have the same stoichiometric coefficients (they both have 1), they will always have the same amount at all times. 
    \item The total number of molecules you should have should always be the same.
    \begin{enumerate}
        \item Since in both forward and backward reactions, you have 2 molecules being transformed into 2 other molecules.
    \end{enumerate}

\end{enumerate}
However, you need the law of mass action in order to find the equilibrium concentrations to act as the fourth equation which you can use to solve for them!

\begin{enumerate}
    \item Since we're given K = 4 and we're trying to find the equilibrium concentrations:
    \begin{equation*}
        \begin{aligned}
            K = \frac{[H_2][CO_2]}{[H_2O][CO]} \\
            4 = \frac{[H_2][CO_2]}{[H_2O][CO]} \\
            4 = \frac{[H_2]^2}{[H_2O]^2} \\
            [H_2] = 2[H_2O]
        \end{aligned}
    \end{equation*}
    \item Note that since we're given that \ce{[H_2O] = [CO]} and \ce{[H_2] = [CO_2]}, we can replace \ce{[CO]} and \ce{[CO_2]} so we can simplify the law of mass action.
    \item Given that we know \ce{[H_2O] + [CO] + [H_2] + [CO_2] = 2M}, let replace \ce{[CO]} and \ce{[CO_2]}
    \begin{equation*}
        \begin{aligned}
           \ce{[H_2O] + [CO] + [H_2] + [CO_2] = 2M} \\
           \ce{[H_2O] + [H_2O] + 2[H_2O] + 2[H_2O] = 2M} \\
           \ce{6[H_2O] = 2M} \\
           \ce{[H_2O] = \frac{1}{3}M}
        \end{aligned}
    \end{equation*}
\end{enumerate}

Equilibrium is reached when the forward and reverse reactions occur at the same rate, so there is no net change in the concentrations of reactants and products. However, where this balancing point occurs can be different for each reaction:
\begin{equation*}
    \begin{aligned}
        \ce{2H2 + O2 <=> 2H2O} \quad K = 2.4 * 10^47, T = 500 K \\
        \ce{Cl2 <=> 2Cl} \quad K = 1.8 * 10 ^-9, T = 1000 K \\
        \ce{H2 + I2 <=> 2HI} \quad K = 0.3, T = ??? K
    \end{aligned}
\end{equation*}

For a general chemical reaction in which the number of gas-phase molecules changes by $\Delta n$: 
\begin{equation*}
    \begin{aligned}
        K_C = (\frac{P_0}{c_0RT})^{\Delta N} K_P \quad K_P = (\frac{c_0RT}{P_0})^{\Delta N} K_c
    \end{aligned}
\end{equation*}

\subsection*{Activities and Heterogeneous Equilibrium}
For ideal gases and molecules in solution, the activity is equal to the concentration divided by a reference concentration $c_0 = 1M$
\begin{equation*}
    \begin{aligned}
        a_A &= \frac{[A]}{c_0} \\
            &=\frac{[A]}{1} \\
    \end{aligned}
\end{equation*}
For solids and pure liquids, the activity is simply equal to 1.
\begin{exmp}
    What is K for the thermal decomp. for calcium carbonate to calcium oxide and carbon dioxide?
\end{exmp}
\begin{equation*}
    \begin{aligned}
        &\ce{CaCO3 <=> CaO + CO2} \\
        K &= \frac{a_{CaO}a_{CO_2}}{a_{CaCO_3}} \\
          &= a_{CO_2} = \frac{\ce{CO2}}{1M}
    \end{aligned}
\end{equation*}
Note - this shows us that the equilibrium constant is depentent on the gas molecules and independent on the others.

\subsection*{Manipulating Equilibrium Constants}
\begin{exmp}
    Ammonia Synthesis Reaction 
    \begin{equation*}
        \begin{aligned}
            \ce{N2 + 3 H2 <=> 2 NH3} \quad K = \ce{\frac{[NH3]^2}{[N2][H2]^3}}
        \end{aligned}
    \end{equation*}
\end{exmp}
Swapping reactants and products \\
Note - swapping reactants and products gives the inverse of the equilibrium constant.

\begin{equation*}
    \begin{aligned}
        \ce{2 NH3 <=> N2 + 3 H2} \quad K^f = \ce{\frac{[N2][H2]^3}{[NH3]^2}}
    \end{aligned}
\end{equation*}

\begin{exmp}
    Multi-step formulation of nitrogen dioxide: 
    \begin{equation*}
        \begin{aligned}
            &\ce{N2 + O2 <=> 2 NO} \quad K_1 = \ce{\frac{[NO]^2}{[N2][O2]}} \\
            &\ce{2 NO + O2 <=> 2 NO2} \quad K_2 = \ce{\frac{[NO2]^2}{[NO]^2[O2]}} \\
            &\ce{N2 + 2 O2 <=> 2 NO2} \quad K = \ce{\frac{[NO2]^2}{[N2][O2]^2}} = K_1K_2
        \end{aligned}
    \end{equation*}
\end{exmp}
Note - When adding reactions, multiply equilibrium constants.

\begin{exmp}
    Multiplying first step reaction by 3: 
    \begin{equation*}
        \begin{aligned}
            &\ce{3N2 + 3O2 <=> 6 NO} \quad K = \ce{\frac{[NO]^6}{[N2]^3[O2]^3}} = K_1^3 \\
        \end{aligned}
    \end{equation*}
\end{exmp}
Note - when multiplying a reaction by a number n, raise the equilibrium constant to the n-th power.

\subsection*{The Reaction Quotient Q}
We've seen from the law of mass action: 
\begin{equation*}
    \begin{aligned}
        \ce{aA + bB <=> cC + dD}
    \end{aligned}
\end{equation*}
The concentrations in \textbf{in chemical equilibrium} satisify:
\begin{equation*}
    \begin{aligned}
        K = \frac{[C]^c[D]^d}{[A]^a[B]^b} 
    \end{aligned}
\end{equation*}
However, what if you're not in chemical equilibrium? We can calculate the reaction quotient Q:
\begin{equation*}
    \begin{aligned}
        Q = \frac{[C]^c_0[D]^d_0}{[A]^a_0[B]^b_0} 
    \end{aligned}
\end{equation*}
This represents what has to happen to reach equilibrium.

\begin{defn}
    \textbf{Reaction Quotient (Q)} 
    \begin{equation*}
        \begin{aligned}
            Q = \frac{[C]^c_0[D]^d_0}{[A]^a_0[B]^b_0} 
        \end{aligned}
    \end{equation*}
    \begin{enumerate}
        \item If Q $<$ K, there is not enough product or too much reactant compared to equilibrium. The system will shift "toward the right" AKA the system will be biased to have more forward reactions happening relative to backward reactions.
        \item If Q $>$ K, there is too much product/too little reactant compared to equilibrium. The system will shift "toward the left" ALA the system will be biased to have more backward reactions happening relative to forward reactions.
        \item If Q = K, the system is in equilibrium.
    \end{enumerate}
\end{defn}

for ICE Table - as many columns as there are species in chemical reaction:
I - Inital: represents inital concentrations in M
C - Change: represents stoichiometric coefficients multiplied by x moles of molecules
E - Equilibrium: Represents inital value added with change value 

\subsubsection*{Quizzes}
\begin{enumerate}
    \item First quiz opens 4/5/2023 and is open between 3PM and Midnight and can be found on Canvas.
    \item Only online resources allowed is Canvas quiz page itself and the online textbook.
    \item Will need to have something to write on and scientific calculator.
    \item Also \textbf{will not cover the lecture on the day the quiz is on}. But does cover previous lectures.
\end{enumerate}

\begin{exmp}
    Ammonia synthesis reaction: \\
    \begin{equation*}
        \begin{aligned}
            \ce{N2 + 3 H2 <=> 2 NH3} \quad K = 0.6 at 500C \\
            \ce{N2 = 0.8M \quad H2 = 2.4M \quad NH3 = 0.6M} \\        
        \end{aligned}
    \end{equation*}
\end{exmp}

Finding Q from inital concentrations:
\begin{equation*}
    \begin{aligned}
        Q = \frac{[C]^c_0[D]^d_0}{[A]^a_0[B]^b_0} \\
        Q = \ce{\frac{[NH3]^2_0}{[N2]_0[H2]^3_0}} \\
        Q = \ce{\frac{[0.6]^2_0}{[0.8]_0[2.4]^3_0}} \\
    \end{aligned}
\end{equation*}

\begin{exmp}
    Hydrogen Flouride Reaction and ICE Table 
    \begin{equation*}
        \begin{aligned}
            \ce{H2 + F2 <=> 2HF}
        \end{aligned}
    \end{equation*}
    3 moles of each species is added to a 1.5 L flask. K = 115. Find equilibrium concentrations.
\end{exmp}
\begin{enumerate}
    \item Finding Inital Concentrations:
    \begin{equation*}
        \begin{aligned}
            \ce{[H2] = \frac{3M}{1.5L} =  2M \quad [F2] = \frac{3M}{1.5L} = 2M \quad [HF] = \frac{3M}{1.5L} = 2M}
        \end{aligned}
    \end{equation*}
    \item Create ICE Table \\
    \begin{tabular}{c|c@{}c@{}c@{}c@{}c}
        \hline
        X   &   $[H_2]$ & ${}+{}$ & $[F_2]$ & ${}\leftrightharpoons{}$ & $[2HF]$ \\
        \hline
        I   &       1       &&   2                            &&  0       \\
        C   &       -x      &&   -x                           &&  2x      \\
        E   &       1 - x     &&   2 - x                        &&  2x      \\
        \hline
      \end{tabular}
    \item Write Equilibrium Expression:
    \begin{equation*}
        \begin{aligned}
            K   &= \ce{\frac{[HF]^2}{[H2][F2]}} \\
                &= \ce{\frac{[2x]^2}{[1-x][2-x]}}
        \end{aligned}
    \end{equation*}
    \item Solve for x:
    \begin{equation*}
        \begin{aligned}
            &K = \ce{\frac{[2x]^2}{[1-x][2-x]}} \\
            & = K(1-x)(2-x) = (2x)^2 \\
            & = 2K - Kx - 2Kx + Kx^2 =  \\
            & = (K-4)x^2 - 3Kx + 2K = 0, \quad a = (K-4) \quad b = -3K \quad c = 2K \\
        \end{aligned}
    \end{equation*}
    \item Plugging in K for a, b, and c and finding X via quadratic equation:
    \begin{equation*}
        \begin{aligned}
            x = \frac{-b\pm\sqrt{b^2-4ac}}{2a} \\
            x = \frac{345\pm\sqrt{345^2-4(111*230)}}{2(111)} \\
            x_1 = 2.14 \quad x_2 = 0.968
        \end{aligned}
    \end{equation*}
    \item However, we are x = 2.14 cannot be solution as the equilibrium concentration for \ce{H2} is $1-x$! Thus the solution is x = 0.968.
    \item Calculating Equilibrium Concentrations:
    \begin{equation*}
        \begin{aligned}
            \ce{[H2] = (1-0.968)M = 0.032M} \\
            \ce{[F2] = (2-0.968)M = 1.032M} \\
            \ce{[HF] = (2*0.968)M = 1.936M} \\
        \end{aligned}
    \end{equation*}
    \item Checking Results (Optional)
    \begin{equation*}
        \begin{aligned}
            K   &= \ce{\frac{[HF]^2}{[H2][F2]}} \\
            &= \ce{\frac{1.936^2}{0.032*1.032}} \\
            &= 113 \approx K
        \end{aligned}
    \end{equation*}
\end{enumerate}

\begin{exmp}
    Decomposition of Nitrosyl Chloride \\
    \begin{equation*}
        \begin{aligned}
            \ce{2 NOCl <=> 2 NO + Cl2} \quad K = 1.6 * 10^-5 @ 35C \\
        \end{aligned}
    \end{equation*}
    Walter White adds 1 mol \ce{NOCl} into a 2L flask, what are the equilibrium concentrations?
\end{exmp}

\begin{enumerate}
    \item Finding Inital Concentrations:
    \begin{equation*}
        \begin{aligned}
            \ce{[NOCl] = \frac{3M}{1.5L} =  2M \quad [F2] = \frac{3M}{1.5L} = 2M \quad [HF] = \frac{3M}{1.5L} = 2M}
        \end{aligned}
    \end{equation*}
    \item Create ICE Table \\
    \begin{tabular}{c|c@{}c@{}c@{}c@{}c}
        \hline
        X   &   $[2 NOCl]$  & ${}\leftrightharpoons{}$ & $[2NO]$ & ${}+{}$ & $[Cl_2]$ \\ %//TODO - correct format of ICE table to match example slides
        \hline
        I   &   0.5         &&   0                            &&  0     \\
        C   &   -2x         &&   2x                           &&  x      \\
        E   &   0.5 - 2x    &&   2x                           &&  x      \\
        \hline
      \end{tabular}
    \item Write Equilibrium Expression:
    \begin{equation*}
        \begin{aligned}
            K   &= \ce{\frac{[NO]^2[CL2]}{[NOCl]^2}} \\
                &= \ce{\frac{(2x)^2x}{(0.5-2x)^2}} \\
                &= \ce{\frac{4x^3}{(0.5-2x)^2}} \\
        \end{aligned}
    \end{equation*}
    \item Solve for x:
    \begin{equation*}
        \begin{aligned}
            K &= \ce{\frac{4x^3}{(0.5-2x)^2}} \\
              &= 4x^3 - 4Kx^2 + 2Kx - 0.25K = 0%//TODO - Add in cubic formula from example here!
        \end{aligned}
    \end{equation*}
    \item However, you get a cubic function here which makes things much harder to solve for X! We've seen that the equilibrium constant is very small, so we should expect to the equilibrium position to be far to the left, and x to be small. \\    
    Lets simplify $0.5-2x \approx 0.5$ (Small X Approximation) and try again.
    \item Write Equilibrium Expression:
    \begin{equation*}
        \begin{aligned}
            K   &= \ce{\frac{[NO]^2[CL2]}{[NOCl]^2}} \\
                &= \ce{\frac{(2x)^2x}{(0.5)^2}} \\
                &= \ce{\frac{4x^3}{0.25}} \\
            K    &= 16x^3
        \end{aligned}
    \end{equation*}
    \item Solving for X:
    \begin{equation*}
        \begin{aligned}
            &K = 16x^3 \\
            &\rightarrow x = \left(\frac{K}{16}\right)^{1/3} \\
            &\rightarrow x = 0.01
        \end{aligned}
    \end{equation*}
    \item Calculating Equilibrium Concentrations:
    \begin{equation*}
        \begin{aligned}
            \ce{[H2] = (2-0.01)M = 0.48M} \\
            \ce{[F2] = (2*0.01)M = 1.032M} \\
            \ce{[HF] = (2*0.968)M = 1.936M} \\
        \end{aligned}
    \end{equation*}
    \item Checking Results (Optional)
    \begin{equation*}
        \begin{aligned}
            K   &= \ce{\frac{[HF]^2}{[H2][F2]}} \\
            &= \ce{\frac{1.936^2}{0.032*1.032}} \\
            &= 113 \approx K
        \end{aligned}
    \end{equation*}
\end{enumerate}

%//TODO - take notes on friday's lecture!

\subsection*{The Effect of Changing Concentrations}
\begin{exmp}
    Formation of hydrogen cyanide from methane and ammonia 
    \begin{equation*}
        \begin{aligned}
            \ce{CH4 + NH3 <=> HCN + 3 H2}
        \end{aligned}
    \end{equation*}
    Let's assume the system is at equilibrium. Therefore, the concentration of each species is equal to the equilibrium concentrations.
    \begin{equation*}
        \begin{aligned}
            Q = \ce{\frac{[HCN][H2]^3}{[CH4][NH3]}} = K
        \end{aligned}
    \end{equation*}
\end{exmp}
what happens if we add some more ammonia to the system?
\begin{enumerate}
    \item The current \ce{NH3} concentration increases, and the reaction quotient Q decreases.
    \item Now we have Q < K - the system is no longer in equilibrium!
    \item System evolves towards the new equilibrium by creating more product (it shifts "toward the right", \emph{away from the added component}).
\end{enumerate}
What happens if instead we now add some more hydrogen to the system?
\begin{enumerate}
    \item The current \ce{H2} concentration increases, and the reaction quotient Q increases.
    \item Now we have Q > K - the system is no longer in equilibrium!
    \item System evolves towards the new equilibrium by creating more reactant (it shifts "towards the left", \emph{away from the added component}).
\end{enumerate}

\subsection*{The Effect of Changing Pressure}
There are many ways to change the pressure of the gas-phase chemical system:
\begin{enumerate}
    \item \textbf{Add or remove a gas that is part of the reaction} - see the previous section.
    \item \textbf{Add an inert gas that doesn't participate in the reaction.} - This changes the total pressure, but not the partial pressures of reactants, products, nor their concentrations, so it does not change the reaction quotient.
    \item \textbf{Change the volume of the system.} 
        \begin{exmp}
            Ammonia Synthesis Reaction:
                \begin{equation*}
                    \begin{aligned}
                        \ce{N2 + 3H2 <=> 2 NH3}
                    \end{aligned}
                \end{equation*}
            Let's assume the system is at equilibrium: 
                \begin{equation*}
                    \begin{aligned}
                        Q = \ce{\frac{[NH3]^2}{[N2][H2]^3}} = K
                    \end{aligned}
                \end{equation*}
        \end{exmp}
\end{enumerate}
\textbf{What happens if we now decreases the volume by a factor of 2?}
\begin{enumerate}
    \item Each concentration increases by a factor of 2. 
    \item The reaction quotient Q changes from K to (4/16)K.
    \item Now we have Q < K - the system is no longer in equilibrium!
    \item System evolves towards the new equilibrium by creating more product (it shifts "towards the right", \emph{towards the side of fewer gas molecules})
\end{enumerate}

\subsection*{The Effect of Changing Temperature}
Many chemical reactions release of absorb energy in the form of heat:
\begin{enumerate}
    \item If energy is released, we call the reaction \emph{exothermic}. 
        \begin{exmp}
            Combustion of methane: 
            \begin{equation*}
                \begin{aligned}
                    \ce{CH4 + O2 <=> 2 H2O + CO2 + energy} 
                \end{aligned}
            \end{equation*}
        \end{exmp}
    \item If energy is absorbed, we call the reaction \emph{endothemic}.
        \begin{exmp}
            Decomposition of dinitrogen tetraoxide
            \begin{equation*}
                \begin{aligned}
                    \ce{N2O4 + energy <=> 2 NO2}
                \end{aligned}
            \end{equation*}
        \end{exmp}
\end{enumerate}
For now, we can think of heat energy just like a reactant (for endothermic reactions) or a product (for exothermic reactions). 
\newline
Increasing the temperature shifts the chemical system... 
\begin{enumerate}
    \item Towards the left for exothermic reactions (energy is a product)
    \item Towards the right for endothemic reactions (energy is a reactant)
\end{enumerate}

\section*{Chapter 7 - Acids and Bases: Acid Strength and the pH scale}

\subsection*{Strong and Weak Acids}
Acid dissociation reaction:
\begin{equation*}
    \begin{aligned}
        \ce{HA_(aq) + H2O_(l) <=> H3O+_(aq) + A-_(aq)} \text{Or,}
        \ce{HA_(aq) + H2O_(l) <=> H+_(aq) + A-_(aq)}\text{\footnote{remember we use \ce{H3O} and \ce{H+} interchangeably}}
    \end{aligned}
\end{equation*}

The equilibrium constant is also called the "acid dissociation constant" and is written as $K_a$:
\begin{equation*}
    \begin{aligned}
        K_a = \ce{\frac{[H3O] [A-]}{[HA]}}
    \end{aligned}
\end{equation*}
For strong acids, they dissociate completely, and thus when we compute $K_a$ gives us $\infty$; indicating equilibrium is "to the right".
\begin{equation*}
    \begin{aligned}
        \ce{HA + H2O -> H3O+ + A-} \\
        K_a >> 1
    \end{aligned}
\end{equation*}
For weak acids, they do not completely dissociate, and we will still have some $\ce{HA}$ left over. Computing $K_a$ will give us a number that is less than 1 indicating equilibrium is "to the left"
\begin{equation*}
    \begin{aligned}
        \ce{HA + H2O -> H3O+ + A- + HA} \\
        K_a < 1
    \end{aligned}
\end{equation*}

\begin{exmp}
    What is the hydronium concentration of a 0.2M solution of hydrochloric acid (HCl) in equilibrium.
\end{exmp}
\begin{enumerate}
    \item Since HCl is a strong acid and completely dissociates, we will have a reaction:
    \begin{equation*}
        \begin{aligned}
            \ce{HA + H2O -> H3O+ + A-} \\
            \ce{[H3O+] = [HCl]_0 = 0.2M}
        \end{aligned}
    \end{equation*}
\end{enumerate}

\begin{exmp}
    What is the hydronium concentration of a 0.2M solution of nitrious acid ($\ce{HNO2}$) in equilibrium with $K_a = 4*10^-4$
\end{exmp}
\begin{enumerate}
    \item You must make a ICE table to find it! \\
    \begin{tabular}{c|c@{}c@{}c@{}c@{}c@{}c@{}c}
        \hline
        X   &   $[HNO_2]$ & ${}+{}$ & $[H_2O]$ & ${}\leftrightharpoons{}$ & $[H_3O^+]$ & ${}+{}$ & $[NO_2^-]$\\
        \hline
        I   & 0.2 && - &&  0 && 0       \\
        C   & -x && - &&  x && x      \\
        E   & 0.2 - x && - && x && x      \\
        \hline
      \end{tabular}
    \item Creating $K_a$ 
    \begin{equation*}
        \begin{aligned}
            K_a &= \ce{\frac{[H3O] [A-]}{[HA]}} \\
            K_a &= \frac{x^2}{0.2-x} \text{Then applying small x approximation} \\
                &= \frac{x^2}{0.2} \\
            x   &= \sqrt{0.2K_a} = \sqrt{0.2*4*10^-4} \\
            x   &= 0.009 \\
            \ce{[H3O+]} &= 0.009
        \end{aligned}
    \end{equation*}
    \item Checking approximation 
    \begin{equation*}
        \begin{aligned}
            \text{Error} &= \frac{x}{0.2} \\
            &= \frac{0.009}{0.2} \\
            \text{Error} &= 4.5\% \quad \text{Error within 5\%, acceptiable to use small x approximation!}
        \end{aligned}
    \end{equation*}
\end{enumerate}

\subsection*{Water Autoionizaztion}
Water undergoes acid dissociation with itself. This is called autoionization:
\begin{equation*}
    \begin{aligned}
        \ce{H2O_(l) + H2O_(l) -> H3O+_(aq) + OH-_(aq)}
    \end{aligned}
\end{equation*}
The equilibrium constant in this case is also called the "water autodissociation constant" and is written as $K_w$. This is a measure for how acidic water is.
\newline
For pure water at room temperature (25C):\footnote{There is no such thing as perfectly "pure" water - it constantly fluxuates between slightly basic and slightly acidic}
\begin{equation*}
    \begin{aligned}
        \ce{[H3O+]} = 10^-7 M \\
        \ce{[OH-]} = 10^-7 M \\
        K_w = 10^-14
    \end{aligned}
\end{equation*}


%An aqueous solution is called:
%\begin{enumerate}
%    \item Neutral if $\ce{[H3O+] = [OH-]}$
%    \item Acidic if $\ce{[H3O+] > [OH-]}$
%    \item Basic if $\ce{[H3O+] < [OH-]}$
%\end{enumerate}

\subsection*{Multiple Simultaneous Equilibria}
Water autoionization:
\begin{equation*}
    \begin{aligned}
        \ce{H2O_(l) + H2O_(l) -> H3O+_(aq) + OH-_(aq)} \quad K_w = [H3O+][OH-]
    \end{aligned}
\end{equation*}

Acid dissociation: 
\begin{equation*}
    \begin{aligned}
        \ce{HA + H2O -> H3O+ + A-} \quad K_a = \ce{\frac{[H3O] [A-]}{[HA]}}
    \end{aligned}
\end{equation*}

In order to find the equilibrium concentrations, we have to solve for ALL things like $\ce{[H3O+]}$ and $\ce{[OH-]}$ using both equilibrium constants at the same time. 
\newline
This can be done, but is complicated to do so - instead, we'll do a simple shortcut:

\begin{exmp}
    What is the hydronium concentration of a 0.2M solution of nitrious acid ($\ce{HNO2}$) in equilibrium with $K_a = 4*10^-4$
\end{exmp}
We found that \ce{[H3O+]} = 0.009M from the dissociation of the nitrious acid. Water autoionization also adds additional hydronium ions, but only $10^-7$M, a miniscule amount relative to the acid dissociation.
\newline
\textbf{For these cases, we can igniore the contribution from the water autoionization.}

\subsection*{The power of Hydrogen}
The concentration of protons (or hydronium ions) is an important property of aqueous solutions. It's often quite small, and is often convenient to express it in a different from that yields more "convenient" numbers: (AKA pH scale)
\begin{equation*}
    \begin{aligned}
        pH = -log_{10}[H^+] \quad [H^+] = 10^{-pH}
    \end{aligned}
\end{equation*}
Where:
\begin{enumerate}
    \item p stands for $-log_10$
    \item $[H^+]$ represents the concentration of $H^+$ in Molar
\end{enumerate}
Note that pH represents the "inverse order of magnitude" of the hydronium concentration. Which means:
\begin{enumerate}
    \item The smaller the pH, the more "acidic" a solution is. (Such as HCl and lemon juice)
    \item The greater the pH, the more "basic" a solution is. (Such as baking soda or household ammonia)
\end{enumerate}

\begin{exmp}
    What is the pH of a 0.5M solution of acetic acid \ce{CH3COOH}, a weak acid with $K_a = 1.8*10^-5$?
\end{exmp}

\begin{equation*}
    \begin{aligned}
        \ce{CH3OOH_(aq) + H2O_(l) <=> CH3OO-_(aq) + H3O+_(aq)}
    \end{aligned}
\end{equation*}

\subsection*{Strong and Weak Bases}
Base hydrolysis reaction:
\begin{equation*}
    \begin{aligned}
        \ce{B_(aq) + H2O_(l) <=> BH+_(aq) + OH-_(aq)} \quad K_b = \frac{[BH+][OH-]}{[[B]]}
    \end{aligned}
\end{equation*}
In this situation, the equilibrium constant is also called the "base ionization constant" and is written as $K_b$.
\newline
Adding a base to water increases the concentration of hydroxide ions.
\newline
For strong bases, they are good at grabbing protons, and thus when we compute $K_b$ gives us $\infty$; indicating equilibrium is "to the right".
\begin{equation*}
    \begin{aligned}
        \ce{B_(aq) + H2O_(l) <=> BH+_(aq) + OH-_(aq)} \\
        K_a >> 1
    \end{aligned}
\end{equation*}
For weak bases, they don't act as well at grabbing protons, and we will still have some $\ce{B}$ left over. Computing $K_b$ will give us a number that is less than 1 indicating equilibrium is "to the left"
\begin{equation*}
    \begin{aligned}
        \ce{B_(aq) + H2O_(l) <=> BH+_(aq) + OH-_(aq) + B_(aq)} \\
        K_a < 1
    \end{aligned}
\end{equation*}

%//TODO - add in section where we can ignore water autoionization!

\begin{exmp}
    What is the hydroxidde concentration of a 0.1M solution of NaOH (sodium hydroxide), a strong base?
    \begin{equation*}
        \begin{aligned}
            \ce{NaOH_(aq) <=> Na+_(aq) + OH-_(aq)}
        \end{aligned}
    \end{equation*}    
\end{exmp}
However, this doesn't even look like a base hydrolysis reaction! However, when we break it down further, we can show that this is still a hydrolysis reaction:
\begin{equation*}
    \begin{aligned}
        \ce{Na+_(aq) + OH-_(aq) + H2O_(l) <=> Na+_(aq) + OH-_(aq) + H2O_(l)}
    \end{aligned}
\end{equation*}
In this case, Na+ is a spectator ion.

\begin{exmp}
    What is the hydroxide concentration of a 0.1M solution of ammonia (\ce{NH3}) witha  weak base where $K_b = 1.8*10^-5$?
\end{exmp}
\begin{enumerate}
    \item Writing reaction:
    \item Creating ICE Table:
    \begin{tabular}{c|c@{}c@{}c@{}c@{}c@{}c@{}c}
        \hline
        X   & $[NH_3]$ & ${}+{}$ & $[H_2O]$ & ${}\leftrightharpoons{}$ & $[NH_3^+]$ & ${}+{}$ & $[OH^-]$ \\
        \hline
        I   &  0.1   &&     &&  0   && 0  \\
        C   &  -x    &&     &&  x   && x  \\
        E   &  0.1-x &&     &&  x   && x  \\      
    \end{tabular}
    \item Writing $K_b$
    \begin{equation*}
        \begin{aligned}
            K_b = \ce{\frac{[NH4+][]}{}} %//TODO - Complete this example from 4/12 lecture!
        \end{aligned}
    \end{equation*}
    \item Solving for x
    \item Is our small-x approximation appropiate?
    \begin{equation*}
        \begin{aligned}
            \frac{x}{0.1} = 1.3\% < 5\% \quad \text{Is acceptiable}
        \end{aligned}
    \end{equation*}
    \item Can we ignore autoionization of water?
    \begin{equation*}
        \begin{aligned}
            0.0013 >> 10^-7 \quad \text{Can ignore autoionization}
        \end{aligned}
    \end{equation*}
\end{enumerate}

\subsection*{Relationship between pH and pOH}
The water autoionization reaction:
\begin{equation*}
    \begin{aligned}
        \ce{H2O + H2O <=> H3O+_(aq) + OH-_(aq)}
    \end{aligned}
\end{equation*}
Guarantees that whenever the system is in equilibrium,
\begin{equation*}
    \begin{aligned}
        K_w = \ce{[H3O+][OH-]}
    \end{aligned}
\end{equation*}
Let's take $-log_10$ of both sides:
\begin{equation*}
    \begin{aligned}
        log_10(K_w) &= log_10(\ce{[H3O+][OH-]}) \\
                    &= -log_10(\ce{[H3O+]}) - log_10(\ce{[OH-]}) \\
                    &= pH + pOH
    \end{aligned}
\end{equation*}
For water at 25C, $K_w = 10^-14$ and therefore
\begin{equation*}
    \begin{aligned}
        pH + pOH = 14 \quad \text{(at room temperature)}
    \end{aligned}
\end{equation*}

\subsection*{Conjugate Acids and Bases}
Acid Dissociation Reaction:
\begin{equation*}
    \begin{aligned}
        \ce{HA_(aq) + H2O_(l) <=> H3O+_(aq) + A-_(aq)} \quad K_a = \ce{\frac{[H3O] [A-]}{[HA]}}
    \end{aligned}
\end{equation*}
\ce{A-} can accept a proton and turn into HA, so it is a base - the conjugate base of HA.
\begin{equation*}
    \begin{aligned}
        \ce{A-_(aq) + H2O_(l) <=> HA_(aq) + H2O_(l) } \quad K_b = \ce{\frac{[HA][OH-]}{[A-]}}
    \end{aligned}
\end{equation*}

Now what if we take the product of the acid dissociation constant $K_a$ and the base ionization $K_b$ for a conjugate acid/base pair?
\begin{equation*}
    \begin{aligned}
        K_a K_b = \ce{\frac{[H3O] [A-]}{[HA]}} \ce{\frac{[HA][OH-]}{[A-]}} = K_w
    \end{aligned}
\end{equation*}

Important parts about conjugate acids and bases!!!
\begin{enumerate}
    \item Acids that are stronger than \ce{H3O} have conjugate bases that are weaker than \ce{H2O}. This makes sense as that we have to have Ka and Kb ALWAYS equal to 14 (or a constant!), thus the higher Ka, the lower Kb must be in order to compensate.
    \item Acids that are weaker than water have conjugate bases that are stronger than \ce{OH-}
    \item Acids that are weaker than \ce{H3O+} and stronger than \ce{OH-} have conjugate bases that have strengths somewhere in between the two cases above.
\end{enumerate}

\begin{exmp}
    At 25C, the accid dissociation constant for acetic acid \ce{CH3COOH} is $K_a = 1.8*10^-5$. What is the pH of a 0.35M solution of acetate \ce{CH3COO-}?
\end{exmp}
\begin{enumerate}
    \item Writing out reaction:
    \item Creating ICE Table:
    \newline
    \begin{tabular}{c|c@{}c@{}c@{}c@{}c@{}c@{}c}
        \hline
        X   & $[CH_3COO^-]$ & ${}+{}$ & $[H_2O]$ & ${}\leftrightharpoons{}$ & $[CH_3COOH]$ & ${}+{}$ & $[OH^-]$ \\
        \hline
        I   &  0.35   &&     &&  0   && 0  \\
        C   &  -x    &&     &&  x   && x  \\
        E   &  0.35-x &&     &&  x   && x  \\      
    \end{tabular}
    \item Writing $K_b$
    \begin{equation*}
        \begin{aligned}
            K_b = \ce{\frac{[CH3COOH][OH-]}{CH3COO-}} %//TODO - Complete this example from 4/12 lecture!
        \end{aligned}
    \end{equation*}
    \item Solving for x
    \item Is our small-x approximation appropiate?\footnote{Double check small x approximation error by dividing the x concentration by inital concentration - this will give you the error}
    \begin{equation*}
        \begin{aligned}
            \frac{x}{0.35} = 0.004\% < 5\% \quad \text{Is acceptiable}
        \end{aligned}
    \end{equation*}
    \item Can we ignore autoionization of water?
    \begin{equation*}
        \begin{aligned}
            1.4*10^-5 >> 10^-7 \quad \text{Can ignore autoionization}
        \end{aligned}
    \end{equation*}
    \item Finding pH and pOH %//TODO - Complete example from 4/12 lecture!
\end{enumerate}

General form of finding concentration for ANY weak acid with an inital concentration of "c" M and given Ka
\begin{enumerate}
    \item Writing out reaction:
    \item Creating ICE Table: 
    \newline
    \begin{tabular}{c|c@{}c@{}c@{}c@{}c@{}c@{}c}
        \hline
        X   & $[HA]$ & ${}+{}$ & $[H_2O]$ & ${}\leftrightharpoons{}$ & $[H^+]$ & ${}+{}$ & $[A^-]$ \\
        \hline
        I   &  c   &&     &&  0   && 0  \\
        C   &  -x    &&     &&  x   && x  \\
        E   &  c-x &&     &&  x   && x  \\      
    \end{tabular}
    \item Writing $K_a$
    \begin{equation*}
        \begin{aligned}
            K_a = \ce{\frac{[H+][A-]}{HA}} \\
            K_a = \frac{x^2}{c-x}
        \end{aligned}
    \end{equation*}
    \item Solving for x (assuming small x approximation)
    \begin{equation*}
        \begin{aligned}
            K_a &= \frac{x^2}{c-x} \\
            K_a &= \frac{x^2}{c} \\
            K_a(c) &= x^2 \\
            x &= \sqrt{K_a*c}
        \end{aligned}
    \end{equation*}
    \item Now that we have x in terms of $K_a$ and c, we can find the minimum concentration to find where small x approximation can apply - Solving for x in our error check:
    \begin{equation*}
        \begin{aligned}
            \frac{x}{c} &= 0.05\% \\
            \frac{\sqrt{K_a*c}}{c} &= 0.05 \\
            \left(\frac{\sqrt{K_a*c}}{c}\right) ^2&= 0.05^2 \\
            \frac{K_a*c}{c^2} &= 0.025 \\
            \frac{K_a}{c} &= 0.025 \\
            c &= \frac{K_a}{0.025}
        \end{aligned}
    \end{equation*}
\end{enumerate}

\begin{enumerate}
    \item As acid concentration increases \ce{H3O+} increases, pH decreases, and percent dissociated decreases
    We quantify this through the use of the pKa constant, where the acid is 50\% dissociated when pH = pKa %//TODO - include formula to find pKa!
    \item 
\end{enumerate}

\subsection*{Polyprotic Acids}
Some acids can donate more than 1 proton, for example carbonic acid \ce{H2CO3}.
There are multiple equilibria to consider simultaneously:
\begin{equation*}
    \begin{aligned}
        \ce{H2CO3 + H2O <=> HCO3- + H3O+} \quad K_{a1} = \ce{\frac{[HCO3-][H3O+]}{H2CO3}} = 4.3*10^-7  \\
        \ce{HCO3- + H2O <=> CO32- + H3O+} \quad K_{a2} = \ce{\frac{[CO3^2-][H3O+]}{HCO2-}} = 4.8*10^-11 \\
    \end{aligned}
\end{equation*}
With each step, the equilibrium constant decreases. 
\newline 
Calculating equiulibrium concentrations is possible, but difficult due to multiple equilibria. But we can make qualitative predictions:
\begin{enumerate}
    \item At high pH, the acid will be completely dissociated
    \item At low pH, the acid will not be dissociated at all.
\end{enumerate}
This leads to a graph where we can plot the theoretical concentrations of each compound as a function of the pH

\subsection*{Acid-Base Properties of Salts}
What happens when you add ions to water? 
\newline
\begin{exmp}
    When adding chloride ions to water, the following is in principle possible:
\end{exmp}
\begin{equation*}
    \begin{aligned}
        \ce{Cl- + H2O <=> HCl + OH-}
    \end{aligned}
\end{equation*}
\begin{enumerate}
    \item However, this doesn't happen since HCl is a strong acid, and Cl- is a very weak base - and doesn't have enough strength to pull the hydrogen ions - so the equilibrium lies far to the left. \textbf{This means that adding anions of strong acids have no effect on pH.} 
    \item Similarly, this is true when doing it for strong bases - adding cations for strong bases have no effect on pH
    \item Thus generally, \textbf{Adding the conjugate bases/acids of strong acids/bases have no effect on pH}
    \item \textbf{However}, this isn't true for when we do the opposite! Adding the conjugate bases/acids of \textbf{weak} acids/bases DO have a effect on pH 
\end{enumerate} 

\section*{Chapter 8: Applications of ... Ion Solutions}

\subsection*{Common Ion Solutions}
We know that pure water has a pH of 7 at room temperature. 
So far, we have studied how the pH changes if you add either acid or base to water. For example:
\begin{enumerate}
    \item A 0.01M solution of sodium hydroxide, a strong base, has a pH of 12 through:
    \begin{equation*}
        \begin{aligned}
            pOH &= -log_{10}(0.01) = 2 \\
            pH &= 14-pOH \\
            pH &= 12
        \end{aligned}
    \end{equation*}
    \item A 0.5M solution of acetic acid, a weak acidm, has a pH of 2.52.
    \item a 0.5M solution of sodium acetate, a salt containing the weak base acetate, has a pH of 9.22.
\end{enumerate}
What happens if you add both a weak acid and its conjugate base to water, for example by adding a salt?
\begin{equation*}
    \begin{aligned}
        \ce{HA + H2O <=> A- + H3O+} \quad K_a = \ce{\frac{[H+][A-]}{HA}} \\
        \ce{NaA_(s) -> Na+_(aq) + A-_(aq)} 
    \end{aligned}
\end{equation*}
\begin{enumerate}
    \item Both the acid and the salt generate \ce{A-} - it is a common ion.
    \item If we look at this step according to Le Chatelier's principle, you'll be adding acetate ions, disturbing equilibrium. Since we have additional acetate anions and hydronium cations, you'll shift the reaction to the left in order to return to equilibrium.
\end{enumerate}

\begin{exmp}
    What is the pH of a solution that is 0.5M acetic acid(\ce{CH3COOH}, $Ka = 1.8*10^-5$) and 0.5M sodium acetate (\ce{NaCH3COO})?
\end{exmp}
\begin{enumerate}
    \item Sodium acetate dissolves completely:
    \begin{equation*}
        \begin{aligned}
            \ce{NaCH3COO_(s) -> Na+ CH3COO-_(aq)}
        \end{aligned}
    \end{equation*}
    \item Solve acetic acid dissocation equilibrium:
    \begin{equation*}
        \begin{aligned}
            \ce{CH3COOH_(aq) + H2O_(l) <=> CH3COO-_(aq) + H3O+_(aq)}
        \end{aligned}
    \end{equation*}
    \item Creating ICE table: %//TODO - Complete ICE Table from 4/17/2023 Slides!
    \begin{tabular}{c|c@{}c@{}c@{}c@{}c@{}c@{}c}
        \hline
        X   & $[\ce{NaCH3COO_}]$ & ${}+{}$ & $[H_2O]$ & ${}\leftrightharpoons{}$ & $[\ce{CH3COO-}]$ & ${}+{}$ & $[H_3O]$ \\
        \hline
        I   &  0.5    &&     &&  0.5    && 0  \\
        C   &  -x     &&     &&  x      && x   \\
        E   &  0.5-x  &&     &&  0.5+x  && x  \\      
    \end{tabular}
    \item Solving for hydronium concentration (x)
\end{enumerate}
From this, we have considered solutions containing a weak acid and its conjugate base:
\begin{enumerate}
    \item Ex - A solution that is 0.5M in acetic acid and in sodium acetate has a pH of 4.74 (this is between 2.52 and 9.22)
    \item The present of the (conjugate?) base inhibits the dissociation of the acid
\end{enumerate}

Now, what if we add a strong base (or strong acid) to this solution?
\begin{exmp}
    A solution that is 0.5M acetic acid(\ce{CH3COOH}, $Ka = 1.8*10^-5$) and 0.5M sodium acetate (\ce{NaCH3COO}) has a pH of 4.74. How does the pH change if we add 0.01 mol of sodium hydroxide to 1L of this solution?
\end{exmp}
\begin{enumerate}
    \item Sodium hydroxide dissolves completely:
    \begin{equation*}
        \begin{aligned}
            \ce{NaOh_(s) -> Na+_(aq) + OH-_(aq)}
        \end{aligned}
    \end{equation*}
    
    \item Hydroxide is a strong base that will "steal" protons. These will come from the strongest acid available - acetic acid.
    \begin{equation*}
        \begin{aligned}
            \ce{CH3COOH_(aq) + H2O_(l) <=> CH3COO-_(aq) + H3O+_(aq)}
        \end{aligned}
    \end{equation*}
    \item Since hydroxide is stuch a strong base, the reaction will go to completion - and we can use basic stoichiometry to calculate the resulting concentrations:
    \begin{tabular}{c|c@{}c@{}c@{}c@{}c@{}c@{}c}
        \hline
        X   $[CH_3COOH]$ & ${}+{}$ & $[H_2O]$ & ${}\leftrightharpoons{}$ & $[CH_3COO^-]$ & ${}+{}$ & $[H_3O^+]$ \\
        \hline
        I   &  0.49     &&     &&  0.51   && 0  \\
        C   &  -x       &&     &&  x   && x  \\
        E   &  0.49-x   &&     &&  0.51+x   &&  x \\      
    \end{tabular}
    \item Finding $K_a$
    \begin{equation*}
        \begin{aligned}
            K_a &= \frac{(0.51+x)(x)}{(0.49-x)} \quad \text{small-x approximation} \\
            K_a &\approx \frac{(0.51+x)(x)}{(0.49)} \\
            x &= \frac{0.49}{0.51}K_a = 1.73*10^{-5} \\
            pH &= 4.76
        \end{aligned}
    \end{equation*}
    \item 
\end{enumerate}
From this, it's very strange that adding a strong base to a common ion solution will change the pH by only 0.02! From this we get that:
\begin{enumerate}
    \item It is difficult to change the pH of a solutionm containing signifficant amounts of weak acid and its conjugate base.
    \item Such solution resists changes in pH - this is called a \textbf{buffered solution} or just "buffer"
    \item This is very important for applications requiring a constant pH (examples include blood)
\end{enumerate}

\subsection*{Buffered Solutions}
Recap - How do buffers maintain a nearly constant pH?
\begin{enumerate}
    \item We start with a solution with similar (read: roughly equal) amounts of weak acid and its conjugate base
    \item Their equilibrium is governed by:
    \begin{equation*}
        \begin{aligned}
            \ce{HA + H2O <=> A- + H3O+} \quad K_a = \ce{\frac{[H+][A-]}{HA}} \quad \ce{[H3O+] = Ka\frac{[HA]}{[A-]}}
        \end{aligned}
    \end{equation*}
    \item Adding strong bases to weak acids leads to them being neutralized by the weak acid, and turning into their conjugate base. 
    \begin{enumerate}
        \item If we add \ce{OH-}, then HA is converted to \ce{A-} If the amount of OH- is much less than that of HA and A-, then the ratio [HA]/[A-] changes by very little. AKA the pH changes very little.
        \item If we add \ce{H+}, then \ce{A-} is converted to HA. If the amount of \ce{H+} is much less than that of HA and \ce{A-}, then the ratio of \ce{[HA]/[A-]} changes very little. AKA the pH changes very little.
    \end{enumerate}
\end{enumerate}

\subsection*{Henderson-Hasselbalch Equation}\footnote{Be careful! Tricky as there are situations in which this section doesn't count for}
In the final step, we considered the equilibrium position of the weak acid dissocation:
\begin{equation*}
    \begin{aligned}
        \ce{HA + H2O <=> A- + H3O+} \quad K_a = \ce{\frac{[H+][A-]}{HA}} 
    \end{aligned}
\end{equation*}
Once you know the equilibrium concentrations of HA and \ce{A-}, you can calculate concentration of hydronium and then pH:
\begin{equation*}
    \begin{aligned}
        pH = pKa + log_{10}\ce{\frac{[A-]}{[HA]}}
    \end{aligned}
\end{equation*}
THIS ONLY APPLIES IF A- is the conjugate base of HA!

The equilibrium cocncentrations of [HA] and [A-] are determined by 
\begin{enumerate}
    \item How much weak acid and its conjugate base were used in the creation of the buffer.
    \item How much weak acid and its conjugate base were present after neutralization wiht added OH- and H+
    \item The new equilibrium of the weak acid dissociation reaction.
    \begin{enumerate}
        \item The last step often has a very small effect on the equilibrium concentrations of the weak acid and its conjugate base. The Henderson-Hassellbalch equation is often used with [HA] and [A-] after neutralization (second step) 
    \end{enumerate}
\end{enumerate}
\begin{exmp}
    0.5M Acetic acid ($K_a = 1.8*10^{-5}$), 0.5M sodium acetate
\end{exmp}
\begin{equation*}
    \begin{aligned}
        pH = -log_{10}(1.8*10^{-5}) + log_{10}(0.5/0.5) = 4.74
    \end{aligned}
\end{equation*}
\textbf{NOTE - Using the HH (Henderson-Hasselbalch) Equation in this way is equivalent to doing small-x approximation} like we have done in previous examples.

In addition to note, you should ONLY use this when x is small compared to 0. Thus you can use this instead of doing the ICE table and assuming small-x approximation

\subsection*{Buffer Capacity}
Buffer Capacity is the amount of strong acid/base a buffer can absorb without a significant change in pH. 
\newline
Let's compare two different buffers that have different concentrations of weak acid and conjugate base, but the same ratio:
\begin{center}
    Buffer 1 \quad Buffer 2 \\
    0.05M Acetic acid \quad 0.05M acetic acid \\
    0.5M Acetate \quad 0.05M acetate
\end{center}
What is the pH of the buffered solution?
\begin{equation*}
    \begin{aligned}
        pH  = pKa + log_{10}(0.5/0,5) = 4.74 \quad pH  = pKa + log_{10}(0.05/0.05) = 4.74
    \end{aligned}
\end{equation*}
What if we add 0.03mol of sodium hydroxide (NaOH)?
\begin{equation*}
    \begin{aligned}
        pH  &= pKa + log_{10}(0.53/0.47) \quad pH  &= pKa + log_{10}(0.08/0.02) \\
        &= 4.80 \quad &= 5.35
    \end{aligned}
\end{equation*} %//TODO - fix the alignment of this so pH and calculated pH are in their own columns alligned by the = sign!

\subsection*{Chapter 8 - Applications of Squeous Equilibria (Titrations and pH Curves)}

\subsection*{Acid/Base Titration}
Titration is a method of quantitative chemical analysis that is used to determine the unknown concentration of a species in solution. 
\newline
Acid base titration problems are a mix of stoichiometry and equilibrium:
\begin{enumerate}
    \item stoichiometry - Add a strong base (acid) to an acid/base
    \item Eq - Concentrations determined and equilbalnce expression used to find hydronium ions
\end{enumerate}

\subsection*{Titrating a strong acid with a strong base}
\textbf{Basic Idea:}
\begin{enumerate}
    \item Strong acid and strong base neutralize each other - pH is determined by leftover hydroxide or hydronium ions.
    \item If strong acid and base neutralize each other exactly - pH is determined by water.
    \item Keep in mind the stoichiometry of neutralization is based on the number of moles, whereas pH is based on concentrations!
    \begin{enumerate}
        \item Remember that titrations are based on us adding volume, so we have to keep track of the volume added since we're changing the concentration firstly based off of volume change alone!
    \end{enumerate}
\end{enumerate}

\begin{exmp}
    Consider titration of 50mL of 0.2M of HCl with 0.1M of NaOH.
\end{exmp}
%//TODO - Add in section before "After adding in 10mL of NaOH!
After adding 10mL of NaOH:
\begin{enumerate}
    \item THe strong base will dissocate completely, and the resulting hydroxide ions will react with the hydronium ions to form water:
    \begin{equation*}
        \begin{aligned}
            \ce{OH- + H3O+ <=> 2H2O}
        \end{aligned}
    \end{equation*}
    \item This equilibrium lies very far towards products since K = $\frac{1}{K_W} = 10^{14}$, so we can consider hydronium and hydroxide completely reacting to form water.
    \newline
    \begin{tabular}{c@{}c@{}c@{}}
        moles H3O from acid && 0.05L x 0.2M = 0.01mol \\
        moles OH- from base && 0.01L x 0.1M = 0.001mol \\
    \end{tabular}
    \newline
    Remaining H3O+ after neutralization:
\end{enumerate}
After adding 20mL of NaOH:
\begin{tabular}{c c}
    moles H3O from acid & 0.05L x 0.2M = 0.01mol \\
    moles OH- from base & 0.02L x 0.1M = 0.002mol \\
\end{tabular}
Remaining H3O+ after neutralization: 0.01 mol - 0.002mol = 0.008mol \\
New hydronium concentraiton =  0.008/0.07L = 0.11M \\
New pH = 0.94 \\
\newline
\begin{tabular}{ c c }
    moles H3O from acid & 0.05L x 0.2M = 0.01mol \\
    moles OH- from base & 0.02L x 0.1M = 0.002mol \\
\end{tabular}
\newline

Continuing until we have 100mL of NaOH:
\newline
\begin{tabular}{ c c }
    moles H3O from acid & 0.05L x 0.2M = 0.01mol \\
    moles OH- from base & 0.10L x 0.1M = 0.01mol \\
\end{tabular}
\newline
Remaining H3O+ after neutralization: 0.01 mol - 0.01mol = 0mol \\
\begin{enumerate}
    \item Since we have added the same amount of OH- as we had H3O+ from the acid, we have reached the equilvalence point of the titration.
    \item 
\end{enumerate}

What if we exceed the equilvalence point? 
\newline
Since we have gone past the acid balance point, pH is no longer determined by the hydronium ions, but rather the hydroxide ions instead

\subsection*{Titrating a Weak Acid with a Strong Base}
Basic Idea:
\begin{enumerate}
    \item The hydroxide ions from the added strog base will convert weak acid into its conjugate base.
    \item Based on the amounts of remainig weak acid and the produced conjugate base, use equilibrium concepts to compute pH.
\end{enumerate}

\begin{exmp}
    Consider the titration of 20mL of 0.25M nitrous acid ($\ce{HNO2}$, $K_a = 4.5*10^-4$) with 0.25M sodium hydroxide ($\ce{NaOH}$)
\end{exmp}
Acid only, before we add any $\ce{NaOH}$
\begin{equation*}
    \begin{aligned}
        \ce{HNO2 + H2O <=> NO2- + H3O+}
    \end{aligned}
\end{equation*}
\newline
\begin{tabular}{c|c@{}c@{}c@{}c@{}c@{}c@{}c}
    \hline
    X   & $[HNO_2]$ & ${}+{}$ & $[H_2O]$ & ${}\leftrightharpoons{}$ & $[NO_2^-]$ & ${}+{}$ & $[H_3O^+]$ \\
    \hline
    I   &  0.25   &&     &&  0   && 0  \\
    C   &  -x  &&     &&  x  && x \\
    E   & 0.25-x &&     &&  x  && x \\      
\end{tabular}
\newline
\begin{equation*}
    \begin{aligned}
        K_a &= \ce{\frac{[NO2-][H3O+]}{[HNO2]}} \\
            &\text{small x approx.} \approx \frac{x^2}{0.25} \\
          x &= 0.0106 \\
        pH &= -log_{10}\ce{[H3O+]} \\
        pH &= 1.97
    \end{aligned}
\end{equation*}


After adding 5mL of NaOH:
\newline
The strong base will dissociate completely, and the resulting hydroxide ions will react with the nitrous acid to form water and nitrite ions:
\begin{equation*}
    \begin{aligned}
        \ce{OH- + HNO2 <=> H2O + NO2-}
    \end{aligned}
\end{equation*}
This equilibrium lies very far towards the product side:
\begin{center}
    \begin{tabular}{ c c }
    Moles of Acid HNO2: & 0.02 L * 0.25M = 0.005 mol \\
    Moles OH- from NaOH: & 0.005 L * 0.25M = 0.00125 mol \\
    Conjugate base NO2- produced: & 0.00125 mol \\
    Remaining HNO2: & 0.005 - 0.00125 = 0.00375 mol \\
    New HNO2 Concentration: & 0.00375mol / 0.025L = 0.15 M \\
    New NO2- Concentration & 0.00125mol / 0.025L  = 0.05 M 
    \end{tabular}    
\end{center}
Creating ICE table: \\
$\cdots$ %//TODO - copy down steps to find pH this step for 4/21/2023 Notes!
\begin{equation*}
    \begin{aligned}
        K_a = ?? \\
        pH = -log_{10}[H_3O^+] = ?? = pKa
    \end{aligned}
\end{equation*}
$\cdots$ \\ %//TODO - include writing out HH equation and finding pH and pKa!
After adding 10mL of NaOH:
\begin{center}
    \begin{tabular}{ c c }
    Moles of Acid HNO2: & 0.02 L * 0.25M = 0.005 mol \\
    Moles OH- from NaOH: & 0.01 L * 0.25M = 0.0025 mol \\
    Conjugate base NO2- produced: & 0.0025 mol \\
    Remaining HNO2: & 0.005 - 0.0025 = 0.0025 mol
    \end{tabular}    
\end{center}
After reacting wth the strong base, we have equal amount of weak acid and its conjugate base ... this is called the midpoint of the titration.
\begin{center}
    \begin{tabular}{ c c }
        New HNO2 Concentration: & 0.0025mol / 0.03L = 0.083 M \\
        New NO2- Concentration & 0.0025mol / 0.03L  = 0.083 M 
    \end{tabular}
\end{center}
Creating ICE table: \\
$\cdots$ %//TODO - create ICE Table after adding 10mL of NaOH!
\begin{equation*}
    \begin{aligned}
        K_a = 4.5 *10^-4 \\
        pH = -log_{10}[H_3O^+] = 3.35 = pKa
    \end{aligned}
\end{equation*}
$\cdots$  \\ %//TODO - include writing out HH equation and finding pH and pKa!
Note! The mid point is the only point where the pH is equal to pKa - where half of your weak acid dissociates\footnote{Equivalent point and mid point are two different ideas} \\
After adding 20mL of NaOH:
\begin{center}
    \begin{tabular}{ c c }
    Moles of Acid HNO2: & 0.02 L * 0.25M = 0.005 mol \\
    Moles OH- from NaOH: & 0.02 L * 0.25M = 0.005 mol \\
    Conjugate base NO2- produced: & 0.005 mol \\
    Remaining HNO2: & 0.005 - 0.005 = 0 mol
    \end{tabular}    
\end{center}
We have added the same amount of strong base as we had weak acid - so you have converted ALL of your weak acid. This is the equilvalence point of the titration.
\begin{center}
    \begin{tabular}{ c c }
        New HNO2 Concentration: & 0 mol / 0.04L = 0 M \\
        New NO2- Concentration & 0.005mol / 0.04L  = 0.125 M 
    \end{tabular}
\end{center}
Note! We have a lower concetration of conjugate base compared to our inital weak acid concentration - but this is because of how we doubled our volume! \\
Note, now that all of our weak acid has been converted into conjugate base, the pH is now determined by the base hydrolsis of $\ce{NO2-}$: \\

Creating ICE Table: \\

$\cdots$ %//TODO - Create ICE table!
\begin{equation*}
    \begin{aligned}
        x &= \sqrt{0.125 K_b} = \sqrt{0.125 \frac{K_w}{K_a}} \\
        &= 1.67*10^-4 \\
        pOH &= -log_{10}[OH^-] \\
        &= 5.78 \\
        pH = 14 - pOH \\
        &= 8.22 
    \end{aligned}
\end{equation*}

Things to note about the pH and NaOH added graph:
\begin{enumerate}
    \item The pH at equilvalence isn't 0.
    \item Notice the large variation (steep slope) in pH at the equilvalence point
    \item At the mid-point, we have equal amounts of weak acid and conjugate base, and the pH changes very SLOWLY ... because we essentially have a buffer!
    \item When applying the Henderson-Hasselbalch equation, the equation works best to "the left" of the equilvalence point, and mostly around the mid-point
\end{enumerate}

\subsubsection*{4/25/2023 Quiz Section Notes}

List of strong Acids
\begin{enumerate}
    \item \ce{HCl}
    \item \ce{HNO3}
    \item \ce{HClO4}, this dissociates into a weak acid, \ce{H+ + HClO3}
    \item \ce{H2SO4}
    \item \ce{HBr}
    \item \ce{HI}
    \item \ce{HClO3}
\end{enumerate}
Every other acid that isn't in this list is considered a weak acid
\newline
If you see Ka, think of how strong an acid is - and when you see Kb, think of 
\newline
pH is used only when you are talking about a buffer - 


\newpage

\section*{Chapter 9.1-9.3 - Energy and Enthalpy}

\subsection*{What is Thermodynamics?}
\begin{enumerate}
    \item Imagine you want to describe the air in this room
    \begin{enumerate}
        \item A (classical) physicist might try to tell you the positions and velocities of every single atom (the classical "state" of the system)\footnote{\textbf{TECHNICALLY}, you can do this, but this is in practice impossible}
        \item A (quantum) physicist might try to tell you the waveform of all atoms (the quantum "state" of the system)\footnote{\textbf{TECHNICALLY}, you can do this, but this is in practice EVEN MORE impossible}
    \end{enumerate}
    \begin{enumerate}
        \item A reasonable person will tell you the pressure is 1 atmosphere and the temperature is 25C\footnote{Much easier to do.}
    \end{enumerate}
    \item Points 1a and 1b represent microscopic desciptions of nature - describing the universe on a microscopic level!
    \item Point 1c represents the macroscopic description of nature
    \item Thermodynamics is a macroscopic description of the universe in terms of a small number of readily observable properties - temperature, pressure, volume, etc etc.
    \item It (mostly) ignores the fact that matter is made of atoms and molecules. It doesn't need it! It's built around very few \textbf{laws of thermodynamics}. 
\end{enumerate}
$\left.
    \begin{tabular}{ll}
        (a) Cock \\
        (b) Cum
    \end{tabular}
\right\}$  = text
    
\subsection*{Energy of a Molecular System}
\begin{enumerate}
    \item Kinetic Energy ($E_g$) - Energy associated with \emph{motion} \\
    Different types of motion:
        \begin{enumerate}
            \item Translation - moving from one point to another.
            \item Rotation - motion about the center of mass\footnote{Technically, an atom cannot rotate due to how its symmetric, but a molecule can creating chemical energy!}
            \item Vibration - motion directed through chemical bonds
        \end{enumerate}
    \item Potential Energy ($E_p$) - Energy associated with \emph{interactions} \\
    In molecular systems, the most important interaction is usually the electrostatic interaction. This includes: \footnote{This is important in water due to how it exists BECAUSE of the \textbf{intra}molecular and \textbf{inter}molecular interactions between water molecules!}
        \begin{enumerate}
            \item Interactions within a molecule (\textbf{intra}molecular)
            \item Interactions between molecules (\textbf{inter}molecular)
        \end{enumerate}
    \item The total energy of a system is the sum of ineic and potential energy: $E = E_k + E_p$
    \item The unit of energy is Joules(J) - where $1J = 1kg*m^2/s^2$
    \item Also, calories are also another unit of energy - $1cal = 4.184J$\footnote{These calories that we deal with in food are technically not "calories" per se, its actually kilocalories}
    \item Also, electronvolts are another unit of energy, suited for the small interactions from chem - $1 eV = 1.6*10^{-19}J$
\end{enumerate}

\subsection*{Conservation of Energy}
\begin{enumerate}
    \item Different forms of energy can be converted into each oher (for example, kinetic and potential energy). But energy cannot be created or destroyed, thus
    \begin{proof}
        The total amount of energy in the universe is constant.
    \end{proof}
    \item Thinking about the entire universe is hard, Let's partition the universe into the "system" we're interested in and everything else (the "surroundings"):
        \begin{enumerate}
            \item The energy of the system can change if there is an equal but opposie change in energy of the suroundings.
            \item In other words: energy can flow between the system and its surroundings
        \end{enumerate}
\end{enumerate}

\subsection*{State and State Changes}
\begin{enumerate}
    \item The energy of the system depends on the current state of the system (we say that "energy is a state function"). Look at energy as a property describing the state of the system as a whole
        \begin{enumerate}
            \item Continuing this logic to an example - the state of a gaseous system is determined by pressure, volum , amount of sybstance, and temperature.
        \end{enumerate}
    \item The energy of the system \emph{does not} depend on how the system was brough into this state.
        \begin{enumerate}
            \item Let us consider a process in which the sate of the system changes from an inital state to a final sstate (for example, we change the temperature of a gas while keeping the volume and substance amount constant)
            \item The change in the system's energy $\Delta E$ is the difference between the energies of the final and inital state:
                \begin{equation*}
                    \begin{aligned}
                        \Delta E = E_{final} - E_{inital}
                    \end{aligned}
                \end{equation*}
            \item Note the sign convention here!
                \begin{equation*}
                    \begin{aligned}
                        \Delta E > 0 \quad \text{System Energy Increases} \\
                        \Delta E < 0 \quad \text{System Energy Decreases} \\
                    \end{aligned}
                \end{equation*}
            \item $\Delta E$ depends only on what the inital and final \emph{states} are, but \textbf{not} on what the \textbf{process} was that led to the change in state!
        \end{enumerate}
\end{enumerate}

\subsection*{First Law of Thermodynamics}
\begin{enumerate}
    \item We already know that if the system energy changes by $\Delta E$, there must be an equal but opposite change $-\Delta E$ in the energy of the surroundings.
    \item Energy flows from the system to the surroundings ($\Delta E < 0$) or from the surroundings to the system ($\Delta E > 0$)
    \item As we'll see, the flow of energy between the system and the surroundings can be seperated into two contributions - \emph{heat (q)} and \emph{work (w)}: \\
    \emph{The First Law of Thermodynamics}
        \begin{equation*}
            \begin{aligned}
                \Delta E = q + w 
            \end{aligned}
        \end{equation*}
    \item Our sign convention:\footnote{NOTE - some books \textbf{and Wikipedia} use the opposite sign convention for the work (w)!!!}
        \begin{enumerate}
            \item q > 0: heat flows from surroundings to system (system energy increases) 
            \item q < 0: heat flows from the system to surroundings (system energy decreases)
            \item w > 0: work is performed \emph{on} system (system energy increases)
            \item w < 0: work is performed \emph{by} system
        \end{enumerate}
\end{enumerate}

\subsection*{Exothermic Processes}
In an exothermic process, heat is released \emph{by} the system into the surroundings (q < 0).
\begin{enumerate}
    \item Physical Example - cooling water releases heat to the surroundings. \\
    Here heat loss is primarily due to the decrease in molecular kinetic energy due to the water molecules slowing down.
    \item Chemical Example - oxidation of glycerol $\ce{C3H8O3}$ by $\ce{KMnO4}$ releases heat to the surroundings.
    Here heat loss is primarily due to the decrease in molecular potential energy due to the molecules reacting to each other.
\end{enumerate}

\subsection*{Endothermic Processes}
In an endothermic process, heat is absorbed by the system from the surroundings (q > 0).
\begin{enumerate}
    \item Physical Example - heating water absorbs heat from the surroundings. \\
    Here heat gain is primarily due to the increase in molecular kinetic energy due to the water molecules slowing down.
    \item Chemical Example - mixure of barium hydroxide ($\ce{Ba(OH)2}$) and ammonium chloride ($NH4Cl$) absorbs heat from the surroundings.
    Here heat gain is primarily due to the increase in molecular potential energy due to the molecules reacting to each other.
\end{enumerate}

\subsection*{Pressure - Volume Work}
\begin{enumerate}
    \item How do chemical systems exchange energy as work?
    \item \emph{Motivating Example} - heat is applied to the air in a hot-air balloon, causing the volume of gas to increase
    \item As the gas expands, it pushes back the atmosphere ... \emph{it performs work on the atmosphere}
\end{enumerate}
The expansion of gas against a constant expternal pressue is a common type of work in chemical processes. This is called "pressure-volume" (PV) work. 
\newline
How much work does the system do to push against a constant external pressure? 
\newline
\begin{enumerate}
    \item Knowing work is the product of force and distance:
        \begin{equation*}
            \begin{aligned}
                w = F x \Delta h
            \end{aligned}
        \end{equation*}
    \item The force we are pushing against comes from the external pressure:
        \begin{equation*}
            \begin{aligned}
                F = P x A
            \end{aligned}
        \end{equation*}
    \item The work required to expand the gas by $\Delta V$ is therefore:
        \begin{equation*}
            \begin{aligned}
                w &= P x \Delta V \text{where $\Delta V$ is}\\
                \Delta V &= A * \Delta h
            \end{aligned}
        \end{equation*}
    \item This work is performed \emph{by} the system is then:
        \begin{equation*}
            \begin{aligned}
                w &= -P x \Delta V
            \end{aligned}
        \end{equation*}
    \item Note! 
        \begin{enumerate}
            \item If a gas \emph{expands} against an external pressure P, it performs work \emph{on the surroundings} (w < 0)
            \item If a gas \emph{compresses} against a external pressure P, the external pressure P is performing work \emph{on the system} (w > 0)
        \end{enumerate}
\end{enumerate}



\end{document}