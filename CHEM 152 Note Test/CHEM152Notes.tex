\documentclass{article}  % //TODO find why latex gives errors for report and using textsc!
\usepackage{fancyhdr}
\usepackage{graphicx}
\usepackage{amsmath}
\usepackage{mhchem}
\usepackage{amssymb}
\usepackage[margin=1in]{geometry}

\usepackage{subfiles} % Best loaded last in the preamble

\title{UW CHEM 152 Notes}
\author{Anthony Le}

\newtheorem{exmp}{Example}
\newtheorem{exrc}{Excersize}
\newtheorem{proof}{Statement}
\newtheorem{defn}{Definition}


\begin{document}

\pagestyle{fancy}
\fancyhead{}
\fancyhead[R]{UW CHEM 152}
\fancyhead[L]{Anthony Le}

\subfile{Subfiles/CHEM152_Chapter6}

\newpage

\subfile{Subfiles/CHEM152_Chapter7}

\newpage

\subfile{Subfiles/CHEM152_Chapter8.tex}

\newpage

\subfile{Subfiles/CHEM152_Chapter9.tex}

\newpage 

\section*{Chapter 10 - Isothermal ???}
\subsection*{Reminder: Isothermal Processes}
\begin{enumerate}
    \item "Isothermal" means "constant temperature". In an isothermal process, temperature does not change.
        \begin{equation*}
            \begin{aligned}
                \Delta T = 0
            \end{aligned}
        \end{equation*}
    \item Changes in energy are related to changes in temperature.
        \begin{equation*}
            \begin{aligned}
                \Delta E = C_v \Delta T
            \end{aligned}
        \end{equation*}
    In an isothermal process, $\Delta T$ is 0, and therefore $\Delta E$ is also 0 (unless heat capacity is infinite).
    \item The First Law of Thermodynamics is:
        \begin{equation*}
            \begin{aligned}
                ?? %//TODO - Insert equations here!
            \end{aligned}
        \end{equation*}
\end{enumerate}

\subsection*{Reversible and Irreversible Processes}
\begin{enumerate}
    \item In a thermodynamic process, we usually change one property in a controlled way, keep other properties constant, and observe how yet another property changes.
    \item For example, we observe how the volume V changes as we increase the temperature T of a gas while keeping the pressure P constant. 
    \item So far we classified processes based on which thermodynamic property stays constant: \emph{isothermal} (constant temperature), \emph{isobaric} (constant pressure), or \emph{isochloric} (constant volume).
    \item We also distinguish processes whether they are \emph{reversible} or \emph{irreversible}:
        \begin{enumerate}
            \item In a \emph{reversible} process, the system is in equilibrium (or at least very close to it) the entire time.
                \begin{enumerate}
                    \item This means that we can \emph{reverse the process by changing the property that we }
                    \item In practice, this often means that we perform a process very, \textbf{very} slowly (theoretically, infinitely slowly, adding an infinitely small amount of heat).
                \end{enumerate}
            \item In an \emph{irreversible} process, the system is not always in equilibrium. 
                \begin{enumerate}
                    \item This usually happens when there are sudden changes in a thermodynamic property.
                \end{enumerate}
        \end{enumerate} 
    \item Process functions like heat (q) and work (w) depend on the way a process is performed; therefore they depend on whether a process is reversible or irreversible. 
    \item Differences in state functions ($\Delta E$, $\Delta T$,...) do not depend on the path of a process, they will be the same fr reversible and irreversible processes (in an infinite time frame)
\end{enumerate}

\subsection*{Isothermal process example}
Let's consider the following process (im too lazy to copy paste from slides)
\begin{enumerate}
    \item Constant T, Changing Volume and Pressure 
    \item Work Performed by Mass $M_1$, ... %//TODO - Complete this example!
\end{enumerate}

\subsection*{Zero-step expansion}
\begin{enumerate}
    \item If we simply remove the weight, the external pressure drops to 0.
    \item However, the internal pressure of the gas isn't zero - the system's no longer at equilibrium!
    \item The gas will expand isothermally and irrersibly to fill the entire volume. 
    \item The gas doesn't have to expand against an external pressure. It does here due to the sudden transition from having an external pressure to having none. 
    \item Since the gas expands into a vaccum - no work is being done: \emph{w =  0}
    \item This process is called free expansion
    \item Since q = -w, no heat flows between the system and it's surroundings.
\end{enumerate}

\subsection*{One-step expansion}
Compared to taking away the entire mass, let's take away $M_1/4$.
\begin{enumerate}
    \item Taking away $M_1/4$ drops the external pressure to:
        \begin{equation*}
            \begin{aligned}
                P_{ext} = (M_1/4)g/A = P_1/4
            \end{aligned}
        \end{equation*}
    \item At that moment, the pressure of the gas is greater than the external pressure - causing the gas to expand.
    \item This leads to the mass being lifted until the internal pressure is equal to the external pressure. In this case:
        \begin{equation*}
            \begin{aligned}
                V_{final} = 4V_1
            \end{aligned}
        \end{equation*}
    \item Work:
        \begin{equation*}
            \begin{aligned}
                w &= -P_{ext} \Delta V \\
                    &= -P_1/4(4V_1-V_1) \\
                    &= -3/4P_1 V-1
            \end{aligned}
        \end{equation*}
    \item Heat: 
        \begin{equation*}
            \begin{aligned}
                q = -w >0
            \end{aligned}
        \end{equation*}
        So heat flows from the reservoir into the system.
\end{enumerate}

\subsection*{Two-Step Expansion} %//TODO - Write out two step expansion problem!
In this expansion, we'll do two steps
\begin{enumerate}
    \item Changing mass from $M_1$ to $M_1/2$
        \begin{enumerate}
            \item $P_{ext} = P_1/2, V_{final}=2V_1$
            \item  
        \end{enumerate}
\end{enumerate}

\subsection*{(No slide title name) Work between 1, 2, and many step processes}
For a 1 step process:
\begin{equation*}
    \begin{aligned}
        w &= -P^{ex}\Delta V \\
        &= -\frac{P}{4} 3V_1 \\
        q = -w &= \frac{3}{4}P_1 V_1
    \end{aligned}
\end{equation*}
Process is irreversible.
For a two step process:
\begin{equation*}
    \begin{aligned}
        w &= -\frac{P_1}{2}V_1 - \frac{P_1}{4}2 V_1 \\
        &= -P_1V_1 \quad \text{More work!} \\
        q = -w &= P_1V_1
    \end{aligned}
\end{equation*}
Process is irreversible.
For a many step process: (skipping a couple steps):
\begin{equation*}
    \begin{aligned}
        w &= -1.21 P_1V-1 \quad \text{Enen more work!} \\
        q = -w &= 1.21 P_1V_1
    \end{aligned}
\end{equation*}
What if we continued this process of adding more steps to infinity?

\subsection*{Infinite-Step Expansion}
\begin{enumerate}
    \item Imagine we perform this process by repeatedly reducing the weight by an infinitely small amount.
    \item We have to add up the pressure volume-work performed these infinitely many, infinitely small steps. (Think integrals - in this class we won't have to do them) 
    \item Mathematically, this is expressed through an integral:
        \begin{equation*}
            \begin{aligned}
                w &= -\int ^{v_2}_{v_1} P dV \\
                &= -\int ^{v_2}_{v_1} \frac{nRT}{V} dV \\
                &= -nRT \int ^{v_2}_{v_1} \frac{1}{V} dV \\
                &= -nRT [\ln V]^{v_2}_{v_1} \\
                &= -nRT(\ln V_2 - ln V_1) \\
                w &= -nRT \ln \frac{V_2}{V_1}
            \end{aligned}
        \end{equation*}
    \item Applying this formula to the previous example ($V_2 = 4V_1$):
        \begin{equation*}
            \begin{aligned}
                w = -P_1 V_1 \ln(4) \\
                q = -w &= 1.4P_1 V_1 
            \end{aligned}
        \end{equation*}
\end{enumerate}
\textbf{Note for step 3:}
\begin{enumerate}
    \item We assumed the external pressure is always the same as the pressure of the gas, which we calculate from the ideal gas law. 
    \item This is only true if the system is at equilibrium the \emph{entire time}.
    \item This process is "reversible" (or "quasi-static").
    \item In practice, this means the process happens very slowly. 
\end{enumerate}

\subsection*{Isothermal expansion summary}
\begin{enumerate}
    \item Expand gas from $V_1$ to $V_2=4V_1$ by reducing the weight $M_1$ (and thereby reducing the external pressure).
    \item $\Delta T = 0$ , therefore $\Delta E = 0$ and $q = -w$.
    \item Amount of work performed by the system depends on the path (ex 1 step, 2 steps, etc)
    \item The most efficient process/path is by taking infinitely many steps AKA a "reversible" process described in the equation below:
        \begin{equation*}
            \begin{aligned}
                w = -\int ^{v_2}_{v_1} P dV \\
            \end{aligned}
        \end{equation*}
    \item Throughout the reversible process, the pressure of the gas is always the same as the external pressure: the gas is always at equilibrium.
\end{enumerate}

\subsection*{(No slide title) Expansion and Compression Cycles}
\begin{enumerate}
    \item Consider a cycle - one step expansion, followed by one-step compression:
        \begin{equation*}
            \begin{aligned}
                w_{net} &= -0.75P_1V_1 + 3 P_1V_1 = 2.25 P_1V_1 \\
                q_{net} &= 0.75P_1V_1 - 3 P_1 V_1 = -2.25 P_1V-1 
            \end{aligned}
        \end{equation*}
    \item The system is back in its inital state, but the surrounding has performed work on the system, and recieved heat in return.
    \item This is true for all \emph{irreversible} cycles!
    \item Only if both expansion and compression are done \emph{reversibly}  we get $w_{net}=q_{net}=0$.
\end{enumerate}









\end{document}