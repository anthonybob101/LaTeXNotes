\documentclass{article}  % //TODO find why latex gives errors for report and using textsc!
\usepackage{fancyhdr}
\usepackage{graphicx}
\usepackage{amsmath}
\usepackage{mhchem}
\usepackage{amssymb}
\usepackage[margin=1in]{geometry}

\usepackage{subfiles} % Best loaded last in the preamble

\title{UW CHEM 152 Notes}
\author{Anthony Le}

\newtheorem{exmp}{Example}
\newtheorem{exrc}{Excersize}
\newtheorem{proof}{Statement}
\newtheorem{defn}{Definition}


\begin{document}

\pagestyle{fancy}
\fancyhead{}
\fancyhead[R]{UW CHEM 152}
\fancyhead[L]{Anthony Le}

\subfile{Subfiles/CHEM152_Chapter6}

\newpage

\subfile{Subfiles/CHEM152_Chapter7}

\newpage

\subfile{Subfiles/CHEM152_Chapter8.tex}

\newpage

\section*{Chapter 9.1-9.3 - Energy and Enthalpy}

\subsection*{What is Thermodynamics?}
\begin{enumerate}
    \item Imagine you want to describe the air in this room
    \begin{enumerate}
        \item A (classical) physicist might try to tell you the positions and velocities of every single atom (the classical "state" of the system)\footnote{\textbf{TECHNICALLY}, you can do this, but this is in practice impossible}
        \item A (quantum) physicist might try to tell you the waveform of all atoms (the quantum "state" of the system)\footnote{\textbf{TECHNICALLY}, you can do this, but this is in practice EVEN MORE impossible}
    \end{enumerate}
    \begin{enumerate}
        \item A reasonable person will tell you the pressure is 1 atmosphere and the temperature is 25C\footnote{Much easier to do.}
    \end{enumerate}
    \item Points 1a and 1b represent microscopic desciptions of nature - describing the universe on a microscopic level!
    \item Point 1c represents the macroscopic description of nature
    \item Thermodynamics is a macroscopic description of the universe in terms of a small number of readily observable properties - temperature, pressure, volume, etc etc.
    \item It (mostly) ignores the fact that matter is made of atoms and molecules. It doesn't need it! It's built around very few \textbf{laws of thermodynamics}. 
\end{enumerate}
$\left.
    \begin{tabular}{ll}
        (a) Cock \\
        (b) Cum
    \end{tabular}
\right\}$  = text
    
\subsection*{Energy of a Molecular System}
\begin{enumerate}
    \item Kinetic Energy ($E_g$) - Energy associated with \emph{motion} \\
    Different types of motion:
        \begin{enumerate}
            \item Translation - moving from one point to another.
            \item Rotation - motion about the center of mass\footnote{Technically, an atom cannot rotate due to how its symmetric, but a molecule can creating chemical energy!}
            \item Vibration - motion directed through chemical bonds
        \end{enumerate}
    \item Potential Energy ($E_p$) - Energy associated with \emph{interactions} \\
    In molecular systems, the most important interaction is usually the electrostatic interaction. This includes: \footnote{This is important in water due to how it exists BECAUSE of the \textbf{intra}molecular and \textbf{inter}molecular interactions between water molecules!}
        \begin{enumerate}
            \item Interactions within a molecule (\textbf{intra}molecular)
            \item Interactions between molecules (\textbf{inter}molecular)
        \end{enumerate}
    \item The total energy of a system is the sum of ineic and potential energy: $E = E_k + E_p$
    \item The unit of energy is Joules(J) - where $1J = 1kg*m^2/s^2$
    \item Also, calories are also another unit of energy - $1cal = 4.184J$\footnote{These calories that we deal with in food are technically not "calories" per se, its actually kilocalories}
    \item Also, electronvolts are another unit of energy, suited for the small interactions from chem - $1 eV = 1.6*10^{-19}J$
\end{enumerate}

\subsection*{Conservation of Energy}
\begin{enumerate}
    \item Different forms of energy can be converted into each oher (for example, kinetic and potential energy). But energy cannot be created or destroyed, thus
    \begin{proof}
        The total amount of energy in the universe is constant.
    \end{proof}
    \item Thinking about the entire universe is hard, Let's partition the universe into the "system" we're interested in and everything else (the "surroundings"):
        \begin{enumerate}
            \item The energy of the system can change if there is an equal but opposie change in energy of the suroundings.
            \item In other words: energy can flow between the system and its surroundings
        \end{enumerate}
\end{enumerate}

\subsection*{State and State Changes}
\begin{enumerate}
    \item The energy of the system depends on the current state of the system (we say that "energy is a state function"). Look at energy as a property describing the state of the system as a whole
        \begin{enumerate}
            \item Continuing this logic to an example - the state of a gaseous system is determined by pressure, volum , amount of sybstance, and temperature.
        \end{enumerate}
    \item The energy of the system \emph{does not} depend on how the system was brough into this state.
        \begin{enumerate}
            \item Let us consider a process in which the sate of the system changes from an inital state to a final sstate (for example, we change the temperature of a gas while keeping the volume and substance amount constant)
            \item The change in the system's energy $\Delta E$ is the difference between the energies of the final and inital state:
                \begin{equation*}
                    \begin{aligned}
                        \Delta E = E_{final} - E_{inital}
                    \end{aligned}
                \end{equation*}
            \item Note the sign convention here!
                \begin{equation*}
                    \begin{aligned}
                        \Delta E > 0 \quad \text{System Energy Increases} \\
                        \Delta E < 0 \quad \text{System Energy Decreases} \\
                    \end{aligned}
                \end{equation*}
            \item $\Delta E$ depends only on what the inital and final \emph{states} are, but \textbf{not} on what the \textbf{process} was that led to the change in state!
        \end{enumerate}
\end{enumerate}

\subsection*{First Law of Thermodynamics}
\begin{enumerate}
    \item We already know that if the system energy changes by $\Delta E$, there must be an equal but opposite change $-\Delta E$ in the energy of the surroundings.
    \item Energy flows from the system to the surroundings ($\Delta E < 0$) or from the surroundings to the system ($\Delta E > 0$)
    \item As we'll see, the flow of energy between the system and the surroundings can be seperated into two contributions - \emph{heat (q)} and \emph{work (w)}: \\
    \emph{The First Law of Thermodynamics}
        \begin{equation*}
            \begin{aligned}
                \Delta E = q + w 
            \end{aligned}
        \end{equation*}
    \item Our sign convention:\footnote{NOTE - some books \textbf{and Wikipedia} use the opposite sign convention for the work (w)!!!}
        \begin{enumerate}
            \item q $>$ 0: heat flows from surroundings to system (system energy increases) 
            \item q $<$ 0: heat flows from the system to surroundings (system energy decreases)
            \item w $>$ 0: work is performed \emph{on} system (system energy increases)
            \item w $<$ 0: work is performed \emph{by} system
        \end{enumerate}
\end{enumerate}

\subsection*{Exothermic Processes}
In an exothermic process, heat is released \emph{by} the system into the surroundings (q < 0).
\begin{enumerate}
    \item Physical Example - cooling water releases heat to the surroundings. \\
    Here heat loss is primarily due to the decrease in molecular kinetic energy due to the water molecules slowing down.
    \item Chemical Example - oxidation of glycerol $\ce{C3H8O3}$ by $\ce{KMnO4}$ releases heat to the surroundings.
    Here heat loss is primarily due to the decrease in molecular potential energy due to the molecules reacting to each other.
\end{enumerate}

\subsection*{Endothermic Processes}
In an endothermic process, heat is absorbed by the system from the surroundings (q > 0).
\begin{enumerate}
    \item Physical Example - heating water absorbs heat from the surroundings. \\
    Here heat gain is primarily due to the increase in molecular kinetic energy due to the water molecules slowing down.
    \item Chemical Example - mixure of barium hydroxide ($\ce{Ba(OH)2}$) and ammonium chloride ($NH4Cl$) absorbs heat from the surroundings.
    Here heat gain is primarily due to the increase in molecular potential energy due to the molecules reacting to each other.
\end{enumerate}

\subsection*{Pressure - Volume Work}
\begin{enumerate}
    \item How do chemical systems exchange energy as work?
    \item \emph{Motivating Example} - heat is applied to the air in a hot-air balloon, causing the volume of gas to increase
    \item As the gas expands, it pushes back the atmosphere ... \emph{it performs work on the atmosphere}
\end{enumerate}
The expansion of gas against a constant expternal pressue is a common type of work in chemical processes. This is called "pressure-volume" (PV) work. 
\newline
How much work does the system do to push against a constant external pressure? 
\newline
\begin{enumerate}
    \item Knowing work is the product of force and distance:
        \begin{equation*}
            \begin{aligned}
                w = F * \Delta h
            \end{aligned}
        \end{equation*}
    \item The force we are pushing against comes from the external pressure:
        \begin{equation*}
            \begin{aligned}
                F = P * A
            \end{aligned}
        \end{equation*}
    \item The work required to expand the gas by $\Delta V$ is therefore:
        \begin{equation*}
            \begin{aligned}
                w &= P * \Delta V \text{where $\Delta V$ is}\\
                \Delta V &= A * \Delta h
            \end{aligned}
        \end{equation*}
    \item This work is performed \emph{by} the system is then:
        \begin{equation*}
            \begin{aligned}
                w &= -P * \Delta V
            \end{aligned}
        \end{equation*}
    \item Note! 
        \begin{enumerate}
            \item If a gas \emph{expands} against an external pressure P, it performs work \emph{on the surroundings} (w < 0)
            \item If a gas \emph{compresses} against a external pressure P, the external pressure P is performing work \emph{on the system} (w > 0)
        \end{enumerate}
\end{enumerate}
\begin{exmp}
    $1.3*10^8 J$ of heat is applied to the air in a hot-air balloon, causing this volume of gas to inflate from $4*10^6L$ to $4.5*10^6L$. By how much (in J) does the energy of the balloon change?
\end{exmp}
\begin{enumerate}
    \item In this example, assume external pressure is 1 atm:
        \begin{equation*}
            \begin{aligned}
                w &= -P * \Delta V \\
                  &= -(1atm)*(V_{final}-V_{inital}) \\
                  &= -0.5*10^6L-atm
            \end{aligned}
        \end{equation*}
        Unit Conversion - 1L-atm = 101.3 J
        \begin{equation*}
            \begin{aligned}
                w &= -0.5 * 10^6L-atm * (101.3 J/L-atm)
                w &= -5.1*10^7 J 
            \end{aligned}
        \end{equation*}
    \item w is the work the \emph{surroundings have done on the system!!}, in this case, the system is performing work on the surroundings, taking $5.1*10^7$ J from the surroundings.
    \item Calculating change in energy:
        \begin{equation*}
            \begin{aligned}
                \Delta E &= q + w \\
                    &= 1.3*10^8J + (-5.1*10^7) \\
                    &= 8*10^7J
            \end{aligned}
        \end{equation*}
\end{enumerate}

\subsection*{Energy and Enthalpy}
\begin{enumerate}
    \item When we add or remove heat (\emph{q}) to a system \emph{while keeping the volume constant ($\Delta V = 0$)}, \textbf{no work is performed}, and the change in energy is:
        \begin{equation*}
            \begin{aligned}
                \Delta E = q_V \leftarrow\text{\small{  the subscript v is a reminder that this is true only at a constant volume!}}
            \end{aligned}
        \end{equation*}
        All the added heat goes into increases the system's energy!
    \item When we add or remove heat (\emph{q}) \emph{while keeping the pressure constant}, work is performed:
        \begin{equation*}
            \begin{aligned}
                w = -P\Delta V
            \end{aligned}
        \end{equation*}
        In this case, the first law can be written as:
        \begin{equation*}
            \begin{aligned}
                \Delta E + P\Delta V = q_p \leftarrow\text{\small{  the subscript p is a reminder that this is true only at a constant pressure!}}
            \end{aligned}
        \end{equation*}
    \item This makes accounting of energy a little more complicated. We can make our lives easier by defining a new quantity called the \emph{enthalpy} of a system, \emph{H}:
        \begin{equation*}
            \begin{aligned}
                H = E + PV 
            \end{aligned}
        \end{equation*}
        If in a process the pressure is constant ($\Delta P = 0$), then the change in enthalpy is:
        \begin{equation*}
            \begin{aligned}
                \Delta H = \Delta E + P \Delta V
            \end{aligned}
        \end{equation*}
        Now reformatting this - if we add or remove heat at a constant heat, the first law becomes:
        \begin{equation*}
            \begin{aligned}
                \Delta H = q_p
            \end{aligned}
        \end{equation*}
        All the added heat goes into increasing the system's enthalpy!
\end{enumerate}

\subsection*{Heat Capacity}
\begin{enumerate}
    \item When heat is added to a system ($q > 0$), usually (not always, as we'll see) the temperature increases ($\Delta T > 0$).
    \item Similarly, when heat is removed from a system ($q < 0$), the temperature usually decreases ($\Delta T < 0$).
    \item The change in temperature is proportional to the amount of heat added or removed. Suggesting a constant controls this proportionality.
    \item The (inverse) constant of proportionality is called the \emph{heat capacity C}:
        \begin{equation*}
            \begin{aligned}
                \Delta T = \frac{1}{C}q \quad or \quad C = \frac{q}{\Delta T}
            \end{aligned}
        \end{equation*}
    \item The heat capacity \emph{C} describes by how much the temperature of a system changes when heat is addedd or removed:
        \begin{enumerate}
            \item If C is large, then the temperature changes only a little if heat is added/removed.
            \item If C is small, then the temperatures changes by a lot if heat is added/removed.
            \item The heat capacity measures a system's ability to resist changing its temperature when heat is added or removed.
        \end{enumerate}
    \item How big \emph{C} is depends on the system (ie: based on size, material, temperature, physical state...).
    \item Temperature is a measure of average kinetic energy. As we have seen, by how much the energy changes ($\Delta E$) when we add or remove heat depends on whether we do so at constant volume of constant pressure. Therefore, the heat capacity C also depends whether we add heat at constant pressure or volume.
    \item Following up on this, the heat capacity is the ratio of (heat added/removed) to (change in temperature)
        \begin{equation*}
            \begin{aligned}
                C = \frac{q}{\Delta T}
            \end{aligned}
        \end{equation*}
    \item If the heat is added or removed \emph{while keeping the volume constant}, we know that:
        \begin{equation*}
            \begin{aligned}
                q_v = \Delta E \leftarrow\text{Only true at constant volume}
            \end{aligned}
        \end{equation*}
        and therefore, the constant-volume heat capacity $C_v$ is:
        \begin{equation*}
            \begin{aligned}
                C_v = \frac{\Delta E}{\Delta T} \quad or \quad \Delta E = C_v \Delta T \leftarrow\text{Always true!}
            \end{aligned}
        \end{equation*}
    \item If the heat is added or removed \emph{while keeping the pressure constant}, we know that:
    \begin{equation*}
        \begin{aligned}
            q_p = \Delta H \leftarrow\text{Only true at constant pressure}
        \end{aligned}
    \end{equation*}
    and therefore, the constant-volume heat capacity $C_v$ is:
    \begin{equation*}
        \begin{aligned}
            C_v = \frac{\Delta H}{\Delta T} \quad or \quad \Delta E = C_p \Delta T \leftarrow\text{Always true!}
        \end{aligned}
    \end{equation*}
\end{enumerate}

\subsection*{Thermodynamics of Gases}
\begin{enumerate}
    \item \textbf{Gases} are popular systems to study in thermodynamics because\dots
        \begin{enumerate}
            \item They can show large changes in volume, making the effects of P-V more apparent (liquids and solids typically do not show changes in their volume very much).
            \item Their behavior is often very similar to an ideal gas, which we often know a lot about.
        \end{enumerate}
    \item Ideal Gas
        \begin{enumerate}
            \item Satisifes the ideal gas law: $PV=nRT$
            \item Is based on the assumption that there are no interactions between gas particles:
                \begin{enumerate}
                    \item The potential energy of a gas is 0 - $E_p = 0$
                    \item Therefore the total energy of an ideal gas is equal to its kinetic energy $E = E_k$
                \end{enumerate}
        \end{enumerate}
    \item Monatomic Ideal Gases
        \begin{enumerate}
            \item From kinetic molecular theory, we know that the average kinetic energy of n moles of a monatomic ideal gas is $E-k (3/2)nRT$ and ... %//TODO - Complete monatomic ideal gas definition from 5/3/23 lecture!
        \end{enumerate}
\end{enumerate}
%//TODO - DO friday and monday lecture notes!

\subsection*{Calculating Enthalpy Changes}
\begin{exmp}
    Use the information below to determine $\Delta_H$ for the reaction:
    \begin{equation*}
        \begin{aligned}
            \ce{3 C + 4 H2 -> C3H8}
        \end{aligned}
    \end{equation*}
\end{exmp}
\begin{equation*}
    \begin{aligned}
        &\ce{C3H8 + 5 O2 -> 3 CO2 + 4 H2O} &\Delta H_1 \\
        %//TODO - Complete Enthalpy Change formulas here!
    \end{aligned}
\end{equation*}

\subsection*{Formation Reactions}
\begin{enumerate}
    \item The \emph{standard enthalpy of formation} $\Delta H^0_f$\footnote{These values for various substances can be found in a table in the textbook or online.} if the change in enthalpy that accompanies the formation of \emph{one mole of a compound} from \emph{its elements in their most stable state} with \emph{all substances in their standard state (1 atm pressure, and ually 25C)}. 
    \item Examples:
        \begin{enumerate}
            \item Formation of Liquid Water:
                \begin{equation*}
                    \begin{aligned}
                        \ce{H2_(g) + 0.5 O2_(g) -> H2O_(l)} \quad \Delta H^0_f = -286 kJ/mol
                    \end{aligned}
                \end{equation*}
            \item Formation of Nitrogen Dioxide:
                \begin{equation*}
                    \begin{aligned}
                        \ce{0.5 N2_(g) + O2_(g) -> NO2_(g)} \quad \Delta H^0_f = 34 kJ/mol
                    \end{aligned}
                \end{equation*}
            \item Formation of Carbon Dioxide:
                \begin{equation*}
                    \begin{aligned}
                        \ce{C_(s) + O2_(g) -> CO2_(g)} \quad \Delta H^0_f = -394 kJ/mol
                    \end{aligned}
                \end{equation*}
            \item Formation of Methane:
                \begin{equation*}
                    \begin{aligned}
                        \ce{C_(s) + 2 H2_(g) -> CH4_(g)} \quad \Delta H^0_f = -75 kJ/mol
                    \end{aligned}
                \end{equation*}
        \end{enumerate}
    \item Note that from these formation examples, we have pure elements in their stable states being converted to 1 mole of a compound! All substances are in standard states!
\end{enumerate}

\subsection*{Standard States (Or Standard Conditions)}
\begin{enumerate}
    \item Many thermodynamic functions depend on the concentrations (or pressures) and temperature of the substances involved.
    \item Furthermore, while we can measure enthalpy \emph{changes} from heat flow experiments, we have no way to measure \emph{absolute} enthalpies.
    \item To calculate thermodynamic properties correctly, we need a common reference point:
        \begin{enumerate}
            \item Compounds: Pure substance in its normal phase at P = 1 atm
            \item Solutions: Concentration of 1 M
            \item Elements: The most stable form of the element at P = 1 atm
        \end{enumerate}
    \item Thermodynamic data are \emph{typically} tabulated for 25C... 
        \begin{enumerate}
            \item \textbf{but note that there is no temperature \emph{specified} for standard state.}
        \end{enumerate}
    \item Since the standard state of a substance depends on its temperature, the phases of all species must be indicated in a thermochemical equation.
    \item Standard states are typical denoted using $"^\circ"$ , such as $\Delta H^\circ_f$
\end{enumerate}

\subsection*{How to use Formation Enthalpies}
\begin{exmp}
    Calculate the reaction enthalpy for the combustion of methane:
        \begin{equation*}
            \begin{aligned}
                \ce{CH4_(g) + 2 O2_(g) -> CO2_(g) + 2 H2O_(l)}
            \end{aligned}
        \end{equation*}
\end{exmp}
Standard enthalpies of formation (for our reaction):
\begin{equation*}
    \begin{aligned}
        &\text{Liquid Water:} &\ce{H2_(g) + 0.5 O2_(g) -> H2O_(l)} & \quad & \Delta H^0_f &= -286 kJ/mol \\
        &\text{Carbon Dioxide:} &\ce{C_(s) + O2_(g) -> CO2_(g)} & \quad & \Delta H^0_f &= -394 kJ/mol \\
        &\text{Methane:} &\ce{C_(s) + 2 H2_(g) -> CH4_(g)} & \quad & \Delta H^0_f &= -75 kJ/mol \\
    \end{aligned}
\end{equation*}

Getting $\Delta H^\circ_{rxn}$:\footnote{n and m represent the stoichiometric coefficients of the products and reactants respectively. In this case, n = 1 and 2 for \ce{CO2} and \ce{H2O}, and m = 1 and 2 for \ce{CH4} and \ce{O2}. Note that since \ce{O2} is in its standard state, its enthalpy is 0???}
\begin{equation*}
    \begin{aligned}
        \Delta H^\circ_{rxn} &= \sum_{products} n * \Delta H^\circ_{f} - \sum_{reactants} m * \Delta H^\circ_{f} \\
        \Delta H^\circ_{rxn} &= (-394) + 2*(-286) - (-75) - 2*(0) kJ/mol \\
        \Delta H^\circ_{rxn} &= -891 kJ/mol \\
    \end{aligned}
\end{equation*}

\section*{Chapter 10 - Isothermal ???}
\subsection*{Reminder: Isothermal Processes}
\begin{enumerate}
    \item "Isothermal" means "constant temperature". In an isothermal process, temperature does not change.
        \begin{equation*}
            \begin{aligned}
                \Delta T = 0
            \end{aligned}
        \end{equation*}
    \item Changes in energy are related to changes in temperature.
        \begin{equation*}
            \begin{aligned}
                \Delta E = C_v \Delta T
            \end{aligned}
        \end{equation*}
    In an isothermal process, $\Delta T$ is 0, and therefore $\Delta E$ is also 0 (unless heat capacity is infinite).
    \item The First Law of Thermodynamics is:
        \begin{equation*}
            \begin{aligned}
                ?? %//TODO - Insert equations here!
            \end{aligned}
        \end{equation*}
\end{enumerate}

\subsection*{Reversible and Irreversible Processes}
\begin{enumerate}
    \item In a thermodynamic process, we usually change one property in a controlled way, keep other properties constant, and observe how yet another property changes.
    \item For example, we observe how the volume V changes as we increase the temperature T of a gas while keeping the pressure P constant. 
    \item So far we classified processes based on which thermodynamic property stays constant: \emph{isothermal} (constant temperature), \emph{isobaric} (constant pressure), or \emph{isochloric} (constant volume).
    \item We also distinguish processes whether they are \emph{reversible} or \emph{irreversible}:
        \begin{enumerate}
            \item In a \emph{reversible} process, the system is in equilibrium (or at least very close to it) the entire time.
                \begin{enumerate}
                    \item This means that we can \emph{reverse the process by changing the property that we }
                    \item In practice, this often means that we perform a process very, \textbf{very} slowly (theoretically, infinitely slowly, adding an infinitely small amount of heat).
                \end{enumerate}
            \item In an \emph{irreversible} process, the system is not always in equilibrium. 
                \begin{enumerate}
                    \item This usually happens when there are sudden changes in a thermodynamic property.
                \end{enumerate}
        \end{enumerate} 
    \item Process functions like heat (q) and work (w) depend on the way a process is performed; therefore they depend on whether a process is reversible or irreversible. 
    \item Differences in state functions ($\Delta E$, $\Delta T$,...) do not depend on the path of a process, they will be the same fr reversible and irreversible processes (in an infinite time frame)
\end{enumerate}

\subsection*{Isothermal process example}
Let's consider the following process (im too lazy to copy paste from slides)
\begin{enumerate}
    \item Constant T, Changing Volume and Pressure 
    \item Work Performed by Mass $M_1$, ... %//TODO - Complete this example!
\end{enumerate}

\subsection*{Zero-step expansion}
\begin{enumerate}
    \item If we simply remove the weight, the external pressure drops to 0.
    \item However, the internal pressure of the gas isn't zero - the system's no longer at equilibrium!
    \item The gas will expand isothermally and irrersibly to fill the entire volume. 
    \item The gas doesn't have to expand against an external pressure. It does here due to the sudden transition from having an external pressure to having none. 
    \item Since the gas expands into a vaccum - no work is being done: \emph{w =  0}
    \item This process is called free expansion
    \item Since q = -w, no heat flows between the system and it's surroundings.
\end{enumerate}

\subsection*{One-step expansion}
Compared to taking away the entire mass, let's take away $M_1/4$.
\begin{enumerate}
    \item Taking away $M_1/4$ drops the external pressure to:
        \begin{equation*}
            \begin{aligned}
                P_{ext} = (M_1/4)g/A = P_1/4
            \end{aligned}
        \end{equation*}
    \item At that moment, the pressure of the gas is greater than the external pressure - causing the gas to expand.
    \item This leads to the mass being lifted until the internal pressure is equal to the external pressure. In this case:
        \begin{equation*}
            \begin{aligned}
                V_{final} = 4V_1
            \end{aligned}
        \end{equation*}
    \item Work:
        \begin{equation*}
            \begin{aligned}
                w &= -P_{ext} \Delta V \\
                    &= -P_1/4(4V_1-V_1) \\
                    &= -3/4P_1 V-1
            \end{aligned}
        \end{equation*}
    \item Heat: 
        \begin{equation*}
            \begin{aligned}
                q = -w >0
            \end{aligned}
        \end{equation*}
        So heat flows from the reservoir into the system.
\end{enumerate}

\subsection*{Two-Step Expansion} %//TODO - Write out two step expansion problem!
In this expansion, we'll do two steps
\begin{enumerate}
    \item Changing mass from $M_1$ to $M_1/2$
        \begin{enumerate}
            \item $P_{ext} = P_1/2, V_{final}=2V_1$
            \item  
        \end{enumerate}
\end{enumerate}


\end{document}