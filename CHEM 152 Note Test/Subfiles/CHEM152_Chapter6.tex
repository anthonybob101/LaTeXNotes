\documentclass[../CHEM152Notes.tex]{subfiles} 
\usepackage{fancyhdr}
\usepackage{graphicx}
\usepackage{amsmath}
\usepackage{mhchem}
\usepackage{amssymb}
\usepackage[margin=1in]{geometry}

\title{UW CHEM 152 Notes}
\author{Anthony Le}

\begin{document}

\pagestyle{fancy}
\fancyhead{}
\fancyhead[R]{UW CHEM 152}
\fancyhead[C]{Chapter 6 - Chemical Equilibrium}
\fancyhead[L]{Anthony Le}

\section*{Chapter 6 - Chemical Equilibrium: Law of Mass Action}

\subsection*{Chemical Reactions}
Consider a chemical reaction of the form \ce{A -> B}.
\begin{enumerate}
    \item The rate of a reaction tells you how many times a reaction happens in a certain amount of time.
    \item if the reaction \ce{A -> B} is first order, the the rate is proportational to the concentration of A: 
    \begin{equation*}
        \begin{aligned}
            \text{Rate} = k * [A]    
        \end{aligned}
    \end{equation*}
    \begin{enumerate}
        \item From the differential rate law, we can find how the concentration of A and B as a function of time. 
        \item Since the rate is dependent of the concentration of A, once all of A is consumed, then we can say the reaction is complete (sometimes)
    \end{enumerate} 
\end{enumerate}

However, this isn't the only type of reaction. There are also:

\subsection*{Reversible Reactions}
\begin{enumerate}
    \item If the reaction can happen both ways: \ce{A -> B} and \ce{B -> A}. 
    \item We write this as \ce{A <=> B}. 
    \item As you can expect, you'll have 2 different rates corresponding to their reactions - so you can find how the concentrations of A and B change with respect to time
\end{enumerate}
These reactions will continue to happen until a time where the forward and reverse rates become equal to each other, and thus the concentrations no longer change AKA chemical equilibrium.

\begin{defn}
    \textbf{Chemical Equilibrium} \\
    Where forward and backward reaction rates become equal to each other and concentrations no longer change.
\end{defn}

\textbf{The goal of this class is to predict the concentrations of all species in equilibrium.}

\subsection*{General Chemical Reactions}
Consider a general \textbf{elementary} (reaction order is equal to the stoichiometric coefficient) chemical reaction: \\
Forward Reaction:
\begin{equation*}
    \begin{aligned}
        \ce{aA + bB -> cC + dD}
    \end{aligned}
\end{equation*}

Backward Reaction:
\begin{equation*}
    \begin{aligned}
        \ce{aA + bB <- cC + dD}
    \end{aligned}
\end{equation*}

Forward Rate:
\begin{equation*}
    \begin{aligned}
        \text{Forward Rate} = k_f[A]^a[B]^b
    \end{aligned}
\end{equation*}

Backward Rate:
\begin{equation*}
    \begin{aligned}
        \text{Backward Rate} = k_b[C]^c[D]^d
    \end{aligned}
\end{equation*}

Chemical Equilibrium (In Terms of Rates):
\begin{equation*}
    \begin{aligned}
        \text{Forward Rate} = \text{Backward Rate} \\
        k_f[A]^a[B]^b = k_b[C]^c[D]^d
    \end{aligned}
\end{equation*}

This essentially describes that rates at where chemical equilibrium happens.
Solving for the stoichiometric constants $k_f$ and $k_b$
\begin{equation*}
    \begin{aligned}
        \frac{[C]^c[D]^d}{[A]^a[B]^b} = \frac{k_f}{k_b}     
    \end{aligned}
\end{equation*}
This gives us the stoichiometric coeffiencents for where equilibrium occurs. This gives us the constant that is dependent on temperature, and not concentration.


\begin{exmp}
    A scientist performs experiments at 500 C to study equilibrium concentrations of the ammonia synthesis reaction. 
    \begin{equation*}
        \begin{aligned}
            \ce{N_2 (g) + 3H_2(g) <=> 2NH_3 (g)}        
        \end{aligned}
    \end{equation*}
\end{exmp}
By finding the equilibrium coefficients for all 3 reactions from the previous equation: 
\begin{equation*}
    \begin{aligned}
        \frac{[C]^c}{[A]^a[B]^b} = \frac{k_f}{k_b} \\
        \frac{[NH_3]^2}{[N_2][H_2]^3} = \frac{k_f}{k_b}
    \end{aligned}
\end{equation*}
From this example (see class slides), we get the Law of Mass Action:

\begin{defn}
    \textbf{Law of Mass Action} \\
    For a reversible gas- or solution-phase chemical reaction 
    \begin{equation*}
        \begin{aligned}
            \ce{aA + bB <=> cC + dD}
        \end{aligned}
    \end{equation*}
    The ratio of concentrations K at equilibrium is constant (regardless of inital concentrations) at a given temperature, where K is defined as \\
    \begin{equation*}
        \begin{aligned}
            K = \frac{[C]^c[D]^d}{[A]^a[B]^b} 
        \end{aligned}
    \end{equation*}
    For gas phase reactions, we also use partial pressures instead of concentrations to describe the chemical equilibrium. \\
    The pressure-based equilibrium constant is:
    \begin{equation*}
        \begin{aligned}
            K_P = \frac{\left(\frac{P_C}{P_0}\right)^c \left(\frac{P_D}{P_0}\right) ^d}{\left(\frac{P_A}{P_0}\right) ^a \left(\frac{P_B}{P_0}\right) ^b}
        \end{aligned}
    \end{equation*}
    Where $P_0 = 1$ atm as the reference temperature. 
\end{defn}

\begin{proof}
    The equilibrium concentrations themselves \textbf{DO} depend on the inital concentrations are. \\
    However the ratio of equilibrium concentrations \textbf{DO NOT} depend on the inital conditions.
\end{proof}

However, using concentrations doesn't exactly make the units work out perfectly - so we use the "activities" of the molecular species instead of their concentrations. For ideal gases and molecules in solution, the activity is equal to the concentration divided by a reference concentration $c_0$, which by common convention 1 molar.
\begin{equation*}
    \begin{aligned}
        a_A  = \frac{[A]}{c_0} \\
        a_B  =\frac{[B]}{c_0} \\
        a_C  = \frac{[C]}{c_0} \\
        a_D  =\frac{[D]}{c_0} \\ 
    \end{aligned}
\end{equation*}
One thing to note is that even though we continue to write the equilibrium constant in form:
\begin{equation*}
    \begin{aligned}
        K = \frac{[C]^c[D]^d}{[A]^a[B]^b} 
    \end{aligned}
\end{equation*}
We actually mean to write the equilibrium constant in form:
\begin{equation*}
    \begin{aligned}
        K = \frac{[a_C]^c[a_D]^d}{[a_A]^a[a_B]^b} 
    \end{aligned}
\end{equation*}

\begin{exmp}
    The reaction 
    \begin{equation*}
        \begin{aligned}
            \ce{H_2O (g) + CO (g) <=> H_2 (g) + CO_2 (g)}
        \end{aligned}
    \end{equation*}
    Has the equilibrium constant K = 4 at 800K. What will the equilibrium concentrations be if you put 1 mole of \ce{H_2O} and 1 mole of \ce{CO} into a container of volume 1L at that temperature?
\end{exmp}

Due to stoichiometry, \ce{[H_2O] = [CO]}, \ce{[H_2] = [CO_2]}, \ce{[H_2O] + [CO] + [H_2] + [CO_2] = 2M}
\begin{enumerate}
    \item Due to how these these 2 chemicals have the same stoichiometric coefficients (they both have 1), they will always have the same amount at all times. 
    \item The total number of molecules you should have should always be the same.
    \begin{enumerate}
        \item Since in both forward and backward reactions, you have 2 molecules being transformed into 2 other molecules.
    \end{enumerate}

\end{enumerate}
However, you need the law of mass action in order to find the equilibrium concentrations to act as the fourth equation which you can use to solve for them!
 
\begin{enumerate}
    \item Since we're given K = 4 and we're trying to find the equilibrium concentrations:
    \begin{equation*}
        \begin{aligned}
            K = \frac{[H_2][CO_2]}{[H_2O][CO]} \\
            4 = \frac{[H_2][CO_2]}{[H_2O][CO]} \\
            4 = \frac{[H_2]^2}{[H_2O]^2} \\
            [H_2] = 2[H_2O]
        \end{aligned}
    \end{equation*}
    \item Note that since we're given that \ce{[H_2O] = [CO]} and \ce{[H_2] = [CO_2]}, we can replace \ce{[CO]} and \ce{[CO_2]} so we can simplify the law of mass action.
    \item Given that we know \ce{[H_2O] + [CO] + [H_2] + [CO_2] = 2M}, let replace \ce{[CO]} and \ce{[CO_2]}
    \begin{equation*}
        \begin{aligned}
           \ce{[H_2O] + [CO] + [H_2] + [CO_2] = 2M} \\
           \ce{[H_2O] + [H_2O] + 2[H_2O] + 2[H_2O] = 2M} \\
           \ce{6[H_2O] = 2M} \\
           \ce{[H_2O] = \frac{1}{3}M}
        \end{aligned}
    \end{equation*}
\end{enumerate}

Equilibrium is reached when the forward and reverse reactions occur at the same rate, so there is no net change in the concentrations of reactants and products. However, where this balancing point occurs can be different for each reaction:
\begin{equation*}
    \begin{aligned}
        \ce{2H2 + O2 <=> 2H2O} \quad K = 2.4 * 10^47, T = 500 K \\
        \ce{Cl2 <=> 2Cl} \quad K = 1.8 * 10 ^-9, T = 1000 K \\
        \ce{H2 + I2 <=> 2HI} \quad K = 0.3, T = ??? K
    \end{aligned}
\end{equation*}

For a general chemical reaction in which the number of gas-phase molecules changes by $\Delta n$: 
\begin{equation*}
    \begin{aligned}
        K_C = (\frac{P_0}{c_0RT})^{\Delta N} K_P \quad K_P = (\frac{c_0RT}{P_0})^{\Delta N} K_c
    \end{aligned}
\end{equation*}

\subsection*{Activities and Heterogeneous Equilibrium}
For ideal gases and molecules in solution, the activity is equal to the concentration divided by a reference concentration $c_0 = 1M$
\begin{equation*}
    \begin{aligned}
        a_A &= \frac{[A]}{c_0} \\
            &=\frac{[A]}{1} \\
    \end{aligned}
\end{equation*}
For solids and pure liquids, the activity is simply equal to 1.
\begin{exmp}
    What is K for the thermal decomp. for calcium carbonate to calcium oxide and carbon dioxide?
\end{exmp}
\begin{equation*}
    \begin{aligned}
        &\ce{CaCO3 <=> CaO + CO2} \\
        K &= \frac{a_{CaO}a_{CO_2}}{a_{CaCO_3}} \\
          &= a_{CO_2} = \frac{\ce{CO2}}{1M}
    \end{aligned}
\end{equation*}
Note - this shows us that the equilibrium constant is depentent on the gas molecules and independent on the others.

\subsection*{Manipulating Equilibrium Constants}
\begin{exmp}
    Ammonia Synthesis Reaction 
    \begin{equation*}
        \begin{aligned}
            \ce{N2 + 3 H2 <=> 2 NH3} \quad K = \ce{\frac{[NH3]^2}{[N2][H2]^3}}
        \end{aligned}
    \end{equation*}
\end{exmp}
Swapping reactants and products \\
Note - swapping reactants and products gives the inverse of the equilibrium constant.

\begin{equation*}
    \begin{aligned}
        \ce{2 NH3 <=> N2 + 3 H2} \quad K^f = \ce{\frac{[N2][H2]^3}{[NH3]^2}}
    \end{aligned}
\end{equation*}

\begin{exmp}
    Multi-step formulation of nitrogen dioxide: 
    \begin{equation*}
        \begin{aligned}
            &\ce{N2 + O2 <=> 2 NO} \quad K_1 = \ce{\frac{[NO]^2}{[N2][O2]}} \\
            &\ce{2 NO + O2 <=> 2 NO2} \quad K_2 = \ce{\frac{[NO2]^2}{[NO]^2[O2]}} \\
            &\ce{N2 + 2 O2 <=> 2 NO2} \quad K = \ce{\frac{[NO2]^2}{[N2][O2]^2}} = K_1K_2
        \end{aligned}
    \end{equation*}
\end{exmp}
Note - When adding reactions, multiply equilibrium constants.

\begin{exmp}
    Multiplying first step reaction by 3: 
    \begin{equation*}
        \begin{aligned}
            &\ce{3N2 + 3O2 <=> 6 NO} \quad K = \ce{\frac{[NO]^6}{[N2]^3[O2]^3}} = K_1^3 \\
        \end{aligned}
    \end{equation*}
\end{exmp}
Note - when multiplying a reaction by a number n, raise the equilibrium constant to the n-th power.

\subsection*{The Reaction Quotient Q}
We've seen from the law of mass action: 
\begin{equation*}
    \begin{aligned}
        \ce{aA + bB <=> cC + dD}
    \end{aligned}
\end{equation*}
The concentrations in \textbf{in chemical equilibrium} satisify:
\begin{equation*}
    \begin{aligned}
        K = \frac{[C]^c[D]^d}{[A]^a[B]^b} 
    \end{aligned}
\end{equation*}
However, what if you're not in chemical equilibrium? We can calculate the reaction quotient Q:
\begin{equation*}
    \begin{aligned}
        Q = \frac{[C]^c_0[D]^d_0}{[A]^a_0[B]^b_0} 
    \end{aligned}
\end{equation*}
This represents what has to happen to reach equilibrium.

\begin{defn}
    \textbf{Reaction Quotient (Q)} 
    \begin{equation*}
        \begin{aligned}
            Q = \frac{[C]^c_0[D]^d_0}{[A]^a_0[B]^b_0} 
        \end{aligned}
    \end{equation*}
    \begin{enumerate}
        \item If Q $<$ K, there is not enough product or too much reactant compared to equilibrium. The system will shift "toward the right" AKA the system will be biased to have more forward reactions happening relative to backward reactions.
        \item If Q $>$ K, there is too much product/too little reactant compared to equilibrium. The system will shift "toward the left" ALA the system will be biased to have more backward reactions happening relative to forward reactions.
        \item If Q = K, the system is in equilibrium.
    \end{enumerate}
\end{defn}

for ICE Table - as many columns as there are species in chemical reaction:
I - Inital: represents inital concentrations in M
C - Change: represents stoichiometric coefficients multiplied by x moles of molecules
E - Equilibrium: Represents inital value added with change value 

\subsubsection*{Quizzes}
\begin{enumerate}
    \item First quiz opens 4/5/2023 and is open between 3PM and Midnight and can be found on Canvas.
    \item Only online resources allowed is Canvas quiz page itself and the online textbook.
    \item Will need to have something to write on and scientific calculator.
    \item Also \textbf{will not cover the lecture on the day the quiz is on}. But does cover previous lectures.
\end{enumerate}

\begin{exmp}
    Ammonia synthesis reaction: \\
    \begin{equation*}
        \begin{aligned}
            \ce{N2 + 3 H2 <=> 2 NH3} \quad K = 0.6 at 500C \\
            \ce{N2 = 0.8M \quad H2 = 2.4M \quad NH3 = 0.6M} \\        
        \end{aligned}
    \end{equation*}
\end{exmp}

Finding Q from inital concentrations:
\begin{equation*}
    \begin{aligned}
        Q = \frac{[C]^c_0[D]^d_0}{[A]^a_0[B]^b_0} \\
        Q = \ce{\frac{[NH3]^2_0}{[N2]_0[H2]^3_0}} \\
        Q = \ce{\frac{[0.6]^2_0}{[0.8]_0[2.4]^3_0}} \\
    \end{aligned}
\end{equation*}

\begin{exmp}
    Hydrogen Flouride Reaction and ICE Table 
    \begin{equation*}
        \begin{aligned}
            \ce{H2 + F2 <=> 2HF}
        \end{aligned}
    \end{equation*}
    3 moles of each species is added to a 1.5 L flask. K = 115. Find equilibrium concentrations.
\end{exmp}
\begin{enumerate}
    \item Finding Inital Concentrations:
    \begin{equation*}
        \begin{aligned}
            \ce{[H2] = \frac{3M}{1.5L} =  2M \quad [F2] = \frac{3M}{1.5L} = 2M \quad [HF] = \frac{3M}{1.5L} = 2M}
        \end{aligned}
    \end{equation*}
    \item Create ICE Table \\
    \begin{tabular}{c|c@{}c@{}c@{}c@{}c}
        \hline
        X   &   $[H_2]$ & ${}+{}$ & $[F_2]$ & ${}\leftrightharpoons{}$ & $[2HF]$ \\
        \hline
        I   &       1       &&   2                            &&  0       \\
        C   &       -x      &&   -x                           &&  2x      \\
        E   &       1 - x     &&   2 - x                        &&  2x      \\
        \hline
      \end{tabular}
    \item Write Equilibrium Expression:
    \begin{equation*}
        \begin{aligned}
            K   &= \ce{\frac{[HF]^2}{[H2][F2]}} \\
                &= \ce{\frac{[2x]^2}{[1-x][2-x]}}
        \end{aligned}
    \end{equation*}
    \item Solve for x:
    \begin{equation*}
        \begin{aligned}
            &K = \ce{\frac{[2x]^2}{[1-x][2-x]}} \\
            & = K(1-x)(2-x) = (2x)^2 \\
            & = 2K - Kx - 2Kx + Kx^2 =  \\
            & = (K-4)x^2 - 3Kx + 2K = 0, \quad a = (K-4) \quad b = -3K \quad c = 2K \\
        \end{aligned}
    \end{equation*}
    \item Plugging in K for a, b, and c and finding X via quadratic equation:
    \begin{equation*}
        \begin{aligned}
            x = \frac{-b\pm\sqrt{b^2-4ac}}{2a} \\
            x = \frac{345\pm\sqrt{345^2-4(111*230)}}{2(111)} \\
            x_1 = 2.14 \quad x_2 = 0.968
        \end{aligned}
    \end{equation*}
    \item However, we are x = 2.14 cannot be solution as the equilibrium concentration for \ce{H2} is $1-x$! Thus the solution is x = 0.968.
    \item Calculating Equilibrium Concentrations:
    \begin{equation*}
        \begin{aligned}
            \ce{[H2] = (1-0.968)M = 0.032M} \\
            \ce{[F2] = (2-0.968)M = 1.032M} \\
            \ce{[HF] = (2*0.968)M = 1.936M} \\
        \end{aligned}
    \end{equation*}
    \item Checking Results (Optional)
    \begin{equation*}
        \begin{aligned}
            K   &= \ce{\frac{[HF]^2}{[H2][F2]}} \\
            &= \ce{\frac{1.936^2}{0.032*1.032}} \\
            &= 113 \approx K
        \end{aligned}
    \end{equation*}
\end{enumerate}

\begin{exmp}
    Decomposition of Nitrosyl Chloride \\
    \begin{equation*}
        \begin{aligned}
            \ce{2 NOCl <=> 2 NO + Cl2} \quad K = 1.6 * 10^-5 @ 35C \\
        \end{aligned}
    \end{equation*}
    Walter White adds 1 mol \ce{NOCl} into a 2L flask, what are the equilibrium concentrations?
\end{exmp}

\begin{enumerate}
    \item Finding Inital Concentrations:
    \begin{equation*}
        \begin{aligned}
            \ce{[NOCl] = \frac{3M}{1.5L} =  2M \quad [F2] = \frac{3M}{1.5L} = 2M \quad [HF] = \frac{3M}{1.5L} = 2M}
        \end{aligned}
    \end{equation*}
    \item Create ICE Table \\
    \begin{tabular}{c|c@{}c@{}c@{}c@{}c}
        \hline
        X   &   $[2 NOCl]$  & ${}\leftrightharpoons{}$ & $[2NO]$ & ${}+{}$ & $[Cl_2]$ \\ %//TODO - correct format of ICE table to match example slides
        \hline
        I   &   0.5         &&   0                            &&  0     \\
        C   &   -2x         &&   2x                           &&  x      \\
        E   &   0.5 - 2x    &&   2x                           &&  x      \\
        \hline
      \end{tabular}
    \item Write Equilibrium Expression:
    \begin{equation*}
        \begin{aligned}
            K   &= \ce{\frac{[NO]^2[CL2]}{[NOCl]^2}} \\
                &= \ce{\frac{(2x)^2x}{(0.5-2x)^2}} \\
                &= \ce{\frac{4x^3}{(0.5-2x)^2}} \\
        \end{aligned}
    \end{equation*}
    \item Solve for x:
    \begin{equation*}
        \begin{aligned}
            K &= \ce{\frac{4x^3}{(0.5-2x)^2}} \\
              &= 4x^3 - 4Kx^2 + 2Kx - 0.25K = 0%//TODO - Add in cubic formula from example here!
        \end{aligned}
    \end{equation*}
    \item However, you get a cubic function here which makes things much harder to solve for X! We've seen that the equilibrium constant is very small, so we should expect to the equilibrium position to be far to the left, and x to be small. \\    
    Lets simplify $0.5-2x \approx 0.5$ (Small X Approximation) and try again.
    \item Write Equilibrium Expression:
    \begin{equation*}
        \begin{aligned}
            K   &= \ce{\frac{[NO]^2[CL2]}{[NOCl]^2}} \\
                &= \ce{\frac{(2x)^2x}{(0.5)^2}} \\
                &= \ce{\frac{4x^3}{0.25}} \\
            K    &= 16x^3
        \end{aligned}
    \end{equation*}
    \item Solving for X:
    \begin{equation*}
        \begin{aligned}
            &K = 16x^3 \\
            &\rightarrow x = \left(\frac{K}{16}\right)^{1/3} \\
            &\rightarrow x = 0.01
        \end{aligned}
    \end{equation*}
    \item Calculating Equilibrium Concentrations:
    \begin{equation*}
        \begin{aligned}
            \ce{[H2] = (2-0.01)M = 0.48M} \\
            \ce{[F2] = (2*0.01)M = 1.032M} \\
            \ce{[HF] = (2*0.968)M = 1.936M} \\
        \end{aligned}
    \end{equation*}
    \item Checking Results (Optional)
    \begin{equation*}
        \begin{aligned}
            K   &= \ce{\frac{[HF]^2}{[H2][F2]}} \\
            &= \ce{\frac{1.936^2}{0.032*1.032}} \\
            &= 113 \approx K
        \end{aligned}
    \end{equation*}
\end{enumerate}

%//TODO - take notes on friday's lecture!

\subsection*{The Effect of Changing Concentrations}
\begin{exmp}
    Formation of hydrogen cyanide from methane and ammonia 
    \begin{equation*}
        \begin{aligned}
            \ce{CH4 + NH3 <=> HCN + 3 H2}
        \end{aligned}
    \end{equation*}
    Let's assume the system is at equilibrium. Therefore, the concentration of each species is equal to the equilibrium concentrations.
    \begin{equation*}
        \begin{aligned}
            Q = \ce{\frac{[HCN][H2]^3}{[CH4][NH3]}} = K
        \end{aligned}
    \end{equation*}
\end{exmp}
what happens if we add some more ammonia to the system?
\begin{enumerate}
    \item The current \ce{NH3} concentration increases, and the reaction quotient Q decreases.
    \item Now we have Q < K - the system is no longer in equilibrium!
    \item System evolves towards the new equilibrium by creating more product (it shifts "toward the right", \emph{away from the added component}).
\end{enumerate}
What happens if instead we now add some more hydrogen to the system?
\begin{enumerate}
    \item The current \ce{H2} concentration increases, and the reaction quotient Q increases.
    \item Now we have Q > K - the system is no longer in equilibrium!
    \item System evolves towards the new equilibrium by creating more reactant (it shifts "towards the left", \emph{away from the added component}).
\end{enumerate}

\subsection*{The Effect of Changing Pressure}
There are many ways to change the pressure of the gas-phase chemical system:
\begin{enumerate}
    \item \textbf{Add or remove a gas that is part of the reaction} - see the previous section.
    \item \textbf{Add an inert gas that doesn't participate in the reaction.} - This changes the total pressure, but not the partial pressures of reactants, products, nor their concentrations, so it does not change the reaction quotient.
    \item \textbf{Change the volume of the system.} 
        \begin{exmp}
            Ammonia Synthesis Reaction:
                \begin{equation*}
                    \begin{aligned}
                        \ce{N2 + 3H2 <=> 2 NH3}
                    \end{aligned}
                \end{equation*}
            Let's assume the system is at equilibrium: 
                \begin{equation*}
                    \begin{aligned}
                        Q = \ce{\frac{[NH3]^2}{[N2][H2]^3}} = K
                    \end{aligned}
                \end{equation*}
        \end{exmp}
\end{enumerate}
\textbf{What happens if we now decreases the volume by a factor of 2?}
\begin{enumerate}
    \item Each concentration increases by a factor of 2. 
    \item The reaction quotient Q changes from K to (4/16)K.
    \item Now we have Q < K - the system is no longer in equilibrium!
    \item System evolves towards the new equilibrium by creating more product (it shifts "towards the right", \emph{towards the side of fewer gas molecules})
\end{enumerate}

\subsection*{The Effect of Changing Temperature}
Many chemical reactions release of absorb energy in the form of heat:
\begin{enumerate}
    \item If energy is released, we call the reaction \emph{exothermic}. 
        \begin{exmp}
            Combustion of methane: 
            \begin{equation*}
                \begin{aligned}
                    \ce{CH4 + O2 <=> 2 H2O + CO2 + energy} 
                \end{aligned}
            \end{equation*}
        \end{exmp}
    \item If energy is absorbed, we call the reaction \emph{endothemic}.
        \begin{exmp}
            Decomposition of dinitrogen tetraoxide
            \begin{equation*}
                \begin{aligned}
                    \ce{N2O4 + energy <=> 2 NO2}
                \end{aligned}
            \end{equation*}
        \end{exmp}
\end{enumerate}
For now, we can think of heat energy just like a reactant (for endothermic reactions) or a product (for exothermic reactions). 
\newline
Increasing the temperature shifts the chemical system... 
\begin{enumerate}
    \item Towards the left for exothermic reactions (energy is a product)
    \item Towards the right for endothemic reactions (energy is a reactant)
\end{enumerate}
\end{document}