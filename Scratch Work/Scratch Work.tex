\documentclass{report} % //TODO find why latex gives errors for report and using textsc!
\usepackage{fancyhdr}
\usepackage{graphicx}
\usepackage{amsmath}
\usepackage[margin=1in]{geometry}

\title{UW IND E 250 Notes}
\author{Anthony Le}
 
\newtheorem{exmp}{Example}
\newtheorem{exrc}{Excersize}
\newtheorem{proof}{Proof}
\newtheorem{defn}{Definition}


\begin{document}

\pagestyle{fancy}
\fancyhead{}
\fancyhead[R]{Scratch Work}
\fancyhead[L]{Anthony Le}

\begin{center}
    \LARGE{\textbf{Scratch Work}}
\end{center}

Givens:
\begin{equation*}
    \begin{aligned}
        LL = 100 psf \quad DL = ?? psf\\
        L = 32 ft \quad TW = 11 ft \\
    \end{aligned}
\end{equation*}
Computing Demand:
\begin{enumerate}
    \item Finding equilvalent uniform distributed load:
    \begin{equation*}
        \begin{aligned}
            w = LL * TW = 100psf * 11ft = 1100lb/ft \\
        \end{aligned}
    \end{equation*}
    \item Finding Reaction Forces: \\
    Assuming uniform distributed load,
    \begin{equation*}
        \begin{aligned}
            A_y = B_y &= \frac{wL}{2} = \frac{1100*32}{2} \\
                      &= 17.6K 
        \end{aligned}
    \end{equation*}
    \item Finding Moment Demand:
    \begin{equation*}
        \begin{aligned}
            M_u &= \frac{wL^2}{8} = \frac{1100*32^2}{8} \\
                &= 140.8K-ft
        \end{aligned}
    \end{equation*} 
    \item Finding Section:
    \begin{equation*}
        \begin{aligned}
            \Phi M_n \geq M_u \\
            \Phi M_n \geq 140.8K-ft \\
        \end{aligned}
    \end{equation*}
    Using AISC 7-16 Table 3-10 for 140.8 K-ft at unbraced length of 32 feet:
    \item 

\end{enumerate}

\newpage
\section*{Horizontial Distribution of Lateral Forces}
\subsection*{Finding Center of Mass:}
For preliminary design, we assumed that the center of mass was in the center of the building. For detailed design, we'll want to find and compute the center of mass via weighted average of seismic weights.

\begin{center}
    Preliminary Design COM:    
\end{center}
\begin{equation*}
    \begin{aligned}
        \text{Width} = 231' &\quad \text{Height} = 102' \\
        \bar{X} = \frac{\text{Width}}{2} &\quad \bar{Y} = \frac{\text{Height}}{2} \\
        \bar{X} = \frac{231}{2} &\quad \bar{Y} = \frac{102}{2} \\
        \bar{X} = 115.5' &\quad \bar{Y} = 51'
    \end{aligned}
\end{equation*}
From this, we get that \fbox{$\bar{X} = 115.5'\quad \bar{Y} = 51'$}

\subsection*{Finding Center of Rigidity/Rotation}
\subsubsection*{Finding Frame Stiffnesses in X and Y direction}
For preliminary design, we can approximate the stiffness of BRBF's with using $K \approx  1/L$, where L is the length of the bay the BRBF spans.
\begin{equation*}
    \begin{aligned}
        K_{xi} &= 1/L = 1/33 \\
        K_{xi} &= 0.030 \\
        K_{yi} &= 1/L = 1/26 \\ 
        K_{yi} &= 0.038 
    \end{aligned}
\end{equation*}
From this, we get that \fbox{$ K_{xi} = 0.030 \quad K_{yi} = 0.038$} \\
\subsubsection*{Computing Center of Rigidity for $X_R$}
The process for finding the center of rigidity will be the same for our preliminary and detailed design:
\begin{equation*}
    \begin{aligned}
        X_R &= \frac{\sum{K_{yi}*X_i}}{\sum{K_{yi}}} \\
        X_R &= \frac{0.038*(66+165+198)}{3*0.038} \\
        X_R &= 143'
    \end{aligned}
\end{equation*}

\subsubsection*{Computing Center of Rigidity for $Y_R$}
The process for finding the center of rigidity will be the same for our preliminary and detailed design:
\begin{equation*}
    \begin{aligned}
        Y_R &= \frac{\sum{K_{yi}*X_i}}{\sum{K_{yi}}} \\
        Y_R &= \frac{0.030*(102+12+12+102)}{4*0.030} \\
        Y_R &= 57'
    \end{aligned}
\end{equation*}
From this, we get that \fbox{$X_R = 143 \quad Y_R = 57$} \\

\subsubsection*{Computing Eccentricity for Inherent Torsion}
To find the eccentricity, we'll need to find the distance between the center of mass and the center of rigidity. However, we should note at this point, that we are going to assume that counter-clockwise twisting will be in the positive direction (following the convention we've been working with such as in the right-hand rule).
\begin{equation*}
    \begin{aligned}
        e_{Y,Inherent} &= Y_R - \bar{Y} \\
        e_{X,Inherent} &= \bar{X} - X_R 
    \end{aligned}
\end{equation*}
Solving for eccentricities:
\begin{equation*}
    \begin{aligned}
        e_{Y,Inherent} &= 57 - 51 \\
        e_{Y,Inherent} &= 6' \\
        e_{X,Inherent} &= 115.5 - 143 \\
        e_{X,Inherent} &= -27.5'
    \end{aligned}
\end{equation*}
From this, we get that \fbox{$e_{X,Inherent} = -27.5' \quad e_{Y,Inherent} = 6'$} \\

\subsection*{Computing Eccentricity for Accential Torsion}
To find the eccentricity, we'll need to multiply the length and width by 5\% in order to get the accidential torsion eccentricity. From this, we'll get equations in the form of:
\begin{equation*}
    \begin{aligned}
        e_{Y,Accidential} &= 0.05 * L_Y \\
        e_{X,Accidential} &= 0.05 * L_X 
    \end{aligned}
\end{equation*}
Solving for eccentricities:
\begin{equation*}
    \begin{aligned}
        e_{Y,Accidential} &= 0.05 * 102' \\
        e_{Y,Accidential} &= 5.1' \\
        e_{X,Accidential} &= 0.05 * 231' \\
        e_{X,Accidential} &= 11.550'
    \end{aligned}
\end{equation*}
From this, we get that \fbox{$e_{X,Accidential} = 11.550' \quad e_{Y,Accidential} = 5.1'$} \\

\subsection*{Computing Total Torsion}
To find the total torsion caused by accidential and inherent torsion, we'll combine the inherent and accidential eccentricities. From this, we'll get equations in the form of:
\begin{equation*}
    \begin{aligned}
        T_X &= V_x (e_{Y,Inherent} \pm e_{Y,Accidential}) \\
        T_Y &= V_y (e_{X,Inherent} \pm e_{X,Accidential}) 
    \end{aligned}
\end{equation*}
Plugging in eccentricities:
\begin{equation*}
    \begin{aligned}
        T_X &= V_x (6' \pm 5.1') \\
        T_X &= V_x (11.10) \text{ For +X} \quad V_x (0.90) \text{ For -X} \\
        T_Y &= V_y (-27.5' \pm 11.550') \\
        T_Y &= V_y (-15.95) \text{ For +Y} \quad V_y (-39.05) \text{ For -Y}
    \end{aligned}
\end{equation*}
 







\newpage
\section*{Preliminary Structural Calculations}
\subsection*{Gravity System Design}
\subsubsection*{Sizing Roof Beams (Covering Gridlines A6-B6)}
%insert snapshot of beam layout
Given:
\begin{equation*}
    \begin{aligned}
        &LL = 100psf &\quad SDL = 55.83psf \\
        &W_{trib} = 11ft &\quad L_b = 32ft 
    \end{aligned}
\end{equation*}
Find: 
\begin{enumerate}
    \item $M_u$
    \item Select W-Section using AISC A360-16 Table 3-10
\end{enumerate}
Solve: 
\begin{enumerate}
    \item Controlling LRFD Load Combo:
        \begin{equation*}
            \begin{aligned}
                1.4D + 1.6l &= 1.4(55.83psf) + 1.6(100psf) \\
                            &= 238.16psf 
            \end{aligned}
        \end{equation*}
        \item Finding LRFD Factored Distributed Load, $W_u$:
        \begin{equation*}
            \begin{aligned}
                W_u &= (238.16psf)(11ft) \\
                W_u &= 2619.78 lb/ft 
            \end{aligned}
        \end{equation*}
    \item Finding Reaction Forces:
        \begin{equation*}
            \begin{aligned}
                A_y = B_y &= \frac{wL}{2} \\
                        &= \frac{(2619.78 lb/ft)(32 ft)}{2} \\
                A_y = B_y &= 41.92K
            \end{aligned}
        \end{equation*}
    \item Finding LRFD Factored Moment, $M_u$:
        \begin{equation*}
            \begin{aligned}
                M_u &= \frac{wL^2}{8} \\
                    &= \frac{(2619.78 lb/ft)(32 ft)^2}{8} \\
                M_u &= 335.33 K-ft
            \end{aligned}
        \end{equation*}
    \item Selecting W-Section:\\
    Assuming an unbraced length using $L_b$ = 32 ft from A360-16 Table 3-10, where:
        \begin{equation*}
            \begin{aligned}
                \Phi M_n \geq M_u \\
                \Phi M_n \geq 335.53 K-ft
            \end{aligned}
        \end{equation*}
\end{enumerate}
\begin{center}
    \fbox{Select W18X86}
\end{center}

\newpage
\subsubsection*{Sizing Roof Girders (B6-B7)}
%Insert Snapshot of Gider Loading
Given:\footnote{Note that the applied forces P were calculated using the reaction of roof beams A6-B6 and the self-weight of the W18X86 beam.}
\begin{equation*}
    \begin{aligned}
        &P = 44.67K &\quad L = 33ft \\%41.92 + \frac{86lb/ft * 32ft}{2} //TODO - Clarify with Leo and Caitlin how they got their P values?
        &a = 11ft 
    \end{aligned}
\end{equation*}
Find: %//TODO - I'm a little unsure about why we're finding the required moment of inertia to select the section based off of the deflection here!
\begin{enumerate}
    \item $L/360$
    \item $M_{max}$
    \item $\Delta _{max}$
    \item $I_{required}$
    \item Select W-Section using AISC A360-16 \textbf{[INSERT TABLE NAME HERE]}
\end{enumerate}
Solve:
\begin{enumerate}
    \item Finding $L/360$
        \begin{equation*}
            \begin{aligned}
                \Delta _{required} &= L/360 \\
                            &= (33ft * 12in)/360\\
                \Delta _{required} &= 1.10in    
            \end{aligned}
        \end{equation*}
    \item Finding $M_{max}$
        \begin{equation*}
            \begin{aligned}
                M_{max} &= P * a \\
                        &= 44.67K * (11ft) \\  %//TODO - How did Leo get a number for P? He doesn't make it clear in the beginning
                        &= 163.79 K-ft
            \end{aligned}
        \end{equation*} 
    \item Finding $\Delta _{max}$
        \begin{equation*}
            \begin{aligned}
                \Delta _{max} &= \frac{Pa}{24EI} * (3L^2-4a^2)\\ %//TODO - How did Leo get a number for I? He doesn't make it clear in the beginning
                        &= \frac{44.67K * 11ft * 12in}{24 * 29000ksi * 1330in^4} * (3(33ft * 12in)^2-4(11ft * 12in)^2) \\
                \Delta _{max} &= 2.01in    
            \end{aligned}
        \end{equation*}
    \item Finding $I_{required}$
        \begin{equation*}
            \begin{aligned}
                \Delta _{max} &> \Delta _{required} \\
                2.01 &> 1.10 \\
            \end{aligned}
        \end{equation*}
        Since $\Delta _{max} > \Delta _{required}$, we'll set $\Delta _{max} = \Delta _{required}$ and solve for I. Rearranging the equation and solving, we get:
        \begin{equation*}
            \begin{aligned}
                I_{required} = 2450in^4
            \end{aligned}
        \end{equation*}
    \item Selecting W-Section using AISC A360-16 \textbf{[INSERT TABLE NAME HERE]}
    Assuming ???? from A360-16 Table ????, where:
        \begin{equation*}
            \begin{aligned}
                I_{required} = 2450in^4
            \end{aligned}
        \end{equation*}
\end{enumerate}
\begin{center}
    \fbox{W27X84}
\end{center}
Loading girder as simple beam with two equal concentrated load symmetrically placed  

\newpage
\subsubsection*{Sizing Roof Girders (Gridlines A5-A6)}
Given:
\begin{equation*}
    \begin{aligned}
        &LL = 20psf \quad &SDL = 57.3psf \\
        &L_b = 33ft \quad &W_{trib} = 16ft \\
    \end{aligned}
\end{equation*}
Find:
\begin{enumerate}
    \item $M_u$
    \item Select W-Section using AISC A360-16 Table 3-10
\end{enumerate}
Solve:
\begin{enumerate}
    \item Finding LRFD psf with Controlling Load Combo:
        \begin{equation*}
            \begin{aligned}
                &1.4L + 1.6L \\ 
                \rightarrow &1.4(57.3) + 1.6(20) \\
                &=112.22psf\\   
            \end{aligned}
        \end{equation*}
    \item Finding LRFD Factored Distributed Load, $W_u$:
        \begin{equation*}
            \begin{aligned}
                W_u &= (112.22psf)(16ft) \\
                W_u &= 1745.52 lb/ft
            \end{aligned}
        \end{equation*}
    \item Finding Reaction Forces:
        \begin{equation*}
            \begin{aligned}
                A_y = B_y &= \frac{wL}{2} \\
                        &= \frac{(1745.52 lb/ft)(33 ft)}{2} \\
                A_y = B_y &= 29.636K
            \end{aligned}
        \end{equation*}
    \item Finding LRFD Factored Moment, $M_u$:
        \begin{equation*}
            \begin{aligned}
                M_u &= \frac{wL^2}{8} \\
                    &= \frac{(1745.52 lb/ft)(33 ft)^2}{8} \\
                M_u &= 244.42 K-ft
            \end{aligned}
        \end{equation*}
    \item Selecting W-Section:\\
        Assuming an unbraced length using $L_b$ = 33 ft from A360-16 Table 3-10, where:
            \begin{equation*}
                \begin{aligned}
                    \Phi M_n \geq M_u \\
                    \Phi M_n \geq 244.42 K-ft
                \end{aligned}
            \end{equation*}
            \begin{center}
                \fbox{Select W12X65}
            \end{center}
    \item Deflection Checks: \\
        Loading the girder with two equal concentrated load symmetrically with uniform distributed load:
        \begin{enumerate}
            \item Finding Self Weight Distributed Load
                \begin{equation*}
                    \begin{aligned}
                        P_{self} &= (86lb/ft)(32ft) \\ %//TODO - isn't this length of 32 ft supposed to be 33 ft?
                        P_{self} &= 2752lbs
                    \end{aligned}
                \end{equation*}
            \item Finding Cladding Distributed Load
                \begin{equation*}
                    \begin{aligned}
                        P_{cladding} &= (16psf)(295.5ft^2) \\ %//TODO - Shouldn't this be supposed to be 295.5*2 ft2 since we're spanning 33, not 16.5ft
                        P_{cladding} &= 4728lbs 
                    \end{aligned}
                \end{equation*}
            \item Finding $M_{max}$ 
                \begin{equation*}
                    \begin{aligned}
                        M_{max} &= P * a \\
                                &= 47.28K * (11ft * 12in) \\  %//TODO - How did Leo get a number for P? He doesn't make it clear in the beginning
                                &= 520.08 K-ft
                    \end{aligned}
                \end{equation*} 
        \item Finding $\Delta _{required}$
            \begin{equation*}
                \begin{aligned}
                    \Delta _{required} &= L/360 \\
                                &= (33ft * 12in)/360\\
                    \Delta _{required} &= 1.10in    
                \end{aligned}
            \end{equation*} 
        \item Finding $\Delta _{max}$
            \begin{equation*}
                \begin{aligned}
                    \Delta _{max} &= \frac{Pa}{24EI} * (3L^2-4a^2)\\ %//TODO - How did Leo get a number for I? He doesn't make it clear in the beginning
                            &= \frac{47.28K * 11ft * 12in}{24 * 29000ksi * 533in^4} * (3(33ft * 12in)^2-4(11ft * 12in)^2) \\
                    \Delta _{max} &= 1.06in \\
                    \Delta _{max} &< \Delta _{required}, \text{Passes deflection checks}    
                \end{aligned}
            \end{equation*}
        \end{enumerate}
\end{enumerate}

\newpage
\subsubsection*{Sizing Roof Interior Columns (Gridline B7)}
Given:
\begin{equation*}
    \begin{aligned}
        &P_x = 89.34K \quad &P_y = 46.06K \\
        &L_{\text{B7 to A7}} = 33ft \quad &L_{\text{B7 to C7}} = 26ft \\
        &\text{Assume Lc/r = 50} \quad &L_{cx} = 15ft
    \end{aligned}
\end{equation*}
Find:
\begin{enumerate}
    \item $A_{required}$
    \item $P_u$
    \item $P_n$
    \item Select W-Section using AISC A360-16 Table 4-1a
\end{enumerate}
Solve:
\begin{enumerate}
    \item Finding Axial Loading on Column:
        \begin{equation*}
            \begin{aligned}
                P_u &= 2(P_x) + 2(P_y) \\
                    &= 2(89.34K) + 2(46.06K) \\
                P_u &= 270.8K  
            \end{aligned}
        \end{equation*}
    \item Selecting W-Section:\\
        Assuming an unbraced length using $L_c/r$ = 50 and selecting from A360-16 Table 4-1a:
            \begin{center}
                \fbox{Select W14X48, $A_{g,provided} = 14.1in^2$}
            \end{center}    
\end{enumerate}
Buckling Failure Checks:
\begin{enumerate}
    \item Finding Elastic Buckling Stress, $F_e$:
        \begin{equation*}
            \begin{aligned}
                F_e &= \frac{\pi^2E}{(L_c/r)^2} \\
                F_e &= \frac{\pi^2*29000}{(50)^2} \\
                F_e &= 114.5K
            \end{aligned}
        \end{equation*}
    \item Finding slenderness ratio limit, $4.71\sqrt{E/F_y}$:
        \begin{equation*}
            \begin{aligned}
                4.71\sqrt{E/F_y} = 4.71\sqrt{29000Ksi/50ksi} = 113.43 
            \end{aligned}
        \end{equation*}
    \item Finding Critical Buckling Load, $F_{cr}$: \\
        Since $50 < 113.43$, we will have failure in buckling 
        \begin{equation*}
            \begin{aligned}
                F_{cr} &= F_y(0.658^{Fy/Fe}) \\
                        &= 50ksi * (0.658^{50ksi/114.5ksi}) \\
                        &= 41.65
            \end{aligned}
        \end{equation*}
    \item Checking if $A_{g,required}$ exceeds $A_{g,provided}$:
        \begin{equation*}
            \begin{aligned}
                P_u &\leq 0.9F_{cr}A_g \\
                266.4 &\leq 0.9(41.6ksi)(A_{g,required}) \\
                A_{g,required} &\geq 7.12 \\
                A_{g,provided} &\geq A_{g,required} \\
                14 &\geq 7.12 
            \end{aligned}
        \end{equation*}
        Selected section passes buckling and slenderness checks
\end{enumerate}

\newpage

\section*{Composite Beam Design Example}
Abbreviation Legend:
\begin{tabular}{ c c }
    SW = Self Weight & TW = Tributary Width 
\end{tabular}
Steps:
\begin{enumerate}
    \item Setting Construction Loads:
        \begin{equation*}
            \begin{aligned}
                w_D = (\text{SW}) (\text{TW})
            \end{aligned}
        \end{equation*}
\end{enumerate}



\end{document}

