\documentclass{report} % //TODO find why latex gives errors for report and using textsc!
\usepackage{fancyhdr}
\usepackage{graphicx}
\usepackage{amsmath}
\usepackage[margin=1in]{geometry}

\title{UW IND E 250 Notes}
\author{Anthony Le}
 
\newtheorem{exmp}{Example}
\newtheorem{exrc}{Excersize}
\newtheorem{proof}{Proof}
\newtheorem{defn}{Definition}


\begin{document}

\pagestyle{fancy}
\fancyhead{}
\fancyhead[R]{Scratch Work}
\fancyhead[L]{Anthony Le}

\begin{center}
    \LARGE{\textbf{Scratch Work}}
\end{center}

Givens:
\begin{equation*}
    \begin{aligned}
        LL = 100 psf \quad DL = ?? psf\\
        L = 32 ft \quad TW = 11 ft \\
    \end{aligned}
\end{equation*}
Computing Demand:
\begin{enumerate}
    \item Finding equilvalent uniform distributed load:
    \begin{equation*}
        \begin{aligned}
            w = LL * TW = 100psf * 11ft = 1100lb/ft \\
        \end{aligned}
    \end{equation*}
    \item Finding Reaction Forces: \\
    Assuming uniform distributed load,
    \begin{equation*}
        \begin{aligned}
            A_y = B_y &= \frac{wL}{2} = \frac{1100*32}{2} \\
                      &= 17.6K 
        \end{aligned}
    \end{equation*}
    \item Finding Moment Demand:
    \begin{equation*}
        \begin{aligned}
            M_u &= \frac{wL^2}{8} = \frac{1100*32^2}{8} \\
                &= 140.8K-ft
        \end{aligned}
    \end{equation*} 
    \item Finding Section:
    \begin{equation*}
        \begin{aligned}
            \Phi M_n \geq M_u \\
            \Phi M_n \geq 140.8K-ft \\
        \end{aligned}
    \end{equation*}
    Using AISC 7-16 Table 3-10 for 140.8 K-ft at unbraced length of 32 feet:
    \item 

\end{enumerate}

\newpage
\section*{Horizontial Distribution of Lateral Forces}
\subsection*{Finding Center of Mass:}
For preliminary design, we assumed that the center of mass was in the center of the building. For detailed design, we'll want to find and compute the center of mass via weighted average of seismic weights.

\begin{center}
    Preliminary Design COM:    
\end{center}
\begin{equation*}
    \begin{aligned}
        \text{Width} = 231' &\quad \text{Height} = 102' \\
        \bar{X} = \frac{\text{Width}}{2} &\quad \bar{Y} = \frac{\text{Height}}{2} \\
        \bar{X} = \frac{231}{2} &\quad \bar{Y} = \frac{102}{2} \\
        \bar{X} = 115.5' &\quad \bar{Y} = 51'
    \end{aligned}
\end{equation*}
From this, we get that \fbox{$\bar{X} = 115.5'\quad \bar{Y} = 51'$}

\subsection*{Finding Center of Rigidity/Rotation}
\subsubsection*{Finding Frame Stiffnesses in X and Y direction}
For preliminary design, we can approximate the stiffness of BRBF's with using $K \approx  1/L$, where L is the length of the bay the BRBF spans.
\begin{equation*}
    \begin{aligned}
        K_{xi} &= 1/L = 1/33 \\
        K_{xi} &= 0.030 \\
        K_{yi} &= 1/L = 1/26 \\ 
        K_{yi} &= 0.038 
    \end{aligned}
\end{equation*}
From this, we get that \fbox{$ K_{xi} = 0.030 \quad K_{yi} = 0.038$} \\
\subsubsection*{Computing Center of Rigidity for $X_R$}
The process for finding the center of rigidity will be the same for our preliminary and detailed design:
\begin{equation*}
    \begin{aligned}
        X_R &= \frac{\sum{K_{yi}*X_i}}{\sum{K_{yi}}} \\
        X_R &= \frac{0.038*(66+165+198)}{3*0.038} \\
        X_R &= 143'
    \end{aligned}
\end{equation*}

\subsubsection*{Computing Center of Rigidity for $Y_R$}
The process for finding the center of rigidity will be the same for our preliminary and detailed design:
\begin{equation*}
    \begin{aligned}
        Y_R &= \frac{\sum{K_{yi}*X_i}}{\sum{K_{yi}}} \\
        Y_R &= \frac{0.030*(102+12+12+102)}{4*0.030} \\
        Y_R &= 57'
    \end{aligned}
\end{equation*}
From this, we get that \fbox{$X_R = 143 \quad Y_R = 57$} \\

\subsubsection*{Computing Eccentricity for Inherent Torsion}
To find the eccentricity, we'll need to find the distance between the center of mass and the center of rigidity. However, we should note at this point, that we are going to assume that counter-clockwise twisting will be in the positive direction (following the convention we've been working with such as in the right-hand rule).
\begin{equation*}
    \begin{aligned}
        e_{Y,Inherent} &= Y_R - \bar{Y} \\
        e_{X,Inherent} &= \bar{X} - X_R 
    \end{aligned}
\end{equation*}
Solving for eccentricities:
\begin{equation*}
    \begin{aligned}
        e_{Y,Inherent} &= 57 - 51 \\
        e_{Y,Inherent} &= 6' \\
        e_{X,Inherent} &= 115.5 - 143 \\
        e_{X,Inherent} &= -27.5'
    \end{aligned}
\end{equation*}
From this, we get that \fbox{$e_{X,Inherent} = -27.5' \quad e_{Y,Inherent} = 6'$} \\

\subsection*{Computing Eccentricity for Accential Torsion}
To find the eccentricity, we'll need to multiply the length and width by 5\% in order to get the accidential torsion eccentricity. From this, we'll get equations in the form of:
\begin{equation*}
    \begin{aligned}
        e_{Y,Accidential} &= 0.05 * L_Y \\
        e_{X,Accidential} &= 0.05 * L_X 
    \end{aligned}
\end{equation*}
Solving for eccentricities:
\begin{equation*}
    \begin{aligned}
        e_{Y,Accidential} &= 0.05 * 102' \\
        e_{Y,Accidential} &= 5.1' \\
        e_{X,Accidential} &= 0.05 * 231' \\
        e_{X,Accidential} &= 11.550'
    \end{aligned}
\end{equation*}
From this, we get that \fbox{$e_{X,Accidential} = 11.550' \quad e_{Y,Accidential} = 5.1'$} \\

\subsection*{Computing Total Torsion}
To find the total torsion caused by accidential and inherent torsion, we'll combine the inherent and accidential eccentricities. From this, we'll get equations in the form of:
\begin{equation*}
    \begin{aligned}
        T_X &= V_x (e_{Y,Inherent} \pm e_{Y,Accidential}) \\
        T_Y &= V_y (e_{X,Inherent} \pm e_{X,Accidential}) 
    \end{aligned}
\end{equation*}
Plugging in eccentricities:
\begin{equation*}
    \begin{aligned}
        T_X &= V_x (6' \pm 5.1') \\
        T_X &= V_x (11.10) \text{ For +X} \quad V_x (0.90) \text{ For -X} \\
        T_Y &= V_y (-27.5' \pm 11.550') \\
        T_Y &= V_y (-15.95) \text{ For +Y} \quad V_y (-39.05) \text{ For -Y}
    \end{aligned}
\end{equation*}
    

\end{document}

