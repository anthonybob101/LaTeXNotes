
\documentclass{article}
\usepackage[landscape]{geometry}
\usepackage{url}
\usepackage{multicol}
\usepackage{array}
\usepackage{amsmath}
\usepackage{esint}
\usepackage{amsfonts}
\usepackage{tikz}
\usetikzlibrary{decorations.pathmorphing}
\usepackage{amsmath,amssymb}
\usepackage{bm}

\usepackage{colortbl}
\usepackage{xcolor}
\usepackage{mathtools}
\usepackage{amsmath,amssymb}
\usepackage{enumitem}
\makeatletter

\newcommand*\bigcdot{\mathpalette\bigcdot@{.5}}
\newcommand*\bigcdot@[2]{\mathbin{\vcenter{\hbox{\scalebox{#2}{$\m@th#1\bullet$}}}}}
\makeatother

\title{130 Cheat Sheet}
\usepackage[brazilian]{babel}
\usepackage[utf8]{inputenc}

\advance\topmargin-.8in
\advance\textheight3in
\advance\textwidth3in
\advance\oddsidemargin-1.5in
\advance\evensidemargin-1.5in
\parindent0pt
\parskip2pt
\newcommand{\hr}{\centerline{\rule{3.5in}{1pt}}}
%\colorbox[HTML]{e4e4e4}{\makebox[\textwidth-2\fboxsep][l]{texto}
\begin{document}

\begin{center}{\huge{\textbf{IND E 315 Cheat Sheet}}}\\
\end{center}
\begin{multicols*}{3}

\tikzstyle{mybox} = [draw=black, fill=white, very thick,
    rectangle, rounded corners, inner sep=10pt, inner ysep=10pt]
\tikzstyle{fancytitle} =[fill=black, text=white, font=\bfseries]

%------------ Events Content ---------------
\begin{tikzpicture}
    \node [mybox] (box){%
        \begin{minipage}{0.3\textwidth}
            \small{
                \begin{tabular}{lp{4.5cm} l}
                    \textit{Union} & $E_1 \cup E_2$ \\ 
                                            & Events in \emph{either} event \\ 
                    \hline
                    \textit{Intersection} & $E_1 \cap E_2$ \\
                                            & Events in \emph{both} events \\
                    \hline
                    \textit{Complement} & $(E_1 \cup E_2)'$ \\
                                            & All events NOT in \emph{either} event \\
                    \hline
                    \textit{Mutually exclusive} & $A \cap B = 0$ \\ 
                                            & Events share no common outcomes \\
                    \hline
                    \textit{Independent} & $P(A|B) = P(A)$ \\ 
                                            & $P(B|A) = P(B)$ \\
                                            & $P(A \cap B) = P(A)P(B)$ \\
                                            & The occurrence of one event has no impact on the other
                \end{tabular}
            }
        \end{minipage}
    };
%------------ Events Header ---------------------
\node[fancytitle, right=10pt] at (box.north west) {Events - Terminology};
\end{tikzpicture}

%------------ Counting Techniques Content ---------------
\begin{tikzpicture}
    \node [mybox] (box){%
        \begin{minipage}{0.3\textwidth}
            \small{
                \begin{tabular}{lp{4.5cm} l}
                    \textit{Multiplication Rule} & $n_1 * n_2 * \dots * n_k$ \\ 
                                            & Product gives total number of ways of completing steps 1 through k \\ 
                    \hline
                    \textit{Permutation Rule} & $_nP_r = \frac{n!}{n_1! n_2! \dots n_r!}$ \\
                                            & Give number of permutations from ordered selection of $r$ objects from $n$ objects   \\
                    \hline
                    \textit{Combination Rule} & $_nC_r = \left( \begin{tabular}{c}
                        n \\
                        r
                        \end{tabular}  \right) = \frac{n!}{r!(n-r)!}$ \\ 
                                            & Give number of combinations from selection of (size) $r$ items from $n$ objects \\
                \end{tabular}
            }
        \end{minipage}
    };
%------------ Counting Techniques Header ---------------------
\node[fancytitle, right=10pt] at (box.north west) {Counting Techniques};
\end{tikzpicture}

%------------ Random Variables Content ---------------
\begin{tikzpicture}
    \node [mybox] (box){%
        \begin{minipage}{0.3\textwidth}
            \small{
                \begin{tabular}{lp{3.5cm} l}
                    \textit{Random Variables (rv)} & Function ($X$) that can take on values ($x$) for each outcome in a sample space \\
                    \hline
                    \textit{Discrete rv's} & A random variable with a \emph{countable} range. \\
                                            & (counts, integers, or natural numbers) \\ 
                    \hline
                    \textit{Continuous rv's} & A random variable with a \emph{uncountable} range. \\
                                            & (Such as length, weight, volume)
                \end{tabular}
            }
        \end{minipage}
    };
%------------ Random Variables Header ---------------------
\node[fancytitle, right=10pt] at (box.north west) {Random Variables};
\end{tikzpicture}


%------------ Probability Content ---------------
\begin{tikzpicture}
    \node [mybox] (box){%
        \begin{minipage}{0.3\textwidth}
            \small{
                \begin{tabular}{lp{5cm} l}
                    \textit{Addition Rule} & \footnotesize{$P(A \cup B) = P(A) + P(B) - P(A \cap B)$} \\ 
                                            & \footnotesize{$P(A \cap B) = P(A) + P(B) - P(A \cup B)$} \\
                                            & If $P(A \cap B) =  0$, then \\
                                            & $P(A \cup B) = P(A) + P(B)$ \\
                    \hline
                    \textit{Conditional Prob} & $P(B|A) = P(A \cap B) / P(A)$ \\
                                            & For $P(A) > 0$ \\ 
                                            & The conditional probability of event $B$ given event $A$ \\ 
                    \hline
                    \textit{Multpl. Rule} & $P(A \cap B) = P(B|A) * P(A)$ \\
                                            & $P(A \cap B) = P(A|B) * P(B)$ \\
                                            & Probability of intersection \\
                    \hline
                    \textit{Total Probability} & $P(B) = P(B \cap A) + P(B \cap A')$ \\ 
                                        & $P(B) = P(B|A) * P(A) + P(B|A') * P(A')$ \\
                    \hline
                    \textit{Complement} & $P(A^C) = 1- P(A)$ \\ 
                                        & DeMorgan's Law: \\
                                        & $(A \cap B)' = A' \cup B' $ \\
                                        & $(A \cup B)' = A' \cap B' $ \\ 
                                        & Gives complements of a union or intersection \\
                \end{tabular}
            }
        \end{minipage}
    };
%------------ Probability Header ---------------------
\node[fancytitle, right=10pt] at (box.north west) {Probability};
\end{tikzpicture}

\newpage


%------------ Discrete Prob. Distribution Content ---------------
\begin{tikzpicture}
    \node [mybox] (box){%
        \begin{minipage}{0.65\textwidth}
            \small{
                %\fbox{
                    \begin{tabular}{ | m{2.5cm} | m{4.5cm}| m{3.5cm} | m{5cm}} 
                        $\bm{X}$ & $\bm{f(x)}$ & $\bm{\mu = E(X)}$ & $\bm{\sigma ^2 = V(X)}$ \\ 
                        \hline
                        General          & $f(x_i) = P(X = x_i)$             & $\mu = \sum_x x * f(x)$              & $\sigma ^2 = V(X) = E(X - \mu)^2$\\
                        Uniform          & $f(x) = 1/(b - a + 1)$                  & $\mu = E(X) = (b + a)/2 $     & $\sigma ^2 = V(x) = ((b - a + 1)^2 - 1)/12 $\\     
                        Binomial         & $f(x) = _nC_x p^x (1 - p)^{n-x}$        & $\mu = E(X) = np$             & $\sigma ^2 = V(X) = np(1-p)$\\
                        Geometric        & $f(x) = p^1 (1-p)^{x-1}$                & $\mu = E(X) = 1/p$            & $\sigma ^2 = V(X) = (1-p)/(p^2)$\\
                        Neg. Binomial    & $f(x) = _{x-1}C_{r-1} p^r (1-p)^{x-r}$   & $\mu = E(X) = r/p$            & $\sigma ^2 = V(X) = (r(1-p))/(p^2)$\\
                    \end{tabular}
                %}
     
            }
        \end{minipage}
    };
%------------ Discrete Prob. Distribution Header ---------------------
\node[fancytitle, right=10pt] at (box.north west) {Discrete Prob. Distribution (General)};
\end{tikzpicture}

%------------ Probability Content ---------------
\begin{tikzpicture}
    \node [mybox] (box){%
        \begin{minipage}{0.3\textwidth}
            \small{
                \begin{tabular}{lp{4.8cm} l}
                    \textit{Probability Mass} & Describes the probability distribution of $X$ \\
                    %\fbox{\parbox{3.4cm}{text}} & \fbox{\parbox{3.4cm}{text}} \\
                    \textbf{(PMF)}              & Plug $x_i$ to $f(x_i)$ to find $P(X = x_i)$ \\
                    \textit{(General)}          & $P(X = x_i) = f(x_i)$\\
                    \textit{(Uniform)}          & $f(x) = 1/(b - a + 1)$\\
                    \textit{(Binomial)}\footnote{p or p(x) represents the individual probability of x outcome}        & $f(x) = _nC_x p^x (1 - p)^{n-x}$\\
                    \textit{(Geometric)}       & $f(x) = p^1 (1-p)^{x-1}$\\
                    \textit{(Neg. Binomial)}   & $f(x) = _{x-1}C_{r-1} p^r (1-p)^{x-r}$\\
                    \hline

                    \textit{Cumulative Dist.} & Describes the probability that $X$ will be less than or equal to $x$ \\
                    \textit{(CDF)}              & $F(x) = P(X \leq x)$ \\    
                    \textbf{(General)}          & $P(a < X \leq b) = F(b) - F(a)$\\
                    %\textit{(Uniform)}         & $ $\\
                    %\textit{(Binomial)}        & $ $\\
                    %\textit{(Geometric)}       & $ $\\
                    %\textit{(Neg. Binomial)}   & $ $\\
                    \hline

                    \textit{Mean}               & Center of a probability distribution. \\
                    \textbf{(General)}          & $\mu = E(X) = \sum_x x * f(x)$ \\
                    \textit{(Uniform)}          & $\mu = E(X) = (b + a)/2 $\\
                    \textit{(Binomial)}         & $\mu = E(X) = np$\\
                    \textit{(Geometric)}        & $\mu = E(X) = 1/p$\\
                    \textit{(Neg. Binomial)}    & $\mu = E(X) = r/p$\\ %not in notes, using stats formula sheet
                    \hline

                    \textit{Variance}           & Varability in a probability distribution.\\ 
                    \textbf{(General)}          & $\sigma ^2 = \sum_x x^2 * f(x) - \mu ^2$, or \\
                                                & $\sigma ^2 = E[X^2] - (E[X])^2$, or \\
                                                & $\sigma ^2 = V(X) = E(X - \mu)^2$ \\ 
                    \textit{(Uniform)}          & $\sigma ^2 = V(x) = ((b - a + 1)^2 - 1)/12 $\\
                    \textit{(Binomial)}         & $\sigma ^2 = V(X) = np(1-p)$\\
                    \textit{(Geometric)}        & $\sigma ^2 = V(X) = (1-p)/(p^2)$\\
                    \textit{(Neg. Binomial)}    & $\sigma ^2 = V(X) = (r(1-p))/(p^2)$\\

                \end{tabular}
            }
        \end{minipage}
    };
%------------ Probability Header ---------------------
\node[fancytitle, right=10pt] at (box.north west) {Probability};
\end{tikzpicture}


\newpage

%------------ Probability Distribution Content ---------------
\begin{tikzpicture}
    \node [mybox] (box){%
        \begin{minipage}{0.3\textwidth}
            \small{
                \begin{tabular}{lp{4.8cm} l}
                    \textit{Probability Density} & $P(a \leq X \leq b) = \int _a ^b f(x) dx$\\
                    %\fbox{\parbox{3.4cm}{text}} & \fbox{\parbox{3.4cm}{text}} \\
                    \textit{(PDF)}          & Describes the probability distribution of $X$ \\                        
                    \textbf{(Continuous)}   & Integrate $f(x)$ between $a$ and $b$ to find $P(a \leq X \leq b)$ \\
                                            & \textbf{Or derivative of CDF}  \\
                    \hline
                    \textit{Cumulative Dist.} & $F(x) = P(X \leq x)$ \\
                    \textit{(CDF)}          & $F(x) = \int ^x _{-\infty} f(u) du$ \\
                    \textbf{(Continuous)}   & \textbf{Or integral of PDF} \\
                    \hline
                    \textit{Multpl. Rule} & $P(A \cap B) = P(B|A) * P(A)$ \\
                                            & $P(A \cap B) = P(A|B) * P(B)$ \\
                                            & Probability of intersection \\
                    \hline
                    \textit{Total Probability} & $P(B) = P(B \cap A) + P(B \cap A')$ \\ 
                                        & $P(B) = P(B|A) * P(A) + P(B|A') * P(A')$ \\
                \end{tabular}
            }
        \end{minipage}
    };
%------------ Probability Header ---------------------
\node[fancytitle, right=10pt] at (box.north west) {Continuous Probability Distribution Functions};
\end{tikzpicture}


\newpage


%------------ Heating ---------------
\begin{tikzpicture}
\node [mybox] (box){%
    \begin{minipage}{0.3\textwidth}
    $\frac{dT}{dt} = -k(T-T_o)$ \\
    $ T_o =$ outside temperature
    \end{minipage}
};
%------------ Heating Header ---------------------
\node[fancytitle, right=10pt] at (box.north west) {Chapter 2};
\end{tikzpicture}

%------------ Mixing ---------------
\begin{tikzpicture}
\node [mybox] (box){%
    \begin{minipage}{0.3\textwidth}
    $\frac{dA}{dt} = c_1r_1-\frac{A}{V}r_2$\\
    $V=V_0 +(r_1 - r_2)t$ \\
    $c_1$, solution mixture in \\
    $r_1$, in rate \\
    $r_2$, out rate
    \end{minipage}
};
%------------ Mixing Header ---------------------
\node[fancytitle, right=10pt] at (box.north west) {Mixing Problem};
\end{tikzpicture}

%------------ Inner Product Spaces ---------------
\begin{tikzpicture}
\node [mybox] (box){%
    \begin{minipage}{0.3\textwidth}
    1. $ \langle v,v \rangle \geq 0$ Furthermore, $ \langle v,v \rangle = 0 \leftrightarrow v = 0 $ \\
	2. $\langle v,u \rangle = \langle u,v \rangle $ \\
	3. $ \langle ku,v \rangle =k\langle u,v \rangle $ \\
	4. $\langle u+v,w\rangle = \langle u,w \rangle + \langle v,w\rangle $ \\
	$ ||v|| = \langle v, v \rangle $ \\
    $ \cos^{-1}(\frac{\langle v,u \rangle}{||v|| ||u||}) $
    \end{minipage}
};
%------------ Inner Product Space Header ---------------------
\node[fancytitle, right=10pt] at (box.north west) {Inner Product Spaces};
\end{tikzpicture}

%------------ Gram-Schmidt Content ---------------
\begin{tikzpicture}
\node [mybox] (box){%
    \begin{minipage}{0.3\textwidth}
    \begin{align*}
  v_{1} &= x_{1} \\
  v_{2} &= x_{2} - \frac{\langle x_2, v_1\rangle}{||v_{1}||^2}v_{1}\\
  &\shortvdotswithin{=} 
  v_{n} &= x_{m} - \sum_{k=1}^{m-1} \frac{\langle x_{m},v_{k} \rangle}{||v_{k}||^{2}}v_{k} 
    \end{align*}
    \end{minipage}
};
%------------ Gram-Schmidt Header ---------------------
\node[fancytitle, right=10pt] at (box.north west) {Gram-Schmidt};
\end{tikzpicture}
%------------ Variation of Parameters Content ---------------------
\begin{tikzpicture}
\node [mybox] (box){%
    \begin{minipage}{0.3\textwidth}
    	\begin{align*}
        	F(x) &= y'' + y' \\
            y_h &= b_1y_1(x) + b_2y_2(x), y_1 y_2 \text{ are L.I.} \\
            y_p &= u_1(x)y_1(x) + u_2(x)y_2(x) \\
            u_1 &= \int^t -\frac{y_2F(t)dt}{w[y_1,y_2](t)} \\
            u_2 &= \int^t \frac{y_1F(t)dt}{w[y_1,y_2](t)} \\    		
            y &= y_h + y_p
    	\end{align*}
    \end{minipage}
};
%------------ Variation of Parameters Header ---------------------
\node[fancytitle, right=10pt] at (box.north west) {Variation of Parameters};
\end{tikzpicture}

%------------ Series Solution Content ---------------
\begin{tikzpicture}
\node [mybox] (box){%
    \begin{minipage}{0.3\textwidth}
    $y'' + p(x)y' + q(x)y = 0$ \\
    Useful when $p(x), q(x)$ not constant \\
    Guess $y = \sum_{n=0}^{\infty}a_n(x-x_0)^n$
    \small{
    	\begin{tabular}{lp{4cm} l}
        \hline
        $e^x$ & $\sum_{n=0}^{\infty} x^n/{n!}$ \\ \hline
        $\sin x$ & $\sum_{n=0}^{\infty} \frac{(-1)^n}{(2n+1)!}x^{2n+1}$ \\ \hline
        $\cos x$ & $\sum_{n=0}^{\infty} \frac{(-1)^n}{(2n)!}x^{2n}$ \\
	\end{tabular}}
    \end{minipage}
};
%------------ Series Solution Header ---------------------
\node[fancytitle, right=10pt] at (box.north west) {Series Solution};
\end{tikzpicture}

%------------ Systems of ODE Content ---------------
\begin{tikzpicture}
\node [mybox] (box){%
    \begin{minipage}{0.3\textwidth}
    \small{
    	\begin{tabular}{lp{4cm} l}
        $\vec{x}' = A\vec{x}$ \\
		\textit{A is diagonalizable} & $\vec{x}(t)=a_{1}e^{\lambda_1 t}\vec{v_1}+\cdots+ a_{n}e^{\lambda_n t}\vec{v_n}$ \\ \hline
        \textit{A is not diagonalizable} & $\vec{x}(t)=a_1e^{\lambda_1 t}\vec{v_1} + a_2e^{\lambda t}(\vec{w} + t\vec{v} )$ \\
        & where $(A - \lambda I)\vec{w} = \vec{v} $\\
        & $\vec{v}$ is an Eigenvector w/ value $\lambda$ \\
        & i.e. $\vec{w}$ is a generalized Eigenvector \\ \hline
        $\vec{x}' = A\vec{x} + \vec{B}$ &Solve $y_h$ \\
        & $\vec{x_1} = e^{\lambda_1t}\vec{v_1}, \vec{x_2} = e^{\lambda_2t}\vec{v_2}$ \\ 			& $\vec{X} = [\vec{x_1},\vec{x_2}]$ \\
        & $\vec{X}\vec{u}'=\vec{B}$ \\
        & $y_p = \vec{X}\vec{u}$ \\
        & $y = y_h + y_p$
	\end{tabular}}
    \end{minipage}
};
%------------ Systems of ODE Header ---------------------
\node[fancytitle, right=10pt] at (box.north west) {Systems};
\end{tikzpicture}

%------------ Exponentiation Content ---------------
\begin{tikzpicture}
\node [mybox] (box){%
    \begin{minipage}{0.3\textwidth}
    \small{
    	\begin{tabular}{lp{4cm} l}
        $A^n = SD^nS^{-1}$ \\
        \textit{D is the diagonalization of A}
	\end{tabular}}
    \end{minipage}
};
%------------ Spring-Mass Header ---------------------
\node[fancytitle, right=10pt] at (box.north west) {Matrix Exponentiation};
\end{tikzpicture}
\
%------------ Laplace Transforms Content ---------------
\begin{tikzpicture}
\node [mybox] (box){%
    \begin{minipage}{0.3\textwidth}
    $L[f](s) = \int_0^{\infty} e^{-sx}f(x)dx $\\
    
    \small{
    	\begin{tabular}{lp{4cm} l}
        $f(t) = t^n, n \geq 0 $ &$F(s) = \frac{n!}{s^{n+1}}, s > 0 $ \\
        $f(t) = e^{at}, a \textit{ constant}$ & $ F(s) = \frac{1}{s-a}, s > a$ \\
        $f(t) = \sin{bt}, b \textit{ constant}$ & $ F(s) = \frac{b}{s^2 + b^2}, s > 0$ \\
        $f(t) = \cos{bt}, b \textit{ constant}$ & $ F(s) = \frac{s}{s^2 + b^2}, s > 0$ \\
        $f(t) = t^{-1/2}$ & $F(s) = \frac{\pi}{s^{1/2}}, s > 0$ \\
        $f(t) = \delta(t-a)$ & $F(s) = e^{-as}$ \\
        $f'$ & $L[f'] = sL[f] - f(0)$ \\
        $f''$ & $L[f''] = s^2 L[f] - sf(0) - f'(0)$ \\
        $L[e^{at}f(t)]$ & $L[f](s-a)$ \\
        $L[u_a(t)f(t-a)]$ & $L[f]e^{-as}$ 
        \end{tabular}}
    \end{minipage}
};
%------------ Laplace Transforms Header ---------------------
\node[fancytitle, right=10pt] at (box.north west) {Laplace Transforms};
\end{tikzpicture}
%------------ Gaussian Integral Content ---------------------
\begin{tikzpicture}
\node [mybox] (box){%
    \begin{minipage}{0.3\textwidth}
	$\int_{-\infty}^{+\infty} e^{-1/2(\vec{x}^TA\vec{x})} = \frac{\sqrt{2\pi}^n}{\sqrt{\det A}}$
	\end{minipage}
};
%------------ Gaussian Integral Header ---------------------
\node[fancytitle, right=10pt] at (box.north west) {Gaussian Integral};
\end{tikzpicture}

%------------ Complex Numbers Content ---------------------
\begin{tikzpicture}
\node [mybox] (box){%
    \begin{minipage}{0.3\textwidth}
    \small{
        	\begin{tabular}{lp{4cm} l}
            \textit{Systems of equations} & If $\vec{w_1} = \vec{u(t)} + i\vec{v(t)}$ is a solution, $\vec{x_1} = \vec{u(t)}, \vec{x_2} = \vec{v(t)}$ are solutions \\ 
            & i.e. $\vec{x_h} = c_1 \vec{x_1} + c_2 \vec{x_2}$ \\
            \hline
            \textit{Euler's Identity} &$e^{ix} = \cos x + i \sin x$
			\end{tabular}
    }
	\end{minipage}
};
%------------ Gaussian Integral Header ---------------------
\node[fancytitle, right=10pt] at (box.north west) {Complex Numbers};
\end{tikzpicture}

%------------ Vector Spaces ---------------
\begin{tikzpicture}
\node [mybox] (box){%
    \begin{minipage}{0.3\textwidth}
    $v_1, v_2 \in V$\\
    1. $v_1 + v_2 \in V$ \\
	2. $k \in \mathbb{F}, kv_1 \in V $ \\
	3. $ v_1 + v_2 = v_2 + v_1 $ \\
	4. $(v_1 + v_2) + v_3 = v_1 + (v_2 + v_3) $ \\
	5. $\forall v \in V, 0 \in V \mid 0 + v_1 = v_1 + 0 = v_1$ \\
    6. $\forall v \in V, \exists -v \in V \mid v + (-v) = (-v) + v = 0 $ \\
    7. $\forall v \in V, 1 \in \mathbb{F} \mid 1*v = v$ \\
    8. $\forall v \in V, k,l \in \mathbb{F}, (kl)v = k (lv)$ \\
    9. $\forall k \in \mathbb{F}, k(v_1 + v_2) = kv_1 + kv_2$ \\
    10. $\forall v \in V, k,l \in \mathbb{F}, (k+l)v = kv + lv$
    \end{minipage}
};
%------------ Vector Space Header ---------------------
\node[fancytitle, right=10pt] at (box.north west) {Vector Spaces};
\end{tikzpicture}
\end{multicols*}
\end{document}


Contact GitHub API Training Shop Blog About
© 2016 GitHub, Inc. Terms Privacy Security Status Help