\documentclass[../IND E 315.tex]{subfiles} 
\usepackage{fancyhdr}
\usepackage{graphicx}
\usepackage{amsmath}
\usepackage{mhchem}
\usepackage{amssymb}
\usepackage[margin=1in]{geometry}

\usepackage{subfiles} % Best loaded last in the preamble
\graphicspath{{./images/}}

\title{UW IND E 315 Notes}
\author{Anthony Le}

\newtheorem{exmp}{Example}
\newtheorem{exrc}{Excersize}
\newtheorem{proof}{Statement}
\newtheorem{defn}{Definition}


\begin{document}

\pagestyle{fancy}
\fancyhead{}
\fancyhead[R]{UW IND E 315}
\fancyhead[L]{Anthony Le}

\section*{Chapter 2 - Probability}
\subsection*{Simple Example}
\begin{exmp}
    What is the probability of getting heads on the toss of a coin?
\end{exmp}
\begin{defn}
    \textbf{Probability} \\
    The likelihood of a particular outcome/event.
    \begin{enumerate}
        \item Formally: a number from the interval [0,1] to indivate the low (0) to high (1) likelihood of occurrence.
    \end{enumerate}
\end{defn}

\begin{defn}
    \textbf{Sample Space} \\
    The set of all possible outcomes of a random experiment.
    \begin{enumerate}
        \item Event: Subset of the sample space
    \end{enumerate}
\end{defn}

\subsection*{Sample Space and Events (2.1)}
    \begin{center}
        \includegraphics[width = 5cm]{Example2.1}
    \end{center}
Given a sample space $S$ where the probability of S is 1, you can find the probability of event $A$ by subtracting the probability of $A'$ from the probability of the sample space. Describing this in math terms:
\begin{equation*}
    \begin{aligned}
        \text{Sample Space where } P(S) = 1 \\
        P(A) = 1- P(A')
    \end{aligned}
\end{equation*}

\begin{exmp}
    What is the example space for the sum of 2 dice?
\end{exmp}
\begin{equation*}
    \begin{aligned}
        S = {2,3,4,5,6,7,8,9,10,11,12}
    \end{aligned}
\end{equation*}
\begin{enumerate}
    \item Let A be the event that the dice rolls an even number:
        \begin{equation*}
            \begin{aligned}
                A = {2,4,6,8,10,12}
            \end{aligned}
        \end{equation*}
    \item Let B be the event that the dice rolls a prime number:
        \begin{equation*}
            \begin{aligned}
                B = {2,3,5,7,11}
            \end{aligned}
        \end{equation*}
\end{enumerate}
From this, find these following:
\begin{enumerate}
    \item $P(A) = ?$
    \item $P(B) = ?$ 
    \item $P(A|B) = ?$
\end{enumerate}

Here's a graphic showing the sample space for the above situation:
    \begin{center}
        \includegraphics[width = 5cm]{Ch2Example2}
    \end{center}
To find $P(A|B)$:
\begin{equation*}
    \begin{aligned}
        P(A|B) = \frac{P(A \cap B)}{P(B)}
    \end{aligned}
\end{equation*}

\subsubsection*{Interpretations and Axioms of Probability (2.3)}
\begin{enumerate}
    \item Let S be the sample space of some random experiment.
    \item Let E be some event within that sample space.
    \item $P(E)$, the probability of the event is \emph{assigned} based on our knowledge of the system under study.
    \item \emph{Mathematically}, $P(E)$ should satisify the three axioms of probability.
        \begin{enumerate}
            \item The axioms ensure that:
                \begin{enumerate}
                    \item The assigned probabilities can be \emph{interpreted as relateive frequencies}
                    \item The assignments are \emph{consistent with our intuitive understanding} of relationships between relative frequencies.
                \end{enumerate}
        \end{enumerate}
\end{enumerate}

\subsubsection*{Axioms of Probability}
\begin{enumerate}
    \item $0 <= P(E) <= 1$ for any event E.
    \item $P(S) = 1$ where $S$ is the sample space.
    \item If $E_1$, $E_2$,... are mutually exclusive events (ex $E_1 \cap E_2 = 0$), then $P(E_1 \cup E_2) = P(E_1) + P(E_2)$.
\end{enumerate}

\subsubsection*{Event: subset of the sample space - combination of events}
%tags - tag vs cup, union, intersection
\begin{enumerate}
    \item Union of two events
        \begin{enumerate}
            \item All events that are contained in \emph{either} of the two events
            \item Own words - Look at events are in event 1 \emph{or} event 2
            \item Denoted at $E_1 \cup E_2$
        \end{enumerate}
        \begin{center}
            \includegraphics[width = 5cm]{Ch2Event_Union}
        \end{center}
    \item Intersection of two events
        \begin{enumerate}
            \item All events that are contained in \emph{both} of the two events
            \item Own words - Look at events that are in event 1 \emph{and} event 2
            \item Denoted as $E_1 \cap E_2$
        \end{enumerate}
        \begin{center}
            \includegraphics[width = 5cm]{Ch2Event_Intersection}
        \end{center}
    \item Complement of an event
        \begin{enumerate}
            \item All events that are not in the event.
            \item Own words - Look at all events that are \emph{not} in events 1 and 2, but still in the sample space
            \item Denoted as $(E_1 \cup E_2)'$ \footnote{Also note that $((E_1 \cup E_2)')' = E_1 \cup E_2$}
        \end{enumerate}
        \begin{center}
            \includegraphics[width = 5cm]{Ch2Event_Complement}
        \end{center}
    \item Mutually Exclusive Events
        \begin{enumerate}
            \item Events share no common outcomes, AKA the occurrence of one event precludes the occurrence of the other.
            \item Denoted as $A \cap B = 0$ 
            \item Example - not being able to roll a head and a tail on a coin at the same time.
        \end{enumerate}
        \begin{center}
            \includegraphics[width = 5cm]{Ch2Event_MutuallyExclusive}
        \end{center}
\end{enumerate}

\subsection*{Counting Techniques (2.2)}
\begin{enumerate}
    \item Now we have built the conceptual framework of probability.
    \item To apply this framework to solve real-world problems, we still need methods. 
    \item We'll first learn three counting techniques:
        \begin{enumerate}
            \item Multiplication rule
            \item Permutation rule
            \item Combination rule
        \end{enumerate}
\end{enumerate}

\subsubsection*{Multiplication Rule}
\begin{defn}
    \textbf{Multiplication Rule}
    \begin{enumerate}
        \item Let an operation consist of $k$ steps and \dots 
            \begin{enumerate}
                \item $n_1$ ways of completing step 1, 
                \item $n_2$ ways of completing step 2, (repeat until you reach step k)
                \item $n_k$ ways of completing step $k$.
            \end{enumerate}
        \item From this, the total number of ways or outcomes are:
            \begin{enumerate}
                \item $n_1 * n_2 * \dots * n_k$
            \end{enumerate}
    \end{enumerate}
\end{defn}


\begin{exmp}
    A message is recieved either on time or late. If we were to receive three messages, what are all the possible results for the three messages?
    \begin{center}
        \includegraphics[width = 5 cm]{Ch2Example3}
    \end{center}
\end{exmp}
\begin{equation*}
    \begin{aligned}
        2 * 2 * 2 = 8 \quad \text{possible results}
    \end{aligned}
\end{equation*}
Another example:
\begin{exmp}
    With 4 colors, 3 fonts, and 3 image positions available for a website, how many different designs are possible? 
\end{exmp}
\begin{equation*}
    \begin{aligned}
        4 * 3 * 3 = 36 \quad \text{possible designs}
    \end{aligned}
\end{equation*}

\subsubsection*{Permutation Rule}
\begin{enumerate}
    \item If the sample space contains 3 items, $S = {a, b, c}$ 
    \begin{enumerate}
        \item Then there are 6 permutations (or 3!)
        \item abc, acb, bac, bca, cab, cba \emph{(order matters)}
        \item This represents the number of ways 3 people can be rearranged
    \end{enumerate} 
\end{enumerate}
\begin{defn}
    \textbf{Permutation Rule}
    \begin{enumerate}
        \item A permutation is a unique sequence of distinct items. 
        \item The number of permutations for a set of $n$ items is $n!$
            \begin{enumerate}
                \item Not sequencing any items, we get that:
                    \begin{equation*}
                        \begin{aligned}
                            P^n = n!
                        \end{aligned}
                    \end{equation*}
                \item To sequence only $r$ items from a set of $n$ \emph{different} items:
                    \begin{equation*}
                        \begin{aligned}
                            P^n_r = \frac{n!}{(n-r)!}
                        \end{aligned}
                    \end{equation*}
                \item To sequence only $r$ items from a set of $n$ \emph{identical} items:\footnote{Where $n = n_1 + n_2 + \dots n_r$  identical items}
                    \begin{equation*}
                        \begin{aligned}
                            P^n_r = \frac{n!}{n_1! n_2! \dots n_r!}
                        \end{aligned}
                    \end{equation*}

            \end{enumerate}
    \end{enumerate}    
\end{defn}

For example:
\begin{exmp}
    Sequence only 3 items from a set of 7 items.
\end{exmp}
\begin{equation*}
    \begin{aligned}
        P^7_3 &= \frac{7!}{(7-3)!} \\
                &= \frac{7!}{4!} \\
                &= \frac{7X6X5X4!}{4!} \\
                &= 210
    \end{aligned}
\end{equation*}
Another example:
\begin{exmp}
    Say we have 4 locations, and we want to place 2 different components on those locations. How many different sequences will we have?
\end{exmp}
In other words, sequence 2 locations from a set of 4 locations.
\begin{equation*}
    \begin{aligned}
        P^4_2 = &= \frac{4!}{(4-2)!} \\
                &= \frac{4*3*2*1}{2*1} \\
                &= 12
    \end{aligned}
\end{equation*}

%To sequence permutations of similar objects
%\begin{enumerate}
%    \item Used for counting the sequences when not all the items are different.
%    \item Where the number of permutations of n where:
%        \begin{enumerate}
%            \item $n_1$ are identical,
%            \item $n_2$ are identical, ..., and
%            \item $n_r$ are identical
%        \end{enumerate}
%    \item Is calculated as:
%        \begin{equation*}
%            \begin{aligned}
%                \frac{n!}{n_1! n_2! \dots n_r!}
%            \end{aligned}
%        \end{equation*}
%\end{enumerate}

\begin{exmp}
    A hospital operating room needs to schedule three knee surgeries and
    two hip surgeries in a day. The number of possible sequences of three knee and two hip surgeries is:
\end{exmp}
\begin{equation*}
    \begin{aligned}
        n &= 2 + 3 = 5 \\
        P^5 &= \frac{n!}{n_1! n_2! \dots n_r!} \\
            &= \frac{5!}{2!3!} \\
            &= 10
    \end{aligned}
\end{equation*}

\subsubsection*{Combination Rule}
\begin{enumerate}
    \item A combination is a selection of $r$ items from a set of $n$ where \emph{their order doesn't matter.}\footnote{Since order doesn't matter you could also look at combinations as whether a solution HAS the items you're looking for. Think for S = {a,b,c}, we found multiple permutations for r = 3. However, as we'll see, we only get 1 \emph{combination} for r = 3}
    \item If $S = {a,b,c}$,
        \begin{enumerate}
            \item If $r = 3$, there is only 1 combination - $abc$\footnote{Think about this as there only being one solution where the solution contains 3 items!}
            \item If $r = 2$, there are 3 combinations - $ab$, $ac$, $bc$
        \end{enumerate}
        \begin{equation*}
            \begin{aligned}
                C^n_r = \left(n \atop r\right) = \frac{n!}{r!(n-r)!} 
            \end{aligned}
        \end{equation*}
\end{enumerate}




\end{document}