\documentclass[../IND E 315.tex]{subfiles} 
\usepackage{fancyhdr}
\usepackage{graphicx}
\usepackage{amsmath}
\usepackage{mhchem}
\usepackage{amssymb}
\usepackage[margin=1in]{geometry}

\usepackage{subfiles} % Best loaded last in the preamble

\title{UW IND E 315 Notes}
\author{Anthony Le}

\newtheorem{exmp}{Example}
\newtheorem{exrc}{Excersize}
\newtheorem{proof}{Statement}
\newtheorem{defn}{Definition}


\begin{document}

\pagestyle{fancy}
\fancyhead{}
\fancyhead[R]{UW IND E 315}
\fancyhead[L]{Anthony Le}

\section*{Lecture 0 - Preliminary Information and Prep}

\subsection*{Class Structure}
\begin{enumerate}
    \item Class will be remote synchronous on zoom, Tuesdays and Thursdays from 9:40AM to 11:20 AM
    \item There will be additional videos to make up for lost class time
    \item This class will be very fast paced! We're going to be fitting 12+ weeks of material into 8 weeks of lecture
\end{enumerate}

\subsection*{Homework}
\begin{enumerate}
    \item Worth 25\%
    \item Problems submitted late will be worth 80\%
    \item HW not accepted more than 24 hours after it's due
    \item No need to submit work
    \item Will have multiple attempts on most questions
\end{enumerate}

\subsection*{Quizzes}
\begin{enumerate}
    \item Worth 15\%
    \item Weekly quizzes to prep for exam
    \item 2-4 questions on canvas
    \item graded on completion
    \item 1-2 questions will be randomly graded for effort
\end{enumerate}

\subsection*{Critical Concepts and Methods in IND E 315}
\begin{enumerate}
    \item Probability (Chapter 2)
    \item Discrete and Continuous Data (Chapter 3 and 4)
    \item Mean, Standard Deviation, and Variance (Chapter 3 and 4)
    \item Graphical Description of Data (Chapter 6)
    \item Central Limit Theorem (Chapter 7)
    \item Confidence Intervals (Chapter 8)
    \item Hypothesis Testing (Chapter 9 and 10)
    \item Linear Regression (Chapter 11)
\end{enumerate}

\subsection*{How Concepts can be applied}
\begin{enumerate}
    \item Discrete and continuous data: running simulation models to model a phenomena (factory output, disaster relief efforts, etc.)
    \item Graphical Description of Data: Data visualization
    \item Central Limit Theorem: Important principle on which most modern hypothesis testing is based
    \item Confidence Intervals: Introducing the idea of error into a prediction
    \item Hypothesis Testing: Testing if what we believe if backed by statistics
    \item Linear Regression: Simplest prediction model
\end{enumerate}

\subsection*{Why we learn statistics}
Many of the major decisions affecting the lives of everything on this planet have some \textbf{statistical justification or basis}, and the methods we teach are relevant to understanding these decisions
\subsubsection*{Practical Relevance}
\begin{enumerate}
    \item learn how to read "numbers"
    \item learn how to collect "numbers" (a rigorous term is "samples")
    \item learn how to draw conclusions analytically
    \item Learn how to conduct "decision-makings" rigorously, diplomatically, officially, publically, fairly, optimally, and so on and so forth 
\end{enumerate}


\end{document}