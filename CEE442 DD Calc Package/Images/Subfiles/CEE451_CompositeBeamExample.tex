\documentclass{report} % //TODO find why latex gives errors for report and using textsc!
\usepackage{fancyhdr}
\usepackage{graphicx}
\usepackage{amsmath}
\usepackage{amssymb}
\usepackage[margin=1in]{geometry}

\title{CEE 451 - Composite Beam Design Example}
\author{Anthony Le}
 
\newtheorem{exmp}{Example}
\newtheorem{exrc}{Excersize}
\newtheorem{proof}{Proof}
\newtheorem{defn}{Definition}


\begin{document}

\pagestyle{fancy}
\fancyhead{}
\fancyhead[R]{CEE 451 - Composite Beam Design Example}



\section*{CEE 451 Wiebe - Composite Beam Design Example}
Abbreviation Legend:
\begin{center}
    \begin{tabular}{ c c c }
        SW = Self Weight & TW = Tributary Width & SI = Superimposed\\
        CL = Construction Load & Conc = Concrete \\
    \end{tabular}
\end{center}
Givens:

\begin{center}
    \begin{tabular}{ c|c|c|c|c }
        Geometry    & Composite Deck & Load Combinations & Service Loads & Construction Loads \\
        %Geometry & Comp.Deck & Load Combos & Service Loads & Const. Loads
        \hline
        TW = 9ft    & Vulcraft & 1.4D       &  D = 30psf + SW  & D = SW  \\
                   & 3VLI 2.5" Slab & 1.2D + 1.6L & L = 80psf & L = 20psf \\
                   & 3" Deck  &            &               &             \\ 
                   & SW = 51 psf&            &               &             \\
                   & f'c = 4ksi &            &               &   
    \end{tabular}
\end{center}

Steps:
\subsection*{Selecting from construction loads}
\begin{enumerate}
    \item Finding Distributed Loads:
        \begin{equation*}
            \begin{aligned}
                w^{(1)}_D &= (\text{Conc SW}) (\text{TW}) \quad \text{Concrete Only!}\\
                          &= (51psf)(9ft) \\
                w^{(1)}_D &= \framebox{459lb/ft} \\
                \\
                w^{(1)}_L &= (\text{CL}) (\text{TW}) \quad \text{Construction Load}\\
                          &= (20psf)(9ft) \\
                w^{(1)}_L &= \framebox{180lb/ft} \\
                \\
                w^{(1)}_U &= \lambda _D w^{(1)}_D + \lambda _L w^{(1)}_L \\
            \end{aligned}
        \end{equation*}
    \item Selecting $w^{(1)}_U$: \footnote{Should we list all of the load combinations here or only the two main combinations?} 
        \begin{equation*}
            \begin{aligned}
                w^{(1)}_U &= max
                    \begin{cases}
                        1.4(w^{(1)}_D) + 0.0(w^{(1)}_L) = 642.6lb/ft \\
                        1.2(w^{(1)}_D) + 1.6(w^{(1)}_L) = 838.8lb/ft \\
                    \end{cases} \\
                w^{(1)}_U &= \framebox{0.839k/ft}
            \end{aligned}
        \end{equation*}
    \item Finding Moment Demand, $M^{(1)}_u$ (ignoring self weight):
        \begin{equation*}
            \begin{aligned}
                M^{(1)}_u   &= \frac{{w^{(1)}_u}L^2}{8} \\
                            &= \frac{(0.839k/ft)(30ft)^2}{8} \\
                M^{(1)}_u   &= \framebox{94.4k-ft}
            \end{aligned}
        \end{equation*}
    \item Finding Shear Demand, $V^{(1)}_u$ (ignoring self weight):
        \begin{equation*}
            \begin{aligned}
                V^{(1)}_u   &= \frac{{w^{(1)}_u}L}{2} \\
                            &= \frac{(0.839k/ft)(30ft)}{2} \\
                V^{(1)}_u   &= \framebox{12.6k}
            \end{aligned}
        \end{equation*}
    \item Since the deck ribs are oriented perpendicular to provides lateral bracing ($L_b = 0$), we can use table 3-2 to find a section with adquate flexural strength ($\Phi _b M_p$) 
        \newline
        Selecting W10X22, 
            \begin{equation*}
                \begin{aligned}
                    \Phi _b M_p = &97.5k-ft &\geq M^{(2)}_u = &94.4k-ft &\checkmark\\
                    \Phi _v V_n = &73.4k &\geq V^{(2)}_u = &12.6k &\checkmark
                \end{aligned}
            \end{equation*}
\end{enumerate}
\subsection*{Factoring in self-weight}
\begin{enumerate}
    \item Finding Distributed Loads:
        \begin{equation*}
            \begin{aligned}
                w^{(2)}_D   &= (\text{Conc SW}) (\text{TW}) + (\text{Beam SW}) \\
                            &= 459lb/ft + 22lb/ft \\
                            &= \framebox{481lb/ft} \\
                w^{(2)}_L &= (\text{CL}) (\text{TW}) \\
                            &= (20psf)(9ft) \\
                w^{(2)}_L &= \framebox{180lb/ft} \\
                w^{(2)}_U   &= \lambda _D w^{(2)}_D + \lambda _L w^{(2)}_L \\
            \end{aligned}
        \end{equation*}
    \item Selecting $w^{(2)}_U$:
    \begin{equation*}
        \begin{aligned}
            w^{(2)}_U &= max
                \begin{cases}
                    1.4(w^{(2)}_D) + 0.0(w^{(2)}_L) = 673.4lb/ft \\
                    1.2(w^{(2)}_D) + 1.6(w^{(2)}_L) = 865.2lb/ft \\
                \end{cases} \\
            w^{(2)}_U &= \framebox{0.865k/ft}
        \end{aligned}
    \end{equation*}
    \item Finding Moment Demand, $M^{(2)}_u$ (ignoring self weight):
        \begin{equation*}
            \begin{aligned}
                M^{(2)}_u   &= \frac{{w^{(1)}_u}L^2}{8} \\
                            &= \frac{(0.865k/ft)(30ft)^2}{8} \\
                M^{(2)}_u   &= \framebox{97.3k-ft}
            \end{aligned}
        \end{equation*}
    \item Finding Shear Demand, $V^{(2)}_u$ (ignoring self weight):
        \begin{equation*}
            \begin{aligned}
                V^{(2)}_u   &= \frac{{w^{(1)}_u}L}{2} \\
                            &= \frac{(0.865k/ft)(30ft)}{2} \\
                V^{(2)}_u   &= \framebox{13.0k}
            \end{aligned}
        \end{equation*}
    \item Checking Beam Capacity Against Demand: \\
    For Selected W10X22,
        \begin{equation*}
            \begin{aligned}
                \Phi _b M_p = &97.5k-ft &\geq M^{(2)}_u = &97.3k-ft &\checkmark\\
                \Phi _v V_n = &73.4k &\geq V^{(2)}_u = &13k &\checkmark
            \end{aligned}
        \end{equation*}
    At this point, you could check for deflections using unfactored loads! Think of using D + L to check for L/360, L/600, or L/240 depending on beam location! 
    \newline
    Now we have a section that at least works for construction. Under service loads, this won't work unless we consider composite action.
\end{enumerate}
\subsection*{Factoring in service and required demand}
\begin{enumerate}
    \item Finding Distributed Loads:
        \begin{equation*}
            \begin{aligned}
                w^{(3)}_D   &= (\text{SIDL + Conc SW}) (\text{TW}) + (\text{Beam SW}) \\
                            &= (30psf + 51 psf)(9ft) + 22lb/ft \\
                            &= \framebox{751lb/ft} \\
                w^{(3)}_L &= (\text{LL}) (\text{TW}) \\
                            &= (80psf)(9ft) \\
                w^{(3)}_L &= \framebox{720lb/ft} \\
                w^{(3)}_U   &= \lambda _D w^{(3)}_D + \lambda _L w^{(3)}_L \\
            \end{aligned}
        \end{equation*}
    \item Selecting $w^{(3)}_U$:
    \begin{equation*}
        \begin{aligned}
            w^{(3)}_U &= max
                \begin{cases}
                    1.4(w^{(3)}_D) + 0.0(w^{(3)}_L) = 1051lb/ft \\
                    1.2(w^{(3)}_D) + 1.6(w^{(3)}_L) = 2053lb/ft \\
                \end{cases} \\
            w^{(3)}_U &= \framebox{2.05k/ft}
        \end{aligned}
    \end{equation*}
    \item Finding Moment Demand, $M^{(3)}_u$:
        \begin{equation*}
            \begin{aligned}
                M^{(3)}_u   &= \frac{{w^{(3)}_u}L^2}{8} \\
                            &= \frac{(2.05k/ft)(30ft)^2}{8} \\
                M^{(3)}_u   &= \framebox{231k-ft}
            \end{aligned}
        \end{equation*}
    \item Finding Shear Demand, $V^{(3)}_u$:
        \begin{equation*}
            \begin{aligned}
                V^{(3)}_u   &= \frac{{w^{(3)}_u}L}{2} \\
                            &= \frac{(2.05k/ft)(30ft)}{2} \\
                V^{(3)}_u   &= \framebox{30.8k}
            \end{aligned}
        \end{equation*}
    \item Using moment demand ($M^{(2)}_u$), we can design the number and type of studs required:
\end{enumerate}
\subsection*{Determining Composite Action (Attempt 1)}
\begin{enumerate}
    \item To start off, we need to first assume $Y_2$ (distance from top steel to concrete flange force), then we can iterate:
        \begin{equation*}
            \begin{aligned}
                Y_2 &= Y_{conc} - \frac{a}{2} \quad \text{a is a function of $\sum Q_n$} \\
                Y_{conc} &= 3in\text{(deck)}  + 2.5in\text{(slab)} = 5.5in \\
                &\framebox{\text{Assume a = 0, and round down to use $Y_2 = 5in$}} 
            \end{aligned}
        \end{equation*}
    \item Using Table 3-19, we need to find $\sum Q_n$ such that:
        \begin{equation*}
            \begin{aligned}
               \Phi _b M_n \geq  M^{(3)}_u \quad &\text{Or}  \quad &\Phi _b M_n \geq 231k-ft \\
               \Phi _b M_n = 246k-ft \quad &\text{For} \quad &\sum Q_{n,required} = 325k
            \end{aligned}
        \end{equation*}
        Note that this implies full composite action ($\sum Q_n = F_yA_g$ and PNA = TFL - top of flange) abd that may be difficult to satisify. 
        \newline
        Because of our deck orientation, we have a fixed stud spacing. For 3VLI, this is 12" according to the catalog.
    \item Finding number of ribs required to develop $\sum Q_n$ over half of beam (Distance from $M_{max}$ to zero moment):\footnote{$N_{sa}$ represents the number of shear anchors/studs required.}
        \begin{equation*}
            \begin{aligned}
                N_{sa} \text{ ribs} &= \frac{L/2}{\text{Stud Spacing}} \\
                                    &= \frac{30/2ft}{1ft} \\
                                    &= \framebox{15 \text{ribs}}
            \end{aligned}
        \end{equation*}
    \item Finding demand per rib:\footnote{When looking at Table 3-21 - Deck Condition = Perpendicular $\rightarrow$ Normal Weight Concrete: For 1 stud per rib, we can't satisify this required strength of 21.7k/rib (all values listed in that row are lower than demand/rib)}
        \begin{equation*}
            \begin{aligned}
                \frac{\sum Q_n}{n \text{ ribs}} &= \frac{325k}{15 \text{ ribs}} \\
                                                &= \framebox{21.7 \text{k/Rib}}
            \end{aligned}
        \end{equation*}
    \item Selecting Shear Studs:\footnote{See Sec.I8.1 for maximum usable stud size - $d_{sa} = 2.5t_f \leq 3/4in$} \\
        Selecting Maximum Usable Stud Size (Sec.I8.1): 
            \begin{equation*}
                \begin{aligned}
                    d_{sa} &= 2.5t_f \leq 3/4in \\
                            &= 2.5(0.360in) \leq 3/4in \\
                            &= 0.9in \leq 0.75in \\
                    d_{sa} &= \framebox{0.75in}
                \end{aligned}
            \end{equation*}
        From this, we can \emph{attempt} to use 3/4in $\Phi$ studs, resulting in a smaller $\sum Q_n$. Alternatively, we could try multiple studs per rib, but this might not work since the flange is narrow, or it may be cost prohibitive.  
    \item Finding $\sum Q_n$ for selected shear studs: \\
        From Table 3-21, for 3/4in $\Phi$ studs with $f'c = 4ksi$, we have $Q_n = 17.2k$ at 1 stud per rib.
            \begin{equation*}
                \begin{aligned}
                    \sum Q_{n,provided} &= N_{sa} Q_n \\
                                    &= 15(17.2k) \\
                                    &= \framebox{258k}
                \end{aligned}
            \end{equation*}
        Comparing from Table 3-19, this will not work! 
            \begin{equation*}
                \begin{aligned}
                    \sum Q_{n,required} &> \sum Q_{n,provided} \\
                    273k-ft &> 258k-ft
                \end{aligned}
            \end{equation*}
        Since this doesn't work there are a couple of options:
            \begin{enumerate}
                \item Choose a larger beam (heavier or deeper, or both)
                \item Increase the slab depth.
            \end{enumerate}
        This next section will chose to select a larger beam.
\end{enumerate}
\subsection*{Determining Composite Action (Attempt 2)}
Continuing from the previous section, lets go up to a heavier beam, a W10X26.
\begin{enumerate}
    \item Checking Beam Capacity Against (self-weight) Demand: \\
    For Selected W10X26,
        \begin{equation*}
            \begin{aligned}
                \Phi _b M_p = &117k-ft &\geq M^{(2)}_u = &97.9k-ft &\checkmark\\
                \Phi _v V_n = &80.3k &\geq V^{(2)}_u = &13.1k &\checkmark
            \end{aligned}
        \end{equation*}
    \item Finding (service/ultimate) Moment Demand, $M^{(3)}_u$: \\
        \textbf{Note that the steps here in Wiebe's notes are skipped in finding M3!}
        \begin{equation*}
            \begin{aligned}
                M^{(3)}_u   &= \framebox{232k-ft} \rightarrow \text{ used to design studs}
            \end{aligned}
        \end{equation*}
    \item Finding (service/ultimate) Shear Demand, $V^{(3)}_u$: \\
        \textbf{Note that the steps here in Wiebe's notes are skipped in finding V3!}
        \begin{equation*}
            \begin{aligned}
                V^{(3)}_u   &= \framebox{30.9k} \leq \Phi _b V_n = 80.3k
            \end{aligned}
        \end{equation*}
    \item Finding $\sum Q_{n,required}$ for $Y_2 = 5in$ so that  $\Phi _b M_n \geq M^{(3)}_u$ from Table 3-19:
        \begin{equation*}
            \begin{aligned}
                \sum Q_{n,required} = \framebox{254k} & \text{ for } & \Phi _b M_n = 241k-ft
            \end{aligned}
        \end{equation*}
        Note that for the stud configuration we computed eariler (15 - 3/4in $\Phi$), $\sum Q_{n,provided} = 258k$ - so we can use it here.
    \item Computing $b_e$: 
        \begin{equation*}
            \begin{aligned}
                b_e &= min
                \begin{cases}
                    2(L/8) = 2(30ft/8) = 7.5ft\\
                    9ft
                \end{cases} \\
                    &= 7.5 ft \\
                b_e &= \framebox{90in} 
            \end{aligned}
        \end{equation*}
    \item Computing compression block depth, a for $Y_2$:
        \begin{equation*}
            \begin{aligned}
                a &= \frac{\sum Q_{n,provided}}{0.85f'cb_e} \\
                    &= \frac{258k}{0.85(4ksi)(90in)} \\
                    &= 0.843in
            \end{aligned}
        \end{equation*}
    \item Computing $Y_2$:
        \begin{equation*}
            \begin{aligned}
                Y_2 &= Y_{conc} - \frac{a}{2} \\
                    &= 5.5in - \frac{0.843in}{2} \\
                    &= \framebox{5.08in}
            \end{aligned}
        \end{equation*}
    \item To be continued! You shouldn't be computing Y2 if you want to avoid double-interpolation. Include calculations for the Y(PNA) to compute Mn and also deflection checks here.
    \end{enumerate}
\end{document}
